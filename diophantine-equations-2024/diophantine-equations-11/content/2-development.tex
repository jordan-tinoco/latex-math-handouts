\section{Desarrollo}

\subsection{Definiciones}

\begin{definition.box}{Fracción continua generalizada}{}
    Definiremos a una fracción continua generalizada como una expresión de la forma:
    \[
        a_0 + \dfrac{b_0}{a_1 + \dfrac{b_1}{a_2 + \dfrac{b_2}{a_{n - 2} + \dfrac{\ddots}{a_{n - 1} + \dfrac{b_{n -1}}{a_n}}}}}
    \]
    donde los $a_i$ y $b_i$, con $i = 0,1, \ldots, n$, son números reales.
\end{definition.box}
Los $a_i$ y $b_i$ se llamarán términos de la fracción continua, es fácil concluir que, es posible encontrar fracciones continuas con una cantidad finita e infinita de términos.
Cuando todos los $b_i$ son iguales a 1 se forman un tipo de fracciones continuas importantes.

\begin{definition.box}{Fracción continua simple}{}
    Si todo $b_i = 1$ y para $i \geq 1$ se tiene que $a_i$ es positivo, entonces la expresión
    \[
        a_0 + \dfrac{1}{a_1 + \dfrac{1}{a_{n - 2} + \dfrac{\ddots}{a_{n - 1} + \dfrac{1}{a_n}}}}
    \]
    se llamará fracción continua simple, y la denotaremos por $[a_0; a_1, a_2, \ldots, a_n]$
\end{definition.box}
Es claro que para un número $[a_0] = a_0$ y para una fracción continua simple infinita se tiene $[a_0; a_1, a_2, \ldots]$.
Esto toma relevancia a partir del siguiente teorema.

\begin{theorem.box}{}{}
    Si $x$ es número racional, entonces $x$ se puede expresar como una fracción continua simple con una cantidad finita de términos.
\end{theorem.box}

Del teorema anterior podemos deducir el siguiente resultado.

\begin{corollary}
    Toda fracción continua simple con cantidad infinita de términos representa un número irracional.
\end{corollary}

Cabe mencionar que el corolario anterior implica que no todo real puede expresarse como una fracción continua.
Además, los números irracionales que sí pueden ser expresados presentan una característica, esto es que sus términos
se repiten de manera cíclica.

\begin{definition.box}{Fracción continua periódica}{}
    Definiremos como fracción continua periódica a una fracción continua simple infinita de la forma
    \[
        [a_0; a_1, a_2, \ldots, a_n, \overline{a_{n + 1}, a_{n + 2},\ldots, a_{n + k}}],
    \]
    donde $k \geq 1$, el periodo es la sucesión de términos $a_{n + 1}, a_{n + 2},\ldots, a_{n + k}$ y la longitud del periodo es $k$.
\end{definition.box}

Cuando se tiene que $n = 0$ diremos que la fracción continua $[\overline{a_0;a_1, \ldots, a_k}]$ es una fracción continua periódica pura.

Resulta que todo número irracional cuadrático puede ser expresado como una fracción continua periódica.
Donde un número irracional cuadrático es una raíz no racional de una ecuación cuadrática, es decir los números de la forma $\frac{a + b\sqrt {d}}{c}$.

\begin{definition.box}{Convergentes $c_k$}{}
    Los convergentes de la fracción continua simple $[a_0; a_1, a_2, \ldots]$ son las fracciones continuas simples finitas
    \[
        c_k = [a_0; a_1, a_2, \ldots, a_k],\ \forall k \in \positiveSet{\Z}.
    \]
\end{definition.box}

La definición de $c_k$ dice que son fracciones continuas finitas, entonces estos convergentes representan números racionales,
por tanto, podemos decir que $c_k = \frac{p}{q}$, este echo nos permite encontrar resultados importantes, como por ejemplo los siguientes teoremas.

\begin{theorem.box}{}{}
    Si $c_n = \frac{p_n}{q_n}$ es el $n-$ésimo convergente de la fracción continua simple $[a_0; a_1, a_2, \ldots]$, entonces
    \begin{align*}
        &p_0 = a_0, && q_0 = 1,\\
        &p_1 = a_1 p_0 + 1, && q_1 = a_1,\\
        &p_k = a_k p_{k - 1} + p_{k - 2},\ \forall k \geq 2 && q_k = a_k q_{k - 1} + q_{k - 2},\ \forall k \geq 2
    \end{align*}
\end{theorem.box}

\begin{theorem.box}{}{}
    Si $c_n = \frac{p_n}{q_n}$ es el $n-$ésimo convergente de una fracción continua simple infinita, entonces
    \[
        p_n q_{n - 1} - p_{n - 1}q_n = (- 1)^{n-1}, \ \text{con}\ k \in \Z^{\geq 1}.
    \]
\end{theorem.box}
Del teorema anterior es posible determinar que todo convergente de una fracción continua representa una fracción irreducible, este echo será de útilidad al momento de resolver ecuaciones diofánticas.
\begin{corollary}
    Si $c_k = \frac{p_k}{q_k}$ es un convergente de una fracción continua simple infinita, entonces $\mcd{p_k}{q_k} = 1$.
\end{corollary}


\subsection{Resolución de ecuación diofánticas}

Si consideramos la ecuación diofántica lineal $ax + by = c$
y consideramos la fracción $\frac{a}{b}$ y de donde su fracción
continua tiene una cantidad $n$ de términos, con lo cual $c_n = \frac{a}{b}$.
Del teorema 1.3 obtenemos que
\[
    a q_{n - 1} - b p_{n - 1} = (-1)^{n- 1}
\]
multiplicando esta expresión por $(-1)^{n - 1}c$ obtenemos
\[
    a \left[(-1)^{n - 1}q_{n - 1}c\right] - b \left[(-1)^{n - 1}p_{n - 1}c\right] = c
\]
con lo cual las soluciones de la ecuación estaría dada por
\begin{align*}
    x &= (-1)^{n - 1}q_{n - 1}c\\
    y &= (-1)^{n}p_{n - 1}c.
\end{align*}
Es decir, que al calcular $c_{n - 1}$ podemos resolver la ecuación diofántica lineal.

Sabemos que la ecuación de Pell
\[
    x^2 - d y^2 = 1
\]
siempre tiene una solución mínima $(x_0, y_0)$


\subsection{Ejercicios y problemas}

Ejercicios y problemas para el autoestudio.

\begin{exercise}
    Encontrar las fracciones continuas de las siguientes fracciones.
    \begin{multicols}{3}
        \begin{enumerate}
            \item $\dfrac{45}{37}$
            \item $\dfrac{51}{25}$
            \item $\dfrac{43}{38}$
            \item $\dfrac{120}{84}$
            \item $\dfrac{35}{14}$
        \end{enumerate}
    \end{multicols}
\end{exercise}

\begin{problem}
    Calcular el valor de
    \[
        \sqrt[8]{2207 - \frac{1}{2207 - \frac{1}{2207 - \ddots}}}.
    \]
    Expresar la respuesta con forma $\frac{a + b\sqrt{c}}{d}$ con $a,b,c,d \in \Z$.
\end{problem}