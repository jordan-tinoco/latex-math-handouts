\section{Desarrollo}

\begin{definition.box}{Fracción continua generalizada}{}
    Definiremos a una fracción continua generalizada como una expresión de la forma:
    \[
        a_1 + \dfrac{b_1}{a_2 + \dfrac{b_2}{a_3 + \dfrac{b_3}{a_{n - 2} + \dfrac{\ddots}{a_{n - 1} + \dfrac{b_{n -1}}{a_n}}}}}
    \]
    donde los $a_i$ y $b_i$, con $i =1,2, \ldots, n$, son números reales.
\end{definition.box}
Donde los $a_i, b_i$ se llamarán términos de la fracción continua.
Por lo cual, es posible encontrar fracciones continuas con una cantidad finita e infinita de términos.
Cuando todos los $b_i$ son iguales a 1 se forman un tipo de fracciones continuas importantes.

\begin{definition.box}{Fracción continua simple}{}
    Si todo $b_i = 1$ y para $i \geq 2$ se tiene que $a_i$ es positivo, entonces la expresión
    \[
        a_1 + \dfrac{1}{a_2 + \dfrac{1}{a_{n - 2} + \dfrac{\ddots}{a_{n - 1} + \dfrac{1}{a_n}}}}
    \]
    se llamará fracción continua simple, y la denotaremos por $[a_1; a_2, a_3, \ldots, a_n]$
\end{definition.box}
Es claro que para un número $[a_1] = a_1$ y para una fracción continua simple infinita se tiene $[a_1; a_2, a_3, \ldots]$.
Esto toma relevancia a partir del siguiente teorema.

\begin{theorem.box}{}{}
    Si $x$ es número racional, entonces $x$ se puede expresar como una fracción continua simple con una cantidad finita de términos.
\end{theorem.box}

Del teorema anterior podemos deducir el siguiente resultado.

\begin{corollary}
    Toda fracción continua simple con cantidad infinita de términos representa un número irracional.
\end{corollary}

\begin{definition.box}{Fracción continua periódica}{}
    Definiremos como fracción continua periódica a una fracción continua simple de la forma
    \[
        [a_1; a_2, a_3, \ldots, a_n, \overline{a_{n + 1}, a_{n + 2},\ldots, a_{n + k}}],
    \]
    donde $m \geq 1$, el periodo es la sucesión de términos $a_{n + 1}, a_{n + 2},\ldots, a_{n + k}$ y la longitud del periodo es $k$.
\end{definition.box}

Cuando se tiene que $n = 0$ diremos que la fracción continua $[\overline{a_1;a_2, \ldots, a_k}]$ es una fracción continua periódica pura.

\begin{definition.box}{Convergentes $c_k$}{}
    Dada la fracción continua infinita $[a_1; a_2, a_3, \ldots]$ se define la fracción
    \[
        c_k = [a_1; a_2, a_3, \ldots, a_k],\ \forall k \in \positiveSet{\Z},
    \]
    cono el convergente de la posición $k$.
\end{definition.box}

\begin{theorem.box}{}{}
    Si $c_n = \frac{p_n}{q_n}$ es el $n-$ésimo convergente de la fracción continua simple $[a_1; a_2, a_3, \ldots]$, entonces
    \begin{align*}
        p_1 = a_1,\ p_2 = a_2 p_1 + 1 \text{ y } p_n = a_n p_{n - 1} + p_{n - 2} \quad \forall n \geq 3\\
        q_1 = 1,\ q_2 = a_2 \text{ y } q_n = a_n q_{n - 1} + q_{n - 2} \quad \forall n \geq 3
    \end{align*}
\end{theorem.box}


\subsection{Ejercicios y problemas}

Ejercicios y problemas para el autoestudio.