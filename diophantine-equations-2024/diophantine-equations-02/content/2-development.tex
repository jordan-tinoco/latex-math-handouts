\section{Desarrollo}

Antes de iniciar el estudio de este método, recordemos algunas propiedades de las desigualdades numéricas.
Los axiomas elementales sobre desigualdades son los siguientes

\begin{enumerate}
    \item Dado un número real $x$, se tiene que $x > 0$, $x = 0$ o $x < 0$.
    \item Si $a > 0$ y $b > 0$, entonces $a + b > 0$ y $ab > 0$.
    \item Si $a > b$, entonces $a + c > b + c$.
\end{enumerate}
Todas las demás desigualdades se derivan de estos axiomas.
Como por ejemplo
    \begin{enumerate}
        \item Si $a > b$ y $c < 0$, entonces $ac < bc$.
        \item Si $0 < a < 1$, entonces $a^2 < a$.
        \item Si $|a| > 1$, entonces $a^2 > a$.
    \end{enumerate}
Asi como el siguiente teorema.
\begin{theorem}
    Si $x$ es un número real, entonces $x^2 \geq 0$.
    La igualdad se da si y solo si $x = 0$.
\end{theorem}


\subsection{Ejercicios y problemas}

Ejercicios y problemas para el autoestudio.

\begin{exercise}
    Hallar todas las duplas positivas $(x,y)$ tal que $x^3 - y^3 = xy + 61$.
\end{exercise}

\begin{exercise}
    Hallar todas las soluciones enteras $(x,y)$ de $x^3 + y^3 = (x + y)^2$.
\end{exercise}

\begin{exercise}
    Determine todas las parejas de enteros $(x, y)$ tales que $1 + 2^x + 2^{2x + 1} = y^2$.
\end{exercise}

\begin{exercise}
    Halle todas las soluciones $(w,x,y,z)$ de enteros positivos tales que
    \[
        x^2 + y^2 + z^2 + 2xy + 2x(z - 1) + 2y(z + 1) = w^2
    \]
\end{exercise}

\begin{exercise}
    Determinar todos los números enteros positivos $(x,y,z)$ que sean solución de las siguientes ecuaciones
    \begin{enumerate}
        \item $\left(1 + \dfrac{1}{x}\right)\left(1 + \dfrac{1}{y}\right)\left(1 + \dfrac{1}{z}\right) = 2$
        \item $xy + yz + zx - xyz = 2$
        \item $(x + y)^2 + 3x + y + 1 = z^2$
    \end{enumerate}
\end{exercise}

\begin{exercise}
    Encuentra todos los pares de números positivos $(a, b)$ tales que $ab^2 + b + 7$ divide a $a^2 b + a + b$.
\end{exercise}