\section{Desarrollo}

Antes de iniciar el estudio de este método, recordemos algunas propiedades de las desigualdades numéricas.
Considerando $a, b, c, d$ números reales, se tiene que
\begin{multicols}{2}
    \begin{enumerate}
        \item Si $a > b$, $c > 0$ entonces $ac > bc$.
        \item Si $a > b$, $c < 0$ entonces $ac < bc$.
        \item Si $a < b$ entonces $a + c < b + c$.
        \item Si $0 < a < 1$ entonces $a^2 < a$.
        \item Si $a > 1$ entonces $a^2 > a$.
    \end{enumerate}
\end{multicols}


\subsection{Ejercicios y problemas}

Ejercicios y problemas para el autoestudio.

\begin{exercise}
    Hallar todas las duplas positivas $(x,y)$ tal que $x^3 - y^3 = xy + 61$.
\end{exercise}

\begin{exercise}
    Hallar todas las soluciones enteras $(x,y)$ de $x^3 + y^3 = (x + y)^2$.
\end{exercise}

\begin{exercise}
    Determine todas las parejas de enteros $(x, y)$ tales que $1 + 2^x + 2^{2x + 1} = y^2$.
\end{exercise}

\begin{exercise}
    Halle todas las soluciones $(w,x,y,z)$ de enteros positivos tales que
    \[
        x^2 + y^2 + z^2 + 2xy + 2x(z - 1) + 2y(z + 1) = w^2
    \]
\end{exercise}

\begin{exercise}
    Determinar todos los números enteros positivos $(x,y,z)$ que sean solución de las siguientes ecuaciones
    \begin{enumerate}
        \item $\left(1 + \dfrac{1}{x}\right)\left(1 + \dfrac{1}{y}\right)\left(1 + \dfrac{1}{z}\right) = 2$
        \item $xy + yz + zx - xyz = 2$
        \item $(x + y)^2 + 3x + y + 1 = z^2$
    \end{enumerate}
\end{exercise}

\begin{exercise}
    Encuentra todos los pares de números positivos $(a, b)$ tales que $ab^2 + b + 7$ divide a $a^2 b + a + b$.
\end{exercise}