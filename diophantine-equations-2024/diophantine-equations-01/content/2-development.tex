\section{Desarrollo}

\subsection{Definiciones}

\begin{box.definition}{Ecuación Diofántica}{}
    Se llama {ecuación diofántica} o {ecuación diofantina} a cualquier ecuación polinomial con coeficientes enteros cuya
    solución se restringe únicamente a aquellos valores enteros que la satisfacen.
\end{box.definition}

Es decir, una expresión de la forma
\[
    a_1 x_1 + a_2 x_2 + \cdots + a_n x_n = b, \quad \text{con} \quad 1 \leq i \leq n,\ a_i, b \in \Z.
\]
Donde la $n$-upla de números enteros $(r_1, r_2, \ldots, r_n)$ hace la igualdad.
Es claro que una ecuación diofántica puede tener una o más $n$-upla que hagan la igualdad.

\begin{box.definition}{}{}
    A una $n$-upla, con $n$ entero, que satisface una ecuación diofántica, se le llama {solución} de la ecuación.
    Una ecuación diofántica con una o más soluciones se llama ecuación {soluble}, así también, una ecuación diofántica sin
    soluciones se llama ecuación {insoluble} o {irresoluble}.
\end{box.definition}

Una ecuación diofántica se dice que tiene una familia de soluciones cuando un conjunto de estas, o todas, puede ser expresada en función de uno o más parámetros enteros.
Como por ejemplo la ecuación $x^2 + 2y^2 = z^2$, vemos que tiene soluciones $(-2, 0, 2)$, $(-1, 2, 3)$, $(2, 4, 6)$, $\ldots$, las cuales podemos expresar como
\[
    \begin{cases}
        x = r^2 - 2\\
        y = 2r\\
        z = r^2 + 2
    \end{cases}
\]
donde $r$ es un número entero sin ninguna restricción, por lo que la ecuación diofántica tiene infinitas soluciones de esta forma.

Ahora, nos centraremos en el conjunto de métodos elementales para la resolución de Ecuaciones Diofánticas, a lo largo del curso
haremos uso de estos métodos, tales como factorización, identidades algebraicas, desigualdades, parametrización y congruencias.

\subsection{Método de factorización}

El método de factorización consiste transformar una expresión algebraica como producto de otras expresiones algebraicas, muchas veces de grado menor.
Como trabajamos en números enteros esto nos permite asociar estas expresiones con los divisores de otros números enteros.
Logrando así, obtener un conjunto de casos muchas veces más sencillos de resolver.

Es decir, si tenemos una expresión de la forma
\[
    (a_1 x_1 + a_2 x_2 + \cdots + a_n x_n)(b_1 x_1 + b_2 x_2 + \cdots + b_n x_n) \cdots (r_1 x_1 + r_2 x_2 + \cdots + r_n x_n) = k,
\]
donde $1 \leq j \leq n$ tenemos $a_j, b_j, \ldots, r_j, k \in \Z$.
Si $k_1, k_2, \ldots, k_m$ son los divisores de $k$, podemos asociar las expresiones del lado izquierdo con estos divisores, formando sistemas de ecuaciones como por ejemplo
\[
    \begin{cases}
        a_1 x_1 + a_2 x_2 + \cdots + a_n x_n = k_1\\
        b_1 x_1 + b_2 x_2 + \cdots + b_n x_n = k_2\\
        \quad \qquad \vdots\\
        r_1 x_1 + r_2 x_2 + \cdots + r_n x_n = k_p,
    \end{cases}
\]
claramente, dependerá del problema como escoger estas asociaciones.

Ahora, algunas identidades algebraicas nos pueden simplificar la factorización, aquí exponemos una pequeña lista de las más comunes.
\vspace{2mm}

\textbf{Identidades útiles}
\begin{align*}
    &\text{(Completación de rectángulo)} && xy + iy + jx + ij = (x + i)(y + j)\\[2mm]
    &\text{(Diferencia de cuadrados)} && a^2 - b^2 = (a - b)(a + b)\\[2mm]
    &\text{(Binomio al cuadrado)} && (a + b)^2 = a^2 + 2ab + b^2\\[2mm]
    &\text{(Trinomio al cuadrado)} && (a + b + c)^2 = a^2 + b^2 + c^2 + 2(ab + bc + ca)\\[1.5mm]
    &\text{(Identidad I)} && a^2 + b^2 + c^2 - ab - bc - ca = \frac{1}{2}\left[ (a - b)^2 + (b - c)^2 + (c - a)^2 \right]\\[1.5mm]
    &\text{(Identidad de Gauss)} && a^3 + b^3 + c^3 - 3abc = (a + b + c)(a^2 + b^2 + c^2 - ab - bc - ca)\\[2mm]
    &\text{(Identidad de Argand)} && a^4 + a^2 + 1 = (a^2 - a + 1)(a^2 + a + 1)\\[2mm]
    &\text{(Identidad de Sophie Germain)} && a^4 + 4 b^4 = (a^2 - 2ab + 2b^2)\left(a^2 + 2ab + 2b^2\right)\\[2mm]
    &\text{(Identidad de Brahmagupta)} && (a^2 + b^2) (x^2 + y^2) = (ax + by)^2 + (ay - bx)^2\\[2mm]
    &\text{(Diferencia de potencias)} && a^n - b^n = (a - b)(a^{n - 1} + a^{n - 2}b + \cdots + b^{n - 1})\\[2mm]
    &\text{(Suma de potencias)} && a^n + b^n = (a + b)(a^{n - 1} - a^{n - 2}b + \cdots + b^{n - 1}) \ \text{para $n$ impar}\\[1.5mm]
    &\text{(Binomio de Newton)} && (a + b)^n = a^n + \binom{n}{1} a^{n - 1}b + \cdots + \binom{n}{k} a^{n -k}b^k + \cdots + b^n
\end{align*}

A continuación, tenemos una serie de ejercicios con los cuales podemos prácticar y aplicar el método de factorización.


\subsection{Ejercicios y problemas}

Ejercicios y problemas para el autoestudio.

\begin{exercise}
    Hallar la cantidad de parejas de enteros positivos $x_1, x_2$ tales que $x_1 \cdot x_2 = 25 \cdot 15^3$.
\end{exercise}

\begin{exercise}
    Probar que la ecuación $x^2 + 2y^2 = z^2$ tiene como solución a $x = a^2 - 2b^2$, $y = 2ab$ y $z = a^2 + 2b^2$ donde $a, b \in \Z$.
\end{exercise}

\begin{exercise}
    Hallar todos los números naturales $x,y$ para los cuales $\dfrac{5}{x} + \dfrac{6}{y} = 1$.
\end{exercise}

\begin{exercise}
    Probar que la ecuación $x^3 = 2y^3$ no tiene soluciones enteras.
\end{exercise}

\begin{exercise}
    Resolver $y^2 = x^3 - x$ sobre los enteros.
\end{exercise}

\begin{exercise}
    Hallar los enteros positivos $x,y,z$ tal que $3^x + 4^y = z^2$.
\end{exercise}

\begin{exercise}
    Encuentre todos los enteros positivos $x, y$ tales que $xy - x + y = 49$.
\end{exercise}

\begin{exercise}
    Encuentre todas las soluciones enteras de la ecuación
    \[
        (x^2 + 1)(y^2 + 1) + 2(x - y)(1 - xy) = 4(1 + xy).
    \]
\end{exercise}

\begin{exercise}
    Encuentre todos los enteros positivos $n$ tales que, $n^4 + 4^n$ es primo.
\end{exercise}

\begin{exercise}
    ¿Para qué valores de $a$ y $b$ se da la igualdad $a^2 - 4ab = - 4b^2 + 9$?
\end{exercise}

\begin{exercise}
    Resuelve la siguiente ecuación en enteros $x^2 + 6xy + 8y^2 + 3x + 6y = 2$.
\end{exercise}

\begin{exercise}
    Para cada entero $n$ sea $s(n)$ el número de pares ordenados $(x, y)$ de enteros positivos para los cuales
    \[
        \frac{1}{x} + \frac{1}{y} = \frac{1}{n}.
    \]
    Encuentre todos los números enteros positivos $n$ para los cuales $s(n) = 5$.
\end{exercise}

\begin{exercise}
    Hallar todas las soluciones enteras de la ecuación $x^3 + y^3 - 3xy = 3$.
\end{exercise}

\begin{exercise}
    Hallar todas las soluciones enteras de $x^2(y - 1) + y^2(x - 1) = 1$.
\end{exercise}

\begin{exercise}
    Sean p y q números primos distintos.
    Encuentre el número de pares de enteros positivos $x, y$ que satisfacen la ecuación
    \[
        \frac{p}{x} + \frac{q}{y} = 1.
    \]
\end{exercise}

\begin{problem}
    Encuentre todos los pares de enteros $(x, y)$ tales que $x^6 + 3x^3 + 1 = y^4$.
\end{problem}

\begin{problem}
    Encuentre todos los pares $(x, y)$ de enteros tales que $xy + \dfrac{x^3+ y^3}{3} = 2007$.
\end{problem}