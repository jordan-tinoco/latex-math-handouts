\newpage
\section{Notas de clase}

\subsection{¿Qué?}

Mostrar el uso de congruencias en enteros para la solución de ecuaciones diofánticas, dando énfasis a las ecuaciones insolubles
y como las congruencias nos permiten probarlo.
También, mostrar como las congruencias dan información de las soluciones, por medio de condiciones que deben cumplirse.

\subsection{¿Cómo?}

Problema para motivar el tema.

\begin{activity}[Ejercicio 3][10 min]
    Hallar los enteros positivos $a,b$ tales que $a^2 - 3b^2 = 8$.
\end{activity}

\textit{Primero dar tiempo para intentar resolverlo por medio de los métodos anteriores y mostrar la dificultad que esto conlleva.
Para resolverlo, solo basta analizar la ecuación en módulo 3.}

\begin{activity}[Ejercicio 8][8 min]
    Ejercicio para los estudiantes.
    Este también se resuelve con un análisis en módulo 3.
\end{activity}

\begin{activity}[Ejercicio 10][15 min]
    Ejemplo.
    Se analiza la ecuación en módulo 9, obteniendo $a^3 + 2b^3 + 4c^3 \modulo{0}{9}$ lo cual no es posible porque no se
    puede formar una combinación 0 y $\pm1$ (que son los restos cúbico módulo 9) de tal manera que se obtenga cero.
\end{activity}