\section{Desarrollo}

\subsection{Congruencia en entero}
Veamos algunas definiciones importantes.
\begin{definition}[Divisibilidad]
    Si $a$ y $b$ son enteros, se dice que $a$ divide a $b$ o que $b$ es múltiplo de $a$ si $b = aq$ para algún entero $q$,
    y se denota por $a \mid b$.
\end{definition}

\begin{definition}[Congruencias]
    Dados dos enteros $a$, $b$ y un entero positivo $m$, decimos que $a$ es congruente con $b$ módulo $m$ si $(a - b)$ es múltiplo de $m$.
    En este caso escribimos $a \modulo{b}{m}$.
\end{definition}

\begin{theorem}[Propiedades de Congrencia]
    Sean los enteros $a,b,c,d,m$ con $m \geq 1$.
    \begin{enumerate}
        \item Si $a \modulo{c}{m}$ y $c \modulo{d}{m}$, entonces $a \modulo{d}{m}$.
        \item Si $a \modulo{c}{m}$ y $b \modulo{d}{m}$, entonces $ab \modulo{cd}{m}$.
        \item Si $a \modulo{c}{m}$, entonces $a^n \modulo{c^n}{m}$ para todo entero positivo $n$.
        \item Si $ab \modulo{bc}{m}$, entonces $a \modulo{c}{\frac{m}{(b,m)}}$ donde $(b,m)$ denota el máximo común divisor de $b$ y $m$.
    \end{enumerate}
\end{theorem}

\begin{theorem}[Pequeño teorema de Fermat]
    Si $p$ es primo y $a$ es un entero primo relativo con $p$, entonces $a^{p - 1} \modulo{1}{p}$.
\end{theorem}



\subsection{Ejercicios y problemas}
Ejercicios y problemas para el autoestudio.