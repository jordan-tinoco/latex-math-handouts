\section{Desarrollo}
Antes de empezar recordemos lo siguiente.

\begin{definition.box}{Máximo divisor común}{}
    Dados $a,b \in \Z^{\neq 0}$, el máximo $d \in \positiveSet{\Z}$ tal que $d \mid a$ y $d \mid b$ es el máximo divisor común,
    y lo denotamos por $d = \mcd a b.$
\end{definition.box}

\begin{definition.box}{}{}
    Dados $a,b \in \Z^{\neq 0}$, si $\mcd a b = 1$, entonces diremos que $a$ y $b$ son coprimos, primos relativos o primos entre sí.
\end{definition.box}

\begin{theorem.box}{Algoritmo de la división}{}
    Dados $a, b \in \Z$ con $b \neq 0$, existen $k,r \in \Z$ tales que $a = kb + r$ con $0 \leq r < |b|$, donde $k,r$ son únicos.
\end{theorem.box}

\begin{theorem.box}{}{}
Sean $a,b,c \in \Z$ con $a,b \neq 0$, la ecuación $ax + by = c$;
\begin{enumerate}
    \item tiene solución si y solo si $\mcd a b \mid c$, además, es equivalente a otra ecuación de coeficientes coprimos,
    \item si $(x_0,y_0)$ es una solución particular, entonces la solución general es
    \[
        x = x_0+\frac{b}{d}t, \quad y = y_0 -\frac{a}{d}t, \quad t \in \Z.
    \]
    \item si $c = \mcd a b$ y $|a|, |b| \neq 1$, entonces una solución particular $(x_0,y_0)$ cumple que $|x_0| < |b|$ y $|y_0| < |a|$.
\end{enumerate}
\end{theorem.box}

\begin{example}
    Resolver la ecuación $5x - 3y = 52$ en enteros positivos.
\end{example}
\begin{solution}
    Primero, como $\mcd 5 3 = 1$ y $1 \mid 52$, entonces la ecuación tiene soluciones enteras.
    Ahora, analizando en módulo 5 tenemos que
    \[
        - 3y \modulo{52}{5} \implies 2y \modulo{2}{5} \implies y \modulo{1}{5}.
    \]
    Claramente $y = 1$ es solución a esta congruencia, sustituyendo en la ecuación original $5x - 3\cdot 1 = 52 \iff 5x = 55$
    por lo cual, la ecuación tiene una solución $(11, 1)$, luego con la solución $(x_0,y_0)=(11,1)$ llegamos a
    \[
        (x,y) = (11+3t,1-5t), \mbox{ donde } t\in \Z.\qedhere
    \]
\end{solution}

\begin{example}
    Resolver la siguiente ecuación $8c+7p=100$.
\end{example}
\begin{solution}
    Claramente la ecuación tiene soluciones enteras, analizando en módulo 8,
    \[
        7p \modulo{100}{8} \implies -p \modulo{4}{8} \implies p \modulo{-4}{8} \implies p \modulo{4}{8}.
    \]
    Rápidamente, podemos decir que $p = 4$ es una solución para dicha congruencia, sustituyendo en la ecuación obtenemos $c = 9$.
    Luego, con la solución $(c_0, p_0) = (9, 4)$ tenemos
    \[
        (c,p) = (9 + 7t, 4 - 8t),\quad \text{con}\ t\in \Z.\qedhere
    \]
\end{solution}



\subsection{Aplicando el algoritmo de Euclides}

\begin{definition.box}{Algoritmo de Euclides}{}
    Sean $a, b \in \Z^{\neq 0}$, si $a > b$ y se tiene $a = bk + r$ con $0 < r < b$, entonces
    \[
        \mcd a b = \mcd b r.
    \]
\end{definition.box}

Podemos usar este algoritmo para resolver ecuaciones lineales de una manera iterativa.

Sean $a,b, c \in \Z$ con $a,b \neq 0$ y $d = \mcd a b$, considerando la ecuación
\[
    ax + by = c.
\]
\begin{enumerate}
    \item Si $d\nmid c$, entonces no hay solución.
    \item Si $d \mid c$, entonces se divide la ecuación por $d$.
    \item Por el paso anterior la ecuación tiene coeficiente coprimos;
    \begin{enumerate}
        \item Si $a \mid c$, entonces $ac_0 = c$, luego $(x,y) = (c_0, 0)$ es solución.
        \item Si $a \nmid c$, entonces tomamos el menor de $|a|, |b|$ (supongamos que es $|a|$) y obtenemos
        \begin{align*}
            b = aq_1 + r_1,\ \text{con}\ 0<r_1<|a|, && c = aq_2 + r_2,\ \text{con}\ 0 < r_2 < |a|.
        \end{align*}
    \end{enumerate}
    \item Sustituimos en la ecuación
    \[
        ax + (a q_1 + r_1) y = aq_2 + r_2 \iff a(x + q_1 y -q_2) + r_1 y = r_2.
    \]
    Si $z = x + q_1 y - q_2$, entonces la ecuación anterior se transforma en $az + r_1 y = r_2$.
    \begin{enumerate}
        \item Si $r_1 \mid r_2$, entonces terminamos con el paso 3.a.
        \item Si $r_1 \nmid r_2$, entonces vamos al paso 3.b y repetimos el proceso.
    \end{enumerate}
\end{enumerate}

\begin{example}
    Resuelva la siguiente ecuación $350x + 425y = 1200$.
\end{example}
\begin{solution}
    Como $(350,425)=25$ y $25 \mid 1200$, dividimos ambos lados de la ecuación por $25$ y tenemos
    \[
        14x+17y=48
    \]
    Por el algoritmo de euclides se tiene que $17 = 1\cdot14+3$ y $48=3\cdot14+6$.
    Sustituyendo y agrupando, tenemos $14(x+y-3)+3y=6.$
    Haciendo $z=x+y-3$ y sustituyendo en esta última ecuación se tiene $14z+3y=6.$
    Como $3\mid6$, para esta ecuación tenemos una solución de la forma $z=0$ y $y=2$.
    Escribiendo $z$ en términos de $x$ y $y=2$, obtenemos el valor de $x=1$.
    Luego, la solución general de la ecuación inicial es:
    \[
        x=1+17k, \quad y=2-14k, \quad k \in \Z. \qedhere
    \]
\end{solution}

Hasta el momento sólo hemos trabajado en ecuaciones lineales de dos variables, pero en realidad la ecuación $ax+by=c$ no es más que un caso particular de la ecuación
\[
    a_1 x_1 + a_2 x_2 + \ldots + a_n x_n = c,
\]
donde $a_1, a_2, \dots, a_n,$ y $c$ son coeficientes.

\begin{theorem.box}{}{}
    La ecuación $a_1 x_1 + a_2 x_2 + \ldots + a_n x_n = c$ tiene solución si y solo si $\mcd {a_1}{a_2,\cdots,a_n} \mid c.$
\end{theorem.box}

\begin{example}
    Resuelva la ecuación $3x+4y+5z=6$
\end{example}
\begin{solution}
    Primero garantizamos que $(3,4,5)=1$ efectivamente divide a 6.
    Trabajando con módulo 5 tenemos que $3x+4y\modulo{1}{5}$, y de esto que $3x+4y=1+5s, \quad s \in \Z.$
    Una solución para esta ecuación es $x=-1+3s$, $y=1-s$.
    Usando el resultado anterior, obtenemos $x=-1+3s+4t$, $y = 1-s-3t$, con $t \in \Z$, y sustituyendo en la ecuación original $z=1-s$.
    Por lo que todas las soluciones son
    \[
        (x,y,z)=(-1+3s+4t,1-s-3t,1-s), \quad s,t \in \Z.\qedhere
    \]
\end{solution}

\begin{definition.box}{}{}
    Sean  $a_1,a_2,\ldots,a_n$ enteros positivos con $\mcd{a_1}{a_2,\ldots,a_n} = 1$ se define a $g(a_1,a_2,\ldots,a_n)$
    como el mayor entero positivo $N$ para el cual
    \[
        a_1 x_1 + a_2 x_2 + \ldots + a_n x_n = N,
    \]
    no tiene soluciones enteras.
\end{definition.box}
El problema de determinar $g(a_1,a_2,\ldots,a_n)$ es conocido como el problema de las monedas de Frobenius.
Fue este quien planteó el problema de encontrar la mayor cantidad de dinero que no se puede pagar con monedas de $a_1, a_2,\ldots,a_n$ centavos.
El ejemplo clásico de este problema es que con monedas de 3 y 5 centavos nunca se podrá llegar a la cantidad de 7 centavos.

\begin{theorem.box}{Sylvester}{}
Sean $a,b \in \positiveSet{\Z}$, con $\mcd a b = 1$, entonces
    \[
        g(a,b) = ab - a - b.
    \]
\end{theorem.box}

Para el caso de $n=2$, existe el Teorema de Sylvester el cual nos brinda el valor de $N$
(rápidamente podemos verificar el ejemplo anterior con 3 y 5),
pero para $n\geq2$ no se conoce ninguna fórmula explicíta, aunque sí se han encontrado rangos en donde $N$ puede estar.
Estos caso para $n\geq2$ sobrepasan los objetivos de este escrito, se invita al lector investigar este tema por su cuenta.

Finalmente por con teorema de Sylvester podemos entender de mejor manera el siguiente teorema.

\begin{theorem.box}{Chicken McNugget}{}\label{chickenMcNugget}
    Sean $a, b\in \positiveSet{\Z}$ con $\mcd a b = 1$, se tiene
    \begin{enumerate}
        \item Si $n = ab - a - b$, entonces $ax + by = n$ es insoluble $\forall x, y \in \positiveSet{\Z}$.
        \item Si $n > ab - a - b$, entonces la ecuación es soluble.
    \end{enumerate}
\end{theorem.box}

Como recomendación general, se deja que siempre se verifique que una ecuación diofántica lineal de dos variables
cumpla con el \textbf{Teorema \ref{chickenMcNugget}} punto 2.




\subsection{Ejercicios y problemas}

Ejercicios y problemas para el autoestudio.

\begin{exercise}
    ¿Tiene la ecuación $24x+18y=12$ soluciones enteras?
\end{exercise}

\begin{exercise}
    Resuelva la ecuación diofántica $125x-25y=28.$
\end{exercise}

\begin{exercise}
    Resuelva la ecuación $69x+123y=3000.$
    (ejemplo donde se utiliza la recursividad del método de euclides.)
\end{exercise}