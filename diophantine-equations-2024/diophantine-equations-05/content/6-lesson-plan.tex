\newpage
\section{Plan de clase}

\subsection{¿Qué?}

Dar a conocer los principios de Inducción Matemática y Descenso al infinito de Fermat, además de mencionar el principio del Buen orden.
Dando énfasis en el uso de estos principios para la solución de ecuaciones diofánticas, tanto solubles como insolubles.
Mostrar ejemplos concretos de estos principios.

\subsection{¿Cómo?}

\begin{activity}[*][8 min]
    Explicar inducción simple, utilizando la analogía de dóminos.
    Hacer énfasis en el vocabulario y en el objectivo principal de la inducción ``El gran problema de la matemática es acceder al infinito con recursos finitos''.
\end{activity}

\begin{activity}[*][10 min]
    Explicar inducción fuerte, explicar que un paso inductivo puede dar ``saltos'' [$P(n) \implies P(n + k)$] o
    depender de uno o más pasos anteriores [$\left(P(1) \land P(2) \land \ldots \land P(n)\right) \implies P(n + 1)$].
\end{activity}

\begin{activity}[Ejemplo 1.1][15 min]
    Leer el ejemplo 1.1, escribir y explicar la solución del ejemplo, además de hacer énfasis en cómo se aplica el vocabulario.
    Comentar que la tésis de inducción generalmente se obtiene del enunciado, que el paso inductivo es muchas veces lo
    más complicado en la inducción y en ciertos problemas hay que buscar el paso base.
    Claramente, para problemas más difíciles esto tiene más complejidad.
\end{activity}

\begin{activity}[Ejercicio 1][10 min]
    Dar unos 3 minutos de intento y luego pasar a la pizarra, que todos los estudiantes participen.
    \par\hspace{5mm}\textbf{Solución.}
    Notar que a partir de $n = 20$ la propiedad se empieza a cumplir, se hacen los primeros 5 o 6 casos que cumple y
    luego es análogo al ejemplo 1.1.
\end{activity}

\begin{activity}[*][10 min]
    Explicar el principio del buen orden, dar los resultados importantes y mostrar un ejemplo con los conjuntos del ejercicio 7 (1, 2, 3).
\end{activity}

\begin{activity}[*][15 min]
    Explicar descenso infinito, utilizar la analogía de una escalera o bien la analogía de una reflexión entre espejos
    hacía el infinito.
    Mostrar la similitud con la inducción.
    Mostrar las variantes.
\end{activity}

\begin{activity}[Ejemplo 1.2][10 min]
    Leer el ejemplo, escribir y explicar la solución por contradicción.
    Hacer énfasis en como se llega a una situación absurda (justificada por el descenso infinito) cuando se asume que existe una solución.
    Además, resolver el ejercicio por medio de un argumento de buen order.
    \par\hspace{5mm}\textbf{Solución 2.} Asumir que existe una solución mínima en el conjunto de soluciones, la cual
    produce otra solución mínima contradiciendo la minimalidad de la primera, luego no hay soluciones.
\end{activity}

\begin{activity}[Ejercicio 2][10 min]
    Se nota que las variables deben ser pares, luego una solución implicaría otra solución más pequeña y asi sucesivamente.
    Se puede usar módulo 2.
\end{activity}

\begin{activity}[Ejercico 3][10 min]
    Usando módulo 3, notamos que $a,b$ son múltiplos de 3, luego una solución implicaría otra solución más pequeña y asi sucesivamente.
\end{activity}

\begin{activity}[Ejercico 4][10 min]
    Utilizar módulo 3 para obtener que dos de las variables son múltiplos de 3.
\end{activity}

\begin{activity}[Ejercico 5][10 min]
    Utilizar módulo 4 para obtener que dos de las variables son múltiplos de 4.
\end{activity}

\textbf{Definición de buen orden con simbología.}
\[
    A \subseteq \N \land A \neq \emptyset \implies \exists m \in A \ \backslash \ \forall n \in A,\ m \leq n.
\]


\newpage
\subsection{Comentarios}

Preguntas claves: ¿me entendieron?
¿me salté algún tema?
¿di tiempo suficiente para pensar los problemas?
¿participaron?
¿problemas muy fáciles o muy difíciles, demasiados o muy pocos?
¿las explicaciones/ejemplos fueron suficientes y buenos?

\foreach \x in {1,...,25}{
    \myhrule{8.4}
}