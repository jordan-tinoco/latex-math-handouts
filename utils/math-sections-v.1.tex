\theoremstyle{definition}

%With section
\newtheorem{section-lemma}{Lema}[section]
\newtheorem{section-theorem}{Teorema}[section]
\newtheorem{section-problem}{Problema}[section]
\newtheorem{section-exercise}{Ejercicio}[section]
\newtheorem{section-corollary}{Corolario}[section]
\newtheorem{section-definition}{Definición}[section]

%Without section
\newtheorem{case}{Caso}
\newtheorem{example}{Ejemplo}
\newtheorem{problem}{Problema}
\newtheorem{remark}{Observación}
\newtheorem{corollary}{Corolario}

%With numeration
\newtheorem*{note}{Nota}
\newtheorem*{definition}{Definición}


%Text Color Box styles
\newtcbtheorem[number within=section]{tcb-theorem-style}{Teorema}
{
    enhanced,
    frame empty,
    interior empty,
    coltitle = black,
    colbacktitle = white,
    fonttitle = \bfseries,
    extras broken = {frame empty, interior empty},
    borderline = {0.4mm}{0mm}{black},
    breakable = true,
    top = 4mm,
    before skip = 3.5mm,
    attach boxed title to top left = {yshift = -3mm, xshift = 3mm},
    boxed title style = {boxrule = 0.3mm, borderline = {0.3mm}{0mm}{black}},
    varwidth boxed title,
    separator sign none, description delimiters parenthesis,
    description font=\bfseries,
    terminator sign={.\hspace{1mm}}
}
{theorem.tcb-label}

\newtcbtheorem[number within=section]{tcb-problem-style}{Problema}
{
    enhanced,
    frame empty,
    interior empty,
    coltitle = black,
    colbacktitle = white,
    fonttitle = \bfseries,
    extras broken = {frame empty, interior empty},
    borderline = {0.4mm}{0mm}{black},
    breakable = true,
    top = 4mm,
    before skip = 3.5mm,
    attach boxed title to top left = {yshift = -3mm, xshift = 3mm},
    boxed title style = {boxrule = 0.3mm, borderline = {0.3mm}{0mm}{black}},
    varwidth boxed title,
    terminator sign={.\hspace{1mm}}
}
{problem.tcb-label}

\newtcbtheorem[number within=section]{tcb-exercise-style}{Ejercicio}
{
    enhanced,
    frame empty,
    interior empty,
    coltitle = black,
    colbacktitle = white,
    fonttitle = \bfseries,
    extras broken = {frame empty, interior empty},
    borderline = {0.4mm}{0mm}{black},
    breakable = true,
    top = 4mm,
    before skip = 3.5mm,
    attach boxed title to top left = {yshift = -3mm, xshift = 3mm},
    boxed title style = {boxrule = 0.3mm, borderline = {0.3mm}{0mm}{black}},
    varwidth boxed title,
    terminator sign={.\hspace{1mm}}
}
{exercise.tcb-label}

\newtcbtheorem[number within=section]{tcb-definition-style}{Definición}
{
    enhanced,
    frame empty,
    interior empty,
    coltitle = black,
    colbacktitle = white,
    fonttitle = \bfseries,
    extras broken = {frame empty, interior empty},
    borderline = {0.4mm}{0mm}{black},
    breakable = true,
    top = 4mm,
    before skip = 3.5mm,
    attach boxed title to top left = {yshift = -3mm, xshift = 3mm},
    boxed title style = {boxrule = 0.3mm, borderline = {0.3mm}{0mm}{black}},
    varwidth boxed title,
    separator sign none, description delimiters parenthesis,
    description font=\bfseries,
    terminator sign={.\hspace{1mm}}
}
{definition.tcb-label}

%Enviroments section-x.tcb
\newenvironment{section-theorem.tcb}[1][]
{
    \ifstrempty{#1}
    {
        \begin{tcb-theorem-style}{}{}
    }
    {
        \begin{tcb-theorem-style}{#1}{}
    }
    }{
    \end{tcb-theorem-style}
}

\newenvironment{section-problem.tcb}
{
    \begin{tcb-problem-style}{}{}
    }{
    \end{tcb-problem-style}
}

\newenvironment{section-exercise.tcb}
{
    \begin{tcb-exercise-style}{}{}
    }{
    \end{tcb-exercise-style}
}

\newenvironment{section-definition.tcb}[1][]
{
    \ifstrempty{#1}
    {
        \begin{tcb-definition-style}{}{}
    }
    {
        \begin{tcb-definition-style}{#1}{}
    }
    }{
    \end{tcb-definition-style}
}