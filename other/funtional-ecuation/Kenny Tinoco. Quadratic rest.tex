\documentclass[12pt]{article}

\usepackage{mlmodern}
\usepackage[T1]{fontenc}
\usepackage[spanish]{babel}

\usepackage{cite}
\usepackage{hyperref}
\usepackage{fancyhdr}
\usepackage[Lenny]{fncychap}
\usepackage{microtype}
\usepackage{makeidx}
\usepackage{bookmark}
\usepackage
[
    a5paper,
    left = 1cm,
    right = 1cm,
    top = 2cm,
    bottom = 1cm
]
{geometry}
\hypersetup
{
    colorlinks = true,
    linkcolor = black
}

\setlength{\parskip}{2mm}
\setlength{\parindent}{0pt}

\makeindex

\renewcommand{\qedsymbol}{$\blacksquare$}
\addto\captionsspanish{\renewcommand{\proofname}{\textnormal{\textbf{Demostración}}}}
\DeclareSymbolFont{yhlargesymbols}{OMX}{yhex}{m}{n}
\DeclareMathAccent{\wideparen}{\mathord}{yhlargesymbols}{"F3}
\DeclarePairedDelimiter{\abs}{\lvert}{\rvert}

%\renewcommand{\theenumi}{\alph{enumi}}

%Number sets
\newcommand{\N}{\ensuremath{\mathbb{N}}}
\newcommand{\Z}{\ensuremath{\mathbb{Z}}}
\newcommand{\Q}{\ensuremath{\mathbb{Q}}}
\newcommand{\R}{\ensuremath{\mathbb{R}}}
\newcommand{\C}{\ensuremath{\mathbb{C}}}


%Useful commands
\newcommand{\ds}{\displaystyle}
\newcommand{\ie}{\ensuremath{\text{i.e.}}}
\newcommand{\eg}{\ensuremath{\text{e.g.}}}
\renewcommand{\emptyset}{\varnothing}
\newcommand{\fullMod}[2]{\equiv #1 \pmod{#2}}
\newcommand{\mcd}[2]{\ensuremath{mcd(#1, #2)}}
\newcommand{\mcm}[2]{\ensuremath{mcm(#1, #2)}}

%Enviroments
\newenvironment{solution}[1][]
{
    \ifstrempty{#1}
    {
        \begin{proof}[\textnormal{\textbf{Solución}}]
    }
    {
        \begin{proof}[\textnormal{\textbf{Solución #1}}]
    }
    }{
    \end{proof}
}
\usepackage{ifthen}
\usepackage{tcolorbox}
\usepackage{varwidth}

\tcbuselibrary{theorems}
\tcbuselibrary{breakable}
\tcbuselibrary{skins}

\theoremstyle{definition}

\newtheorem{counter}{Contador}[section]
\newtheorem{problem}{Problema}[section]
\newtheorem{corollary}{Corolario}[chapter]
\newtheorem{example}{Ejemplo}[chapter]
\newtheorem{definition}{Definición}[chapter]
\newtheorem{remark}{Observación}
\newtheorem*{note}{Nota}

\newenvironment{solution}[1][]
{
    \begin{proof}[\textnormal{\textbf{Solución\ifthenelse{\equal{#1}{}}{}{ #1}}}]
    }{
    \end{proof}
}

\newtcbtheorem[number within=chapter]{theorem}{Teorema}
{
    enhanced
    ,colback = gray!10!white
    ,frame hidden
    ,boxrule = 0sp
    ,borderline west = {2.5pt}{0pt}{black}
    ,sharp corners
    ,attach title to upper
    ,coltitle = black
    ,fonttitle = \bfseries
    ,description font = \mdseries
    ,separator sign none,
    ,terminator sign={.\hspace{2mm}}
    ,description delimiters parenthesis,
    right=1mm,
    top=0mm,
    left=1.5mm,
    bottom=0mm,
    breakable = true
}{t}




\title{Restos cuadráticos y Símbolo de Legendre}
\author{Kenny J. Tinoco}
\date{Octubre de 2024}

\begin{document}
   \maketitle
   \begin{definition.box}{Símbolo de Legendre}{}
      Sea $p> 2$ un primo y $a$ un entero cualquiera, se tiene que
      \[
         \symLegendre a p =
         \begin{cases}
            0, &\text{si}\ p \mid q.\\
            1, &\text{si}\ p \nmid q \ \text{y}\ a \ \text{es resto cuadrático en módulo}\ p.\\
            -1,&\text{si}\ a \ \text{no es resto cuadrático en módulo}\ p.\\
         \end{cases}
      \]
   \end{definition.box}

   \begin{lemma}[Propiedades del símbolo de Legendre]
      El símbolo de Legendre cumple los siguientes resultados.
      \begin{enumerate}
         \item [i)] Si $a \modulo{b}{p}$, entonces $\symLegendre a p = \symLegendre b p$.
         \item [ii)] Si $p \nmid a$, entonces $\symLegendre{a^2}{p} = 1$.
         \item [iii)] Para todos lo enteros $a,b$ se cumple $\symLegendre{ab}{p} = \symLegendre{a}{p} \symLegendre{b}{p}$.
         \item [iv)] $\symLegendre{-1}{p} = (-1)^{\frac{p - 1}{2}}$, en concreto $\symLegendre{-1}{p} = 1$ si y solo si $p \modulo{1}{4}$.
      \end{enumerate}
   \end{lemma}

   \begin{theorem.box}{Ley de reciprocidad cuadrática}{}
      Sean $p, q$ dos primos impares, se cumple que
      \[
         \symLegendre{p}{q} \symLegendre{q}{p} = (-1)^{\frac{p - 1}{2} \cdot \frac{q - 1}{2}}.
      \]
   \end{theorem.box}

   \begin{remark.box}{}{}
      De la ley de reciprocidad cuadrática, obtenemos que
      \[
         \symLegendre p q =
         \begin{cases}
            \symLegendre q p, &\text{si}\ p\ \text{ó}\ q \modulo{1}{4}\\
            - \symLegendre q p, &\text{si}\ p \equiv q \modulo{3}{4}
         \end{cases}
      \]
   \end{remark.box}

   \begin{theorem.box}{Criterio de Euler}{}
      Sea $p > 2$ un primo y $a$ un entero cualquiera, entonces se cumple
      \[
         \symLegendre a p \modulo{a^{\frac{p - 1}{2}}}{p}.
      \]
   \end{theorem.box}

   \begin{theorem.box}{}{}
      Sea $p > 2$ un primo, entonces se cumple que
      \[
         \symLegendre{2}{p} = (-1)^{\frac{p^2 - 1}{8}}.
      \]
   \end{theorem.box}

   \begin{remark.box}{}{}
      Del teorema anterior, obtenemos que
      \[
         \symLegendre{2}{p} =
         \begin{cases}
            1, &\text{si}\ p \modulo{1,7}{8}\\
            -1, &\text{si}\ p \modulo{3,5}{8}
         \end{cases}
      \]
   \end{remark.box}

   Veamos algunos ejercicios.
   \begin{exercise}
      Determinar los valores de los siguientes símbolos de Legendre.
      \begin{multicols}{3}
         \begin{enumerate}
            \item $\symLegendre{44}{103}$
            \item $\symLegendre{-60}{1019}$
            \item $\symLegendre{2010}{1019}$
            \item $\symLegendre{139}{433}$
            \item $\symLegendre{523}{1103}$
         \end{enumerate}
      \end{multicols}
   \end{exercise}

   \begin{exercise}
      Resolver las siguientes congruencias.
      \begin{multicols}{2}
         \begin{enumerate}
            \item $x^2 \modulo{196}{1357}$
            \item $x^2 + x \modulo{0}{13}$
            \item $x^2 + 3x + 2 \modulo{0}{7}$
            \item $x^2 + 5x + 13 \modulo{0}{11}$
            \item $x^2 - 3x + 2 \modulo{8}{17}$
            \item $25x^2 + 7x \modulo{7}{17}$
            \item $x^2 + x + 7 \modulo{0}{189}$
         \end{enumerate}
      \end{multicols}
   \end{exercise}

\end{document}