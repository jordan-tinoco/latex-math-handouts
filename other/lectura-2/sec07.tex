\section{El castellano es un bello idioma}

Un texto matemático sigue siendo un texto escrito en una lengua, en este caso el castellano, que cuenta con sus normas ortográficas, gramaticales, léxicas y discursivas.
Es menester respetarlas, no desde la coerción, sino desde la naturalidad e incluso el disfrute.
Ante este tema, lamentablemente, de nuevo nos encontramos con una situación incómoda y triste.
Para su vergüenza, muchos profesores de ciencias piensan que ``escribir bien es cosa de la gente de letras''.
No es cierto.
Escribir bien es cosa de gente con cultura y claridad de pensamiento; en particular, es un asunto de relevancia para cualquier profesor de universidad que se precie de serlo.
Para su vergüenza, un nutrido grupo de alumnos piensa igual, unos animados por lo que oyen a esos profesores, otros conscientes de que escribir no impide aprobar exámenes, sacar buenas notas en trabajos escritos, y aún menos sacar un título universitario.
Anteriormente hemos ofrecido argumentos contundentes (sección 1) para rebatir la perniciosa idea de que en las clases de ciencias no hay que saber escribir bien.
No insistiremos más en ellos. En esta sección vamos entrar en los detalles de la corrección lingüística, asunto espinoso con el que pocos profesores de ciencias se atreven a tratar.
Nuestros textos tienen que estar escritos en un castellano correcto y, si ello es posible, con voluntad de estilo.
Hemos dividido los errores en las siguientes categorías: ortográficos, gramaticales y léxicos.
Obviamente, no vamos a cubrir extensamente cada una de las categorías, sino que haremos énfasis en los errores más frecuentes que se encuentran en la escritura de textos matemáticos.


\subsection{Errores ortográficos}

Quizás uno de los errores más frecuentes es la acentuación incorrecta (¿o deberíamos decir la falta de acentuación?).
La tilde, ese símbolo ortográfico que usamos para identificar la sílaba tónica, tiene también la función de distinguir palabras.
Así, no es lo mismo escribir perdida y referirse a una ocasión que a la muerte de un ser querido, en cuyo caso sería pérdida.
Una acentuación incorrecta distrae al lector, pues le fuerza a pararse y averiguar a cuál de los posibles significados se está refiriendo el autor.
Ello causa una irritación soterrada y predispone al lector contra el texto.
¿Cómo puede confiar un lector exigente, tal como un profesor de universidad, en que el texto es tiene calidad si la acentuación es confusa y errática?

Ante nuestras quejas por la acentuación incorrecta, algunos alumnos nos han contestado, a veces de modo desafiante, que lo habían pasado por el corrector de Word.
Si hay dos palabras que se escriben exactamente igual salvo por el acento el corrector de Word no detectará la diferencia, pues no es capaz, de momento, de discriminar en función de la semántica (del significado).
Recomendamos al lector que repase las reglas de acentuación y use su capacidad mental en lugar de confiar ciegamente en el Word; además, este no les valdría en un examen, por ejemplo.
Para un repaso de las reglas de acentuación, véase el artículo sobre la tilde del Diccionario panhispánico de dudas [RAE13a] (DPD a partir de ahora).
Y, sí, el examen es de inglés.

Otro punto donde se observan numerosos casos de incorrección es en el uso de las mayúsculas.
Quizás sea por la influencia de inglés, pero nos encontramos con usos que corresponden a sus reglas en lugar de a las del castellano.
Los días de la semana, los meses del año y las estaciones se escriben en minúscula: lunes, junio y otoño, y no $\bigotimes$Lunes, $\bigotimes$Junio y $\bigotimes$Otoño (usamos el símbolo $\bigotimes$ para indicar ejemplos de incorrecciones).
Otros errores relacionados con el uso de las mayúsculas en los textos matemáticos son los siguientes:
\begin{enumerate}
    \item Los nombres de los teoremas van en minúsculas: teorema fundamental del cálculo y no $\bigotimes$Teorema Fundamental del Cálculo.
    \item Los nombres de las disciplinas científicas en el contexto académico (asignaturas, grados, materias de estudio) van en mayúsculas: Soy licenciado en Informática.
    En caso contrario, van en minúscula: Ninguna persona puede abarcar las matemáticas actuales.
    \item Los títulos de los trabajos no llevan la mayúscula inicial de cada palabra.
    Esa es una norma inglesa.
    Es incorrecto escribir en el título de un trabajo $\bigotimes$El Método de la Bisección.
\end{enumerate}


Para más información, véase el artículo sobre el uso de las mayúsculas en el DPD [RAE13c].
Y por último, entramos en uno de los grandes problemas de nuestros alumnos y que más confusión causan: la puntuación. Un texto mal puntuado rompe constantemente las expectativas de lectura y se convierte pronto en un galimatías. La puntuación está pensada para cimentar la estructura del texto, para matizar sus distintas partes, para guiar en la lectura fluida del texto. Cuando se usa contra natura consigue los efectos más perversos; el más inmediato, la sensación de chapuza. Un texto mal puntuado habla elocuentemente de la confusión de ideas y de la falta de capacidad para revisar un escrito de su autor.


Vamos a repasar brevemente los principales usos de la puntuación; de nuevo, se remite al lector a los excelentes artículos del DPD [RAE13b, RAE13d]

\textbf{Uso de la coma.}\par
Quizás el signo de puntuación más potente y al mismo tiempo más difícil de usar.
La coma sirve principalmente para separar partes de la oración. El uso de la coma es obligatorio en ciertos casos (el llamado uso normativo) y en otros es la elección del autor (uso estilístico). La coma se usa en los siguientes casos:


En las aposiciones explicativas (interrupciones hechas en la oración para explicar algo): La función f(x), que es continua y derivable, tiene un único máximo en [0,1].
Coma de enumeración: Se pone cuando se hace una enumeración de objetos. Si los dos últimos elementos van unidos por la conjunción y u o, no hay coma entre ellos: La función tiene máximos relativos en x = 1, x = 3
2 y x = 5.
Coma elíptica. Cuando se repite el verbo en dos oraciones seguidas, la segunda vez se puede sustituir por una coma: La función f(x) es continua; su derivada, no. En este caso es recomendable usar el punto y coma tal cual aparece en el ejemplo anterior.

Coma delante de las conjunciones. Se pone coma delante de las conjunciones (o locuciones conjuntivas) que unen dos oraciones compuestas. Ejemplos de esas conjunciones son: pero, aunque, sino que (adversativas); así que, de manera que, conque, luego (consecutivas o ilativas); esto es, o sea, es decir (explicativas). Por ejemplo: La sucesión an es creciente, pero ello no implica que tienda a infinito. En el caso de las oraciones causales la coma se coloca según su tipo. Las oraciones causales se clasifican en causales lógicas, en que una oración se deduce de la otra, y las causales enunciativas, en que se enuncia algo como consecuencia lógica de otro hecho. Por ejemplo, n2 es par, porque n es par es causal lógica y n es par porque n2 es par es causal enunciativa. En las causales lógicas se separan las oraciones por una coma y en las causales enunciativas, no.

Coma sintáctica. En ocasiones, por propósito de énfasis o estilo, se invierte el orden natural de las partes de la oración. En este caso, ello se advierte poniendo una coma detrás de la parte anticipada, sobre todo si esa parte es muy larga. Se separa con coma en En el conjunto de los números enteros y bajo ciertas condiciones, es posible extender la función f(x), pero no en Elegante es esta demostración. Este es uno de los casos donde la coma tiene un uso opcional, pero que siempre debería estar sancionado por el sentido común. Es obligatorio poner la coma en el caso de las oraciones adverbiales, cuando la subordinada está antepuesta: Antes de que pasemos a estudiar la continuidad, vamos a calcular el dominio de la función.

Coma para evitar la ambigüedad. Este uso no es normativo, pero una lectura atenta del texto indica enseguida dónde ponerlo. No es lo mismo Me vestí, como me indicaron (me vestí porque me invitaron a ello) que Me vestí como me indicaron (me vestí en la forma en que me indicaron). Y, no, no somos caníbales.

Usos fijos. Se pone coma después de ciertos enlaces, tales como en primer lugar, por lo tanto, por otro lado, por último, con todo, sin embargo, no obstante, y también después de locuciones adverbiales cuando afectan a la frase entera: Por lo tanto, las funciones derivables son continuas.

Separación de sujeto y verbo. Salvo que medie una aposición explicativa, no se usa coma para separar el sujeto y el verbo: ○xToda sucesión creciente y acotada, tiene límite.


Uso del punto y coma. Por alguna razón misteriosa los estudiantes usan poco este signo de puntuación. Es un error, ya que es tremendamente versátil a la hora de organizar un texto y separar con claridad sus partes.
Enumeraciones complejas.Se usa en este caso cuando la enumeración es larga o los elementos de la enumeración llevan comas también: La integral consta de: los límites de integración, escritos abajo y arriba del símbolo ∫ ; el integrando, escrito a continuación; y el diferencial dx, que indica respecto a qué variable se integra.
Separación de oraciones. En castellano la manera de separar oraciones sintácticamente independientes pero con relación semántica es con punto y coma, y no con coma. Por ejemplo: Vemos que, para K arbitrario, existe un n0 tal que an > K si n > n0; no se puede concluir que el límite de an exista.
Para separar conectores. Siguiendo con el punto de más arriba sobre usos fijos de la coma, con conectores de sentido adversativo, consecutivo o concesivo se pone una coma tras ellos, sobre todo cuando se unen dos oraciones en una más larga: Si la serie converge, entonces su término general tiende a cero; no obstante, el recíproco es falso.
Enumeraciones en varias líneas. En una lista que se escribe en líneas independientes se pone punto y coma en cada línea, salvo en la última que se pone punto:
Sea G un grafo. Las siguientes condiciones son equivalentes:

el grafo G es conexo;

existe un camino entre dos vértices arbitrarios del grafo;

no existen aristas puente en el grafo.


\subsection{Errores gramaticales}
Los errores más comunes que nos hemos encontrado en los trabajos escritos de los alumnos son los siguientes:

Errores de sintaxis. El orden de las partes de una oración es esencial para entender su significado. La elección de ese orden ha de estar guiado por un afán de máxima claridad. Por ejemplo, la frase ○xVamos a calcular la derivada de la función segunda es incorrecta; el lector no sabe si se refiere a la segunda de dos funciones o a la derivada segunda de una única función. Debería haberse escrito Vamos a calcular la derivada segunda de la función.
Errores de concordancia. En general, estos errores aparecen por una falta de revisión del texto. El estudiante cambia una parte de la frase y no tiene el rigor metodológico de revisar cómo afecta ese cambio al resto de la oración. Un caso que sí es frecuente es el del uso impersonal del verbo haber. Este verbo, cuando se usa como impersonal, no tiene sujeto y se conjuga en la tercera persona del singular: hay, ha habido, hubo, etc. Así, es incorrecto decir ○xHan habido dos soluciones de la ecuación, y lo correcto es Ha habido dos soluciones de la ecuación.
Errores en los verbos. Cuando el alumno no tiene clara una demostración o la solución de un problema, a veces tiende a disfrazarlo mediante el uso incorrecto de las perífrasis verbales. Los principales errores se pueden clasificar en tres grupos:

Uso innecesario: x○Hemos sido capaces de demostrar el resultado, que debería decirse simple y llanamente Hemos demostrado el resultado.
Error de construcción, en particular con los complementos de régimen verbal. Estos complementos son los sintagmas preposicionales que van unidos a un verbo y sin el cual este no tiene significado completo; por ejemplo, contar con alguien. Es incorrecto decir x○El resultado depende el teorema 1, sino El resultado depende del teorema 1.
Errores de uso. Uno de los casos más frecuentes es la confusión entre deber y deber de. El primero tiene valor de obligación y el segundo de probabilidad. No se puede decir x○El resultado debe de ser cierto (esto es, el resultado es probablemente cierto), sino El resultado debe ser cierto (esto es, el resultado es cierto con seguridad).


\subsection{Errores léxicos}
Los errores léxicos que hemos observado en nuestros alumnos principalmente se reducen a una sola clase: pobreza léxica. Un texto brilla si se usan diferentes palabras, aunque sea para expresar la misma idea. Cada palabra tiene su matiz, su personalidad, su sonoridad, y una elección cuidada de las palabras dice mucho del texto y, por ende, del autor. La pobreza léxica se caracteriza por el uso de un conjunto pequeño de vocablos para expresar ideas para las cuales existen palabras con más riqueza semántica. Típicamente, en un trabajo escrito un alumno usaría el verbo decir repetidamente, que es un verbo de carácter general; sin embargo, hay otros muchos verbos que expresan matices distintos, tales como comentar, explicar, referir, señalar, glosar, enumerar, argumentar, objetar, replicar, comunicar, entre otros.