\section{Principios generales de escritura}

En esta sección vamos a enunciar unos principios generales para escribir matemáticas, aunque, en realidad, se aplican a muchos otros tipos de escritura. El lector se dará cuenta de que estos principios se basan lisa y llanamente en el sentido común y que todos persiguen el objetivo de hacer de un texto matemático un acto sereno y profundo de comunicación.

\hspace{5mm}\textbf{Claridad.}
Nada se puede comunicar si previamente el autor no tiene una idea clara de ello.
Tan clara tiene que estar la idea en la mente del autor que este tiene que estar ansioso por comunicarla.
Pero la idea tiene que transmitirse de modo inteligible, presentarse en un orden racional, fruto de la reflexión, podando los detalles que resulten innecesarios al lector ideal de nuestro texto.
Por ejemplo, si la demostración va a ser larga, es útil describir al lector el plan general de un modo informal, y luego pasar a los detalles técnicos de modo formal.

\hspace{5mm}\textbf{Concisión.}
La concisión es difícil de alcanzar y lleva mucha práctica.
La concisión es el arte de significar lo más posible con el menor número de palabras.
Un texto conciso está bien cimentado y en él no sobra ni falta nada.
La concisión se alcanza a través de un proceso de depuración del texto que incluye, entre otros, los siguientes procesos:
eliminar palabras con poco significado (redundancias, palabras que no maticen, muletillas);
descartar el material que el lector pueda deducir por sí mismo;
cuando sea posible, sustituir subordinadas adjetivas por adjetivos;
usar palabras precisas y cortas en lugar de perífrasis;
eliminar temas secundarios que afecten al tema principal (más tarde se pueden comentar, pero una vez dicho lo esencial).

\hspace{5mm}\textbf{Simplicidad.}
Relacionadas con los dos puntos anteriores está la simplicidad.
No debe confundirse simplicidad con pobreza de escritura.
La primera se refiere a una organización de la escritura que es inmediatamente inteligible y transparente;
la segunda, precisamente, a una falta de ello por defecto.
La simplicidad es una cuestión de estilo matemático también.
Un teorema se puede probar por más de un camino y a veces el camino más corto es el preferible.

\hspace{5mm}\textbf{Escritura en frases y párrafos.}
Este punto puede parecer totalmente superfluo, por obvio, pero la experiencia demuestra que no se insiste lo suficiente. En un texto matemático, como en cualquier otro, las ideas deben organizarse en frases, con su sentido completo, y estas, a su vez, en párrafos. Cada párrafo debe contener una idea principal, la cual se explica en frases claras y concisas. Mezclar ideas distintas en un mismo párrafo es una mala práctica y habla penosamente de la claridad mental del autor sobre el tema en cuestión. Una cuestión importante es la puntuación, que cuando se utiliza de modo cabal, ayuda enormemente a estructurar el texto y dotarlo de claridad en sus partes.

\hspace{5mm}\textbf{Precisión.}
En matemáticas el lenguaje es extremadamente preciso. Usarlo sin esa precisión conduce a sumir al lector en una profunda confusión. Por ejemplo, hay diferencias entre expresión, igualdad, ecuación y fórmula, y cuando se escribe hay que tenerlas en cuenta. Una expresión es una sucesión de símbolos que expresa una relación matemática; por ejemplo, x2 + 2x + 1 es una expresión. Una igualdad establece que dos expresiones son iguales y siempre lleva el signo “=”; por ejemplo, (3x- 1)2 = 8x2 - 4x es una igualdad. Si la igualdad tiene carácter cuantitativo, se habla de ecuación y está se puede resolver; en el ejemplo anterior, la única solución es x = 1. Una fórmula es una igualdad que tiene carácter general; se habla de la fórmula del área del círculo A = πr2, pero no de la igualdad entre A y πr2.

\hspace{5mm}\textbf{Uso juicioso de los símbolos matemáticos.}
Un texto matemático sigue siendo un texto y no hay que sustituir las palabras por símbolos, pues lo hace intricado de leer. Debería resistirse el uso de los símbolos tanto como sea posible. Expliquemos las matemáticas con palabras del castellano mientras la necesidad de formalización no nos obligue a usar símbolos. No hay cosa más farragosa de leer que un texto matemático en que se han sustituido un gran número de palabras por símbolos matemáticos. Es preferible escribir para todo número real que ∀x ∈ ℝ. Todo símbolo matemático ha de tener necesariamente una función gramatical dentro de la frase.

Algunos símbolos pueden actuar como verbos; por ejemplo, x = y es equivalente a x es igual a y. Pero la expresión x2 + 2x + 1 no tiene verbo y ha de tratarse como un nombre; por ejemplo, x2 + 2x + 1 es siempre positiva para todo x. Cuando los cálculos son largos conviene ponerlos en una línea aparte y centrados:

Si es necesario, se pueden dividir un cálculo largo en varios cálculos más pequeños y explicar cada uno con palabras.

\hspace{5mm}\textbf{Organización del texto.}
Como cualquier otro texto de envergadura, un texto matemático necesita planificación. Antes de escribirlo, tenemos que tomar notas, visualizar mínimamente sus partes y cuál va a ser su orden, imaginar el nivel de conocimiento de nuestro lector, en qué conceptos vamos a hacer más énfasis, cuál va a ser el nivel de repetición, entre otros. Ponerse a escribir sin haber planificado el texto es lo que se llama escribir por acumulación. Vamos sumando párrafo tras párrafo, trabajosamente, con penuria, pero nada tiene coherencia ni estructura y faltan detalles esenciales, pues el autor no previó nada. El lector se percata de esto tras la lectura de las primeras frases y ya alberga sospechas inquietantes de las intenciones del autor y el texto.

\hspace{5mm}\textbf{Aspectos formales del texto.}
Hay que cuidar el contenido matemático y la corrección lingüística. Si falla cualquiera de estos dos elementos, el texto se vuelve insufrible. Sobre la corrección lingüística, se profundizará en la sección 7. Respecto al contenido matemático, aquí ofrecemos unos mínimos consejos de sentido común: (1) como hemos dicho, hay que cuidar el uso de los símbolos, su encaje en las frases, y, a ser posible, emplear el número exacto de símbolos; (2) asimismo, hay que numerar claramente los resultados para que sea fácil referirse a ellos más tarde; (3) hay que ser consistente con la notación y usar la que está aceptada por convención; (4) usar figuras para ilustrar demostraciones; (5) dar formato al documento de manera que sea cómodo de leer.

\hspace{5mm}\textbf{Revisión continua del texto.}
Una clave para escribir un buen texto es la revisión, y con esto queremos decir la revisión continua del texto. Como consecuencia de esas revisiones, volveremos a escribir y a escribir párrafos enteros. No pasa nada. Si el resultado final es un texto bien escrito, todas esas revisiones merecen la pena. Algunas revisiones tienen que tener un criterio específico. Por ejemplo, una serie de revisiones deberían ser gramaticales; otra, del contenido matemático; otra, de la estructura en párrafos; otra, del ritmo y la fluidez del texto; otra, del léxico y las repeticiones sintácticas; otra del orden de las partes. Además, mi consejo es que, tras trabajar en el texto intensamente durante unos días, nos olvidemos de él, y tras un cierto tiempo volvamos a revisarlo de nuevo con renovada exhaustividad. Tras ese paréntesis se borra la huella mental, que en ocasiones nos hace pasar por alto errores y aceptar como bueno lo escrito (cuando, probablemente, no lo es). La lectura de nuestro documento debe hacerse con mucha atención y despacio; no es una lectura rápida en absoluto. Algunos autores recomiendan leer en voz alto para forzar la lectura lenta y comprensiva (véase Kevin Houston [Hou13]). El matemático Paul Halmos en su famoso ensayo How to write mathematics [Hal70] recomienda escribir en espiral. Si un texto va a tener, digamos, 5 párrafos, una posible sucesión de escritura podría ser 1 - 2 - 1 - 2 - 3 - 1 - 2 - 3 - 4 - 1 - 2 - 3 - 4 - 5. En realidad, el método que predica Halmos combina escritura y revisión en el mismo proceso.