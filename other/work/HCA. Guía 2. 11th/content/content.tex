\section{Exercise}

This guide to functions and logarithms will be received on Monday, April 15 (two members).
For more information, see chapter 3 of the book, pages 136 to 148.

\begin{exer}
    Graph each logarithmic function.
    Then state its domain and rage.

    \begin{tasks}(3)
        \task $y = \log_4 (x)$
        \task $y = \log_5 (x)$
        \task $y = \log_8 (x)$
    \end{tasks}
\end{exer}

\begin{exer}
    Describe each transformation of $f(x) = \log_2 (x)$.
    Then use a graph of $f(x)$ to sketch the graph of $g(x)$.
    \begin{tasks}(3)
        \task $g(x) = \log_2 (x) - 3$
        \task $g(x) = \log_2 (x + 2) + 4$
        \task $g(x) = - \dfrac{1}{2} \log_2 (x)$
    \end{tasks}
\end{exer}

\begin{exer}
    Write a function rule for $g(x)$, the described transformation of $f(x)$.
    Then confirm your answer by graphing $f(x)$ and $g(x)$ in the same window.
    \begin{enumerate}
        \item $f(x) = \log (x)$ translated 2 units left and 3 units down.
        \item $f(x) = \ln (x)$ translated 5 units left and then reflected in the $y$-axis.
        \item $f(x) = \log (x)$ reflected in the $x$-axis and translated 2 units up.
    \end{enumerate}
\end{exer}

\begin{exer}
    \textbf{Explain:} Why is 0 not in the domain of $f(x) = \log (x)$?
\end{exer}

\begin{exer}
    Specify the intervals of $x$ for which $f(x) = \ln (x)$ is:
    \begin{tasks}(4)
        \task positive
        \task negative
        \task zero
        \task undefined
    \end{tasks}
    Are these intervals the same for $f(x) = \log (x)$?
\end{exer}

\begin{exer}
    Example 7 illustrates how the Richter scale is used to compare earthquakes.
    Earthquakes with a magnitude less than $3.0$ are often not felt, while the most severe earthquakes my measure $7.0$ or higher.

    Use $R = \log \left(\dfrac{A}{A_0}\right)$ to find the magnitude of the 2017 earthquake near Brenas, Puerto Rico, if its amplitude was 1200 times as great as $A_0$.
\end{exer}

\begin{exer}
    Expand each logarithmic expression.
    Assume all variables are positive values.
    \begin{tasks}(3)
        \task $\log (xy^3)$
        \task $\log (ab^2 c)$
        \task $\log (7x^4)$
        \task $\log \left(\dfrac{3}{x^2}\right)$
        \task $\log \left(100 \sqrt[3]{x^2}\right)$
        \task $\log_3 \left(\sqrt {27xy^2}\right)$
        \task $\log_7 \left(\dfrac{49\sqrt {y}}{x}\right)$
        \task $\log_2 \left(\dfrac{\sqrt {2x}}{16y}\right)$
    \end{tasks}
\end{exer}

\begin{exer}
    Write each expression as a single logarithm.
    \begin{tasks}(1)
        \task $x\log (3) - 2\log (y) + \dfrac{1}{3}\log (z)$
        \task $x\ln (3) + y\ln (2) - z\ln (3)$
        \task $\ln (2x) + 3\ln (4x^2) - \ln (8x^4)$
        \task $3\log \left(\dfrac{3}{2}\right) + 2\log (6x) - 2\log (3)$
    \end{tasks}
\end{exer}