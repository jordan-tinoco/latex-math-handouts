\section{Ejercicios}

\begin{exercise}
    Anotar los resultados obtenidos al lanzar un dado en la siguiente tabla
    \begin{table}[H]
        \centering
        \begin{tabular}{|c|c|c|c|c|c|c|c|c|c|}
            \hline
            4 & 2 & 6 & 6 & 1 & 3 & 5 & 4 & 2 & 5 \\\hline
            4 & 5 & 1 & 2 & 6 & 4 & 3 & 3 & 6 & 6 \\\hline
            1 & 2 & 2 & 5 & 4 & 1 & 5 & 3 & 1 & 4 \\\hline
        \end{tabular}
    \end{table}
    Completar la tabla de frecuencia, \underline{\textit{truncar o redondear a la milésima}}.
    \begin{table}[H]
        \centering
        \begin{tabular}{|p{1.8cm}|p{1.8cm}|p{2cm}|p{1.8cm}|p{2cm}|p{2cm}|p{2cm}|}
            \hline
            Resultado (Variable) & Frecuencia absoluta & Frecuencia absoluta acumulada & Frecuencia relativa & Frecuencia relativa acumulada & Frecuencia porcentual & Frecuencia porcentual acumulada\\
            \hline\hline
            1 & & & & & & \\\hline
            2 & & & & & & \\\hline
            3 & & & & & & \\\hline
            4 & & & & & & \\\hline
            5 & & & & & & \\\hline
            6 & & & & & & \\\hline
        \end{tabular}
    \end{table}
\end{exercise}
\begin{solution}
    Veamos los datos obtenidos.
    \begin{table}[H]
        \centering
        \begin{tabular}{|p{1.8cm}|p{1.8cm}|p{2cm}|p{1.8cm}|p{2cm}|p{2cm}|p{2cm}|}
            \hline
            Resultado (Variable) & Frecuencia absoluta & Frecuencia absoluta acumulada & Frecuencia relativa & Frecuencia relativa acumulada & Frecuencia porcentual & Frecuencia porcentual acumulada\\
            \hline\hline
            1 & 5 &  5 & 0.166 & 0.166 & 16.6 \% & 16.6 \% \\\hline
            2 & 5 & 10 & 0.166 & 0.332 & 16.6 \% & 33.2 \% \\\hline
            3 & 4 & 14 & 0.133 & 0.465 & 13.3 \% & 46.5 \% \\\hline
            4 & 6 & 20 & 0.200 & 0.665 & 20.0 \% & 66.5 \% \\\hline
            5 & 5 & 25 & 0.166 & 0.831 & 16.6 \% & 83.1 \% \\\hline
            6 & 5 & 30 & 0.166 & 0.997 & 16.6 \% & 99.7 \% \\\hline
        \end{tabular}
    \end{table}
\end{solution}

\begin{exercise}
    Se realiza una encuesta a jóvenes sobre el día de preferencia para ir al cine, sus respuestas fueron:
    martes, jueves, viernes, sábado, sábado, domingo, viernes, jueves, lunes, miércoles, viernes, viernes, sábado, viernes, martes.
    Con los resultados obtenidos anteriormente complete la tabla de frecuencia.
    \begin{table}[H]
        \centering
        \begin{tabular}{|p{1.8cm}|p{1.8cm}|p{1.8cm}|p{1.8cm}|p{2cm}|p{2cm}|p{2cm}|}
            \hline
            Resultado (Variable) & Frecuencia absoluta & Frecuencia absoluta relativa & Frecuencia relativa & Frecuencia relativa acumulada & Frecuencia porcentual & Frecuencia porcentual acumulada\\
            \hline\hline
            Lunes & & & & & & \\\hline
            Martes & & & & & & \\\hline
            Miércoles & & & & & & \\\hline
            Jueves & & & & & & \\\hline
            Viernes & & & & & & \\\hline
            Sábado & & & & & & \\\hline
            Domingo & & & & & & \\\hline
        \end{tabular}
    \end{table}
\end{exercise}
\begin{solution}
    Veamos los datos obtenidos.
    \begin{table}[H]
        \centering
        \begin{tabular}{|p{1.8cm}|p{1.8cm}|p{2cm}|p{1.8cm}|p{2cm}|p{2cm}|p{2cm}|}
            \hline
            Resultado (Variable) & Frecuencia absoluta & Frecuencia absoluta acumulada & Frecuencia relativa & Frecuencia relativa acumulada & Frecuencia porcentual & Frecuencia porcentual acumulada\\
            \hline\hline
            Lunes     & 1 &  1 & 0.066 & 0.066 &  6.6 \% &  6.6 \% \\\hline
            Martes    & 2 &  3 & 0.133 & 0.199 & 13.3 \% & 19.9 \% \\\hline
            Miércoles & 1 &  4 & 0.066 & 0.265 &  6.6 \% & 26.5 \% \\\hline
            Jueves    & 2 &  6 & 0.133 & 0.398 & 13.3 \% & 39.8 \% \\\hline
            Viernes   & 5 & 11 & 0.333 & 0.731 & 33.3 \% & 73.1 \% \\\hline
            Sábado    & 3 & 14 & 0.200 & 0.931 & 20.0 \% & 93.1 \% \\\hline
            Domingo   & 1 & 15 & 0.066 & 0.997 &  6.6 \% & 99.7 \% \\\hline
        \end{tabular}
    \end{table}
\end{solution}

\begin{exercise}
    Los siguientes datos corresponden a las edades (en años) de los alumnos que integran el coro de una I.E.
    \begin{table}[H]
        \centering
        \begin{tabular}{c c c c c c c c c c}
            16 & 13 & 14 & 14 & 15 & 15 & 16 & 14 & 15 & 16 \\
            15 & 14 & 15 & 13 & 14 & 13 & 15 & 13 & 14 & 14 \\
        \end{tabular}
    \end{table}
    \begin{enumerate}
        \item Construya una tabla de distribución que refleje el comportamiento de las edades de los alumnos.
        \item Determine el promedio que represente las edades de los alumnos.
        \item Calcular el 50\% de las edades de los alumnos.
        \item Calcular la edad más frecuente de los alumnos que integran el coro.
    \end{enumerate}
\end{exercise}
\begin{solution}
    \begin{enumerate}
        \item Tenemos la siguiente tabla de frecuencia.
        \begin{table}[H]
            \centering
            \begin{tabular}{|p{1.7cm}|p{1.8cm}|p{2cm}|p{1.8cm}|p{2cm}|p{2cm}|p{2cm}|}
                \hline
                Edades & Frecuencia absoluta & Frecuencia absoluta acumulada & Frecuencia relativa & Frecuencia relativa acumulada & Frecuencia porcentual & Frecuencia porcentual acumulada\\
                \hline\hline
                13 & 4 &  4 & 0.200 & 0.200 & 20.0 \% & 20.0 \% \\\hline
                14 & 7 & 11 & 0.350 & 0.550 & 35.0 \% & 55.0 \% \\\hline
                15 & 6 & 17 & 0.300 & 0.850 & 30.0 \% & 85.0 \% \\\hline
                16 & 3 & 20 & 0.150 & 1.000 & 15.0 \% & 100 \% \\\hline
            \end{tabular}
        \end{table}
        \item El promedio es igual a la media de las edades, esta está dada por $\overline{x} = \sum_{i = 1}^{n} \frac{x_i \cdot f_i}{n}$, es decir
        \[
            \overline{x} = \frac{13\cdot 4 + 14\cdot 7 + 15\cdot 6 + 16\cdot3}{20} = \frac{288}{20} = 14.4
        \]
        \item Sabemos que la mediana está en la posición dada por $\frac{n}{2} = \frac{20}{2} = 10$ donde $n = 2$ es la cantidad de edades, luego la mediana es 13.
        Es decir que el 50\% de las edades son menores a 13 años.
        \item Y claramente la moda sería la edad con más frecuencia, vemos la mayor frecuencia es 7, por tanto, la moda es 14. \qedhere
    \end{enumerate}
\end{solution}

\begin{exercise}
    Supongamos que tienes los siguientes datos que representan las ventas diarias de una tienda de ropa durante una semana:
    10, 8, 12, 15, 14, 11, 9, 12, 16, 19, 20, 9, 10, 9, 15, 25.

    Realiza lo siguiente:
    \begin{enumerate}
        \item Crea una tabla de distribución de frecuencia para estos datos no agrupados.
        \item Calcula la media de las ventas.
        \item Encuentra la mediana de las ventas.
        \item Determina la moda de las ventas.
    \end{enumerate}
\end{exercise}
\begin{solution}
    \begin{enumerate}
        \item Tenemos la siguiente tabla de frecuencia.
        \begin{table}[H]
            \centering
            \begin{tabular}{|p{1.7cm}|p{1.8cm}|p{2cm}|p{1.8cm}|p{2cm}|p{2cm}|p{2cm}|}
                \hline
                Ventas & Frecuencia absoluta & Frecuencia absoluta acumulada & Frecuencia relativa & Frecuencia relativa acumulada & Frecuencia porcentual & Frecuencia porcentual acumulada\\
                \hline\hline
                 8 & 1 &  1 & 0.062 & 0.062 &  6.2 \% &  6.2 \% \\\hline
                 9 & 3 &  4 & 0.187 & 0.249 & 18.7 \% & 24.9 \% \\\hline
                10 & 2 &  6 & 0.125 & 0.374 & 12.5 \% & 37.4 \% \\\hline
                11 & 1 &  7 & 0.062 & 0.436 &  6.2 \% & 43.6 \% \\\hline
                12 & 2 &  9 & 0.125 & 0.561 & 12.5 \% & 56.1 \% \\\hline
                14 & 1 & 10 & 0.062 & 0.623 &  6.2 \% & 62.3 \% \\\hline
                15 & 2 & 12 & 0.125 & 0.748 & 12.5 \% & 74.8 \% \\\hline
                16 & 1 & 13 & 0.062 & 0.810 &  6.2 \% & 81.0 \% \\\hline
                19 & 1 & 14 & 0.062 & 0.872 &  6.2 \% & 87.2 \% \\\hline
                20 & 1 & 15 & 0.062 & 0.934 &  6.2 \% & 93.4 \% \\\hline
                25 & 1 & 16 & 0.062 & 0.996 &  6.2 \% & 99.6 \% \\\hline
            \end{tabular}
        \end{table}
        \item La media de las ventas, está dada por $\overline{x} = \sum_{i = 1}^{n} \frac{x_i \cdot f_i}{n}$, es decir
        \[
            \overline{x} = \frac{8 + 9\cdot 3 + 10 \cdot 2 + 11 + 12\cdot 2 + 14 + 15\cdot 2 + 16 + 19 + 20 + 25}{16} = \frac{214}{16} = 13.37
        \]
        \item Sabemos que la mediana está en la posición dada por $\frac{n}{2} = \frac{16}{2} = 8$ donde $n = 16$ es la cantidad de ventas, luego la mediana es 12.
        \item Y claramente la moda sería la venta con más frecuencia, vemos la mayor frecuencia es 3, por tanto, la moda es 9. \qedhere
    \end{enumerate}
\end{solution}

\begin{exercise}
    Supongamos que tienes los siguientes datos que representan los precios de productos en una tienda de electrónica:
    500, 800, 650, 900, 700, 550, 800, 1000, 800, 700, 400, 600, 800, 500, 800, 1200.

    Realiza lo siguiente:
    \begin{enumerate}
        \item Crea una tabla de distribución de frecuencia para estos datos no agrupados.
        \item Calcula la media de los precios.
        \item Encuentra la mediana de los precios.
        \item Determina la moda de los precios.
    \end{enumerate}
\end{exercise}
\begin{solution}
    \begin{enumerate}
        \item Tenemos la siguiente tabla de frecuencia.
        \begin{table}[H]
            \centering
            \begin{tabular}{|p{1.7cm}|p{1.8cm}|p{2cm}|p{1.8cm}|p{2cm}|p{2cm}|p{2cm}|}
                \hline
                Precios & Frecuencia absoluta & Frecuencia absoluta acumulada & Frecuencia relativa & Frecuencia relativa acumulada & Frecuencia porcentual & Frecuencia porcentual acumulada\\
                \hline\hline
                 400 & 1 &  1 & 0.062 & 0.062 &  6.2 \% &  6.2 \% \\\hline
                 500 & 2 &  3 & 0.125 & 0.187 & 12.5 \% & 18.7 \% \\\hline
                 550 & 1 &  4 & 0.062 & 0.249 &  6.2 \% & 24.9 \% \\\hline
                 600 & 1 &  5 & 0.062 & 0.311 &  6.2 \% & 31.1 \% \\\hline
                 650 & 1 &  6 & 0.062 & 0.373 &  6.2 \% & 37.3 \% \\\hline
                 700 & 2 &  8 & 0.125 & 0.498 & 12.5 \% & 49.8 \% \\\hline
                 800 & 5 & 13 & 0.312 & 0.810 & 31.2 \% & 81.0 \% \\\hline
                 900 & 1 & 14 & 0.062 & 0.872 &  6.2 \% & 87.2 \% \\\hline
                1000 & 1 & 15 & 0.062 & 0.934 &  6.2 \% & 93.4 \% \\\hline
                1200 & 1 & 16 & 0.062 & 0.996 &  6.2 \% & 99.6 \% \\\hline
            \end{tabular}
        \end{table}
        \item La media de las precios, está dada por $\overline{x} = \sum_{i = 1}^{n} \frac{x_i \cdot f_i}{n}$, es decir
        \[
            \overline{x} = \frac{400 + 500\cdot2 + 550 + 600 + 650 + 700\cdot2 + 800\cdot5 + 900 + 1000 + 1200}{16} = \frac{11700}{16} = 731.25
        \]
        \item Sabemos que la mediana está en la posición dada por $\frac{n}{2} = \frac{16}{2} = 8$ donde $n = 16$ es la cantidad de precios, luego la mediana es 700.
        \item Y claramente la moda sería el precio con más frecuencia, vemos la mayor frecuencia es 5, por tanto, la moda es 800. \qedhere
    \end{enumerate}
\end{solution}

\begin{exercise}
    Supongamos que tienes los siguientes datos que representan los tiempos de entrega de paquetes en una tienda en línea:
    2, 1, 3, 4, 2, 3, 2, 1, 3, 4, 5, 2, 1, 2, 2, 3, 5.

    Realiza lo siguiente:
    \begin{enumerate}
        \item Crea una tabla de distribución de frecuencia para estos datos.
        \item Calcula la media de los tiempos de entrega.
        \item Encuentra la mediana de los tiempos de entrega.
        \item Determina la moda de los tiempos de entrega.
    \end{enumerate}
\end{exercise}
\begin{solution}
    \begin{enumerate}
        \item Tenemos la siguiente tabla de frecuencia.
        \begin{table}[H]
            \centering
            \begin{tabular}{|p{1.7cm}|p{1.8cm}|p{2cm}|p{1.8cm}|p{2cm}|p{2cm}|p{2cm}|}
                \hline
                Tiempos & Frecuencia absoluta & Frecuencia absoluta acumulada & Frecuencia relativa & Frecuencia relativa acumulada & Frecuencia porcentual & Frecuencia porcentual acumulada\\
                \hline\hline
                1 & 3 &  3 & 0.176 & 0.176 & 17.6 \% & 17.6 \% \\\hline
                2 & 6 &  9 & 0.352 & 0.528 & 35.2 \% & 52.8 \% \\\hline
                3 & 4 & 13 & 0.235 & 0.763 & 23.5 \% & 76.3 \% \\\hline
                4 & 2 & 15 & 0.117 & 0.880 & 11.7 \% & 88.0 \% \\\hline
                5 & 2 & 17 & 0.117 & 0.997 & 11.7 \% & 99.7 \% \\\hline
            \end{tabular}
        \end{table}
        \item La media de los tiempos de entrega, está dada por $\overline{x} = \sum_{i = 1}^{n} \frac{x_i \cdot f_i}{n}$, es decir
        \[
            \overline{x} = \frac{1\cdot 3 + 2\cdot 6 + 3\cdot4 + 4\cdot2 + 5\cdot2}{17} = \frac{45}{17} = 2.64
        \]
        \item Sabemos que la mediana está en la posición dada por $\frac{n + 1}{2} = \frac{18}{2} = 9$ (por ser $n$ impar) donde $n = 17$ es la cantidad de tiempos de entregas, luego la mediana es 2.
        \item Y claramente la moda sería el tiempo con más frecuencia, vemos la mayor frecuencia es 6, por tanto, la moda es 2. \qedhere
    \end{enumerate}
\end{solution}