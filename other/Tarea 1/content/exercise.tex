\section{Ejercicios}

\begin{exercise}
    Los siguientes datos representan la altura (en centímetros) de un grupo de personas:
    
    165, 170, 155, 180, 173, 162, 158, 175, 168, 160, 177, 169, 171, 166, 171, 166, 164, 176, 172, 167, 163, 158, 155, 180, 155, 155, 170, 175, 155, 155.
    \begin{enumerate}
        \item Organiza estos datos en una distribución de frecuencia que presenta la altura en centímetro.
        \item Determine la altura en centímetro más frecuente del grupo de personas.
    \end{enumerate}
\end{exercise}
\begin{solution}
    \begin{enumerate}
        \item Para una cantidad $N = 29$ de alturas, tenemos la siguiente tabla de frecuencia:
        \begin{table}[H]
            \centering
            \begin{tabular}{|p{1cm}|p{1.8cm}|p{2cm}|p{1.8cm}|p{2cm}|p{2cm}|p{2cm}|}
                \hline
                Altura & Frecuencia absoluta & Frecuencia absoluta acumulada & Frecuencia relativa & Frecuencia relativa acumulada & Frecuencia porcentual & Frecuencia porcentual acumulada\\
                \hline\hline
                155 & 6 &  6 & 0.206 & 0.206 & 20.6\% & 20.6 \% \\\hline
                158 & 2 &  8 & 0.068 & 0.274 & 6.80\% & 27.4 \% \\\hline
                160 & 1 &  9 & 0.034 & 0.308 & 3.4 \% & 30.8 \% \\\hline
                162 & 1 & 10 & 0.034 & 0.342 & 3.4 \% & 34.2 \% \\\hline
                163 & 1 & 11 & 0.034 & 0.376 & 3.4 \% & 37.6 \% \\\hline
                164 & 1 & 12 & 0.034 & 0.41  & 3.4 \% & 41.0 \% \\\hline
                165 & 1 & 13 & 0.034 & 0.444 & 3.4 \% & 44.4 \% \\\hline
                166 & 2 & 15 & 0.068 & 0.512 & 6.80\% & 51.2 \% \\\hline
                167 & 1 & 16 & 0.034 & 0.546 & 3.4 \% & 54.6 \% \\\hline
                168 & 1 & 17 & 0.034 & 0.58  & 3.4 \% & 58.0 \% \\\hline
                169 & 1 & 18 & 0.034 & 0.614 & 3.4 \% & 61.4 \% \\\hline
                170 & 2 & 20 & 0.068 & 0.682 & 6.80\% & 68.2 \% \\\hline
                171 & 2 & 22 & 0.068 & 0.75  & 6.80\% & 75.0 \% \\\hline
                172 & 1 & 23 & 0.034 & 0.784 & 3.4 \% & 78.4 \% \\\hline
                175 & 2 & 25 & 0.068 & 0.852 & 6.80\% & 85.2 \% \\\hline
                176 & 1 & 26 & 0.034 & 0.886 & 3.4 \% & 88.6 \% \\\hline
                177 & 1 & 27 & 0.034 & 0.92  & 3.4 \% & 92.0 \% \\\hline
                180 & 2 & 29 & 0.068 & 0.998 & 6.80\% & 99.8 \% \\\hline
            \end{tabular}
        \end{table}
        \item La altura más frecuente en el grupo es de 155cm que representa el 20 \% de todas las alturas. \qedhere
    \end{enumerate}
\end{solution}

\begin{exercise}
    Los siguientes datos representan el peso (en kilogramo) de una muestra de productos en una fábrica:
    
    52, 55, 57, 60, 63, 64, 66, 68, 70, 71, 72, 74, 76, 78, 80, 70, 100, 70, 50, 70, 80, 70, 60, 50.
    \begin{enumerate}
        \item Organiza estos datos en una distribución de frecuencia con intervalos de clase de tamaño 5, comenzando desde 50.
        \item Calcular el peso mas común en la muestra seleccionada.
        \item Determine el 50\% de los pesos tomados de la muestra seleccionada.
    \end{enumerate}
\end{exercise}
\begin{solution}
    \begin{enumerate}
        \item Para una cantidad de 24 productos obtenemos $N = 24$ de pesos en kilogramos, tenemos la siguiente tabla de frecuencia.
        Cabe destacar que el límite inferior se incluye en el conteo de la frecuencia y el límite superior se excluye del conteo.
        \begin{table}[H]
            \centering
            \begin{tabular}{|p{1.7cm}|p{1.8cm}|p{2cm}|p{1.8cm}|p{2cm}|p{2cm}|p{2cm}|}
                \hline
                kilogramo & Frecuencia absoluta & Frecuencia absoluta acumulada & Frecuencia relativa & Frecuencia relativa acumulada & Frecuencia porcentual & Frecuencia porcentual acumulada\\
                \hline\hline
                 50 - 55   & 3 &  3 & 0.125 & 0.125 & 12.5 \% & 12.5 \% \\\hline
                 55 - 60   & 2 &  5 & 0.083 & 0.208 & 8.30 \% & 20.8 \% \\\hline
                 60 - 65   & 4 &  9 & 0.166 & 0.374 & 16.6 \% & 37.4 \% \\\hline
                 65 - 70   & 2 & 11 & 0.083 & 0.457 & 8.30 \% & 45.7 \% \\\hline
                 70 - 75   & 8 & 19 & 0.333 & 0.79  & 33.3 \% & 79.0 \% \\\hline
                 75 - 80   & 2 & 21 & 0.083 & 0.873 & 8.30 \% & 87.3 \% \\\hline
                 80 - 85   & 2 & 23 & 0.083 & 0.956 & 8.30 \% & 95.6 \% \\\hline
                 85 - 90   & 0 & 23 & 0.000 & 0.956 & 0.00 \% & 95.6 \% \\\hline
                 90 - 100  & 0 & 23 & 0.000 & 0.956 & 0.00 \% & 95.6 \% \\\hline
                100 - 105  & 1 & 24 & 0.041 & 0.997 & 4.10 \% & 99.7 \% \\\hline
            \end{tabular}
        \end{table}
        \item Para encontrar peso más frecuente en los datos, utilizaremos la moda de los mismo.
        Dicha moda está dada por
        \[
            M_o = L_i + \frac{f_i - f_{i - 1}}{(f_i - f_{i - 1}) + (f_i - f_{i + 1})}\cdot (L_s - L_i)
        \]
        donde $L_i$ y $L_s$ son los límites inferior y superior, respectivamente, del intervalo con mayor frecuencia.
        Es decir, para intervalo 70 - 75 con frecuencia 8.
        Por lo que tenemos
        \[
            M_o = 70 + \frac{8 - 2}{(8 - 2) + (8 - 2)}\cdot (75 - 70) = 70 + 5\cdot \frac{1}{2} = 72.5.
        \]
        Por lo tanto, el peso más frecuente es de 72.5 kg.
        \item Esta información la podemos obtener encontrando la mediana de la tabla de distribución, sabemos que la mediana está dada por $M_e = L_i + \frac{\frac{n}{2} - F_{i - 1}}{f_i}\cdot (L_s - L_i)$.
        Esto es, la posición $\frac{24}{2} = 12$, se encuentra en la intervalo 70 - 75, por lo tanto,
        \[
            M_e = L_i + \frac{\frac{n}{2} - F_{i - 1}}{f_i}\cdot (L_s - L_i) = 70 + \frac{12 - 11}{19}\cdot (75 - 70) \approx 70.26.
        \]
        Es decir, que la mitada de los pesos fueron mayores a 70.26 kg.\qedhere
    \end{enumerate}
\end{solution}

\begin{exercise}
    En los siguientes datos que representan los tiempos de entrega de paquetes en una tienda en línea:
    2, 1, 3, 4, 2, 3, 2, 1, 3, 2, 1, 3, 1, 3, 3, 3, 1, 4, 1, 3, 4, 1, 2, 3, 4, 5, 6, 8, 9, 3, 8, 10.

    Realiza lo siguiente:
    \begin{enumerate}
        \item Crea una tabla de distribución de frecuencia para estos datos.
        \item Calcula la media de los tiempos de entrega.
        \item Encuentra la mediana de los tiempos de entrega.
        \item Determina la moda de los tiempos de entrega.
    \end{enumerate}
\end{exercise}
\begin{solution}
    \begin{enumerate}
        \item Para una cantidad de $N = 44$ tiempos de entregas, tenemos la siguiente tabla de frecuencia.
        \begin{table}[H]
            \centering
            \begin{tabular}{|p{1.7cm}|p{1.8cm}|p{2cm}|p{1.8cm}|p{2cm}|p{2cm}|p{2cm}|}
                \hline
                Entrega & Frecuencia absoluta & Frecuencia absoluta acumulada & Frecuencia relativa & Frecuencia relativa acumulada & Frecuencia porcentual & Frecuencia porcentual acumulada\\
                \hline\hline
                 1 &  7 &  7 & 0.159 & 0.159 & 15.9 \% & 15.9 \% \\\hline
                 2 &  5 & 12 & 0.113 & 0.272 & 11.3 \% & 27.2 \% \\\hline
                 3 & 10 & 22 & 0.227 & 0.499 & 22.7 \% & 49.9 \% \\\hline
                 4 &  4 & 26 & 0.090 & 0.589 &  9.0 \% & 58.9 \% \\\hline
                 5 &  5 & 31 & 0.113 & 0.702 & 11.3 \% & 70.2 \% \\\hline
                 6 &  1 & 32 & 0.022 & 0.724 &  2.2 \% & 72.4 \% \\\hline
                 8 &  2 & 34 & 0.045 & 0.769 &  4.5 \% & 76.9 \% \\\hline
                 9 &  9 & 43 & 0.204 & 0.973 & 20.4 \% & 97.3 \% \\\hline
                10 &  1 & 44 & 0.022 & 0.995 &  2.2 \% & 99.5 \% \\\hline
            \end{tabular}
        \end{table}
        \item Sabemos que la formula de la media de los tiempos está dada por $\overline{x} = \sum_{i = 1}^{n} \frac{x_i \cdot f_i}{n}$, es decir
        \[
            \overline{x} = \frac{1\cdot 7 + 2\cdot 5 + 3\cdot 10 + 4\cdot 4 + 5 \cdot 5 + 6 \cdot 1 + 8 \cdot 2 + 9 \cdot 9 + 10 \cdot 1}{44} = \frac{201}{44} \approx 4.56
        \]
        \item Sabemos que la mediana está en la posición dada por $\frac{n}{2} = \frac{44}{2} = 22$ donde $n = 44$ es la cantidad de tiempos de entrega, luego la mediana es 3.
        \item Y claramente la moda sería el tiempo de entrega con más frecuencia, vemos la mayor frecuencia es 10, por tanto, la moda también es 3. \qedhere
    \end{enumerate}
\end{solution}

\begin{exercise}
    Los siguientes datos agrupados que representan las ventas diarias de una tienda de electrónica durante una semana.
    Las clases y sus respectivas frecuencias son las siguientes:
    \begin{table}[H]
        \centering
        \begin{tabular}{|c|c|c|}
            \hline
            \multicolumn{2}{|c|}{Intervalo} & \multicolumn{1}{|c|}{Frecuencia}\\
            \hline\hline
            10 & 12 & 4 \\\hline
            12 & 14 & 6 \\\hline
            14 & 16 & 8 \\\hline
            16 & 18 & 3 \\\hline
        \end{tabular}
    \end{table}
    Realizar lo siguiente:
    \begin{enumerate}
        \item Crea una tabla de distribución de frecuencia para estas clases.
        \item Calcula la media de ventas.
        \item Determine el 50\% de las ventas diarias en las tiendas electrónicas.
    \end{enumerate}
\end{exercise}
\begin{solution}
    \begin{enumerate}
        \item Tenemos la siguiente tabla de frecuencias.
        \begin{table}[H]
            \centering
            \begin{tabular}{|p{1.7cm}| c |p{1.8cm}|p{2cm}|p{1.8cm}|p{2cm}|p{2cm}|p{2cm}|}
                \hline
                Ventas & x & Frecuencia absoluta & Frecuencia absoluta acumulada & Frecuencia relativa & Frecuencia relativa acumulada & Frecuencia porcentual & Frecuencia porcentual acumulada\\
                \hline\hline
                10 - 12 & 11 & 4 &  4 & 0.190 & 0.190 & 19.0 \% & 19.0 \% \\\hline
                12 - 14 & 13 & 6 & 10 & 0.285 & 0.475 & 28.5 \% & 47.5 \% \\\hline
                14 - 16 & 15 & 8 & 18 & 0.380 & 0.855 & 38.5 \% & 85.5 \% \\\hline
                16 - 18 & 17 & 3 & 21 & 0.142 & 0.997 & 14.2 \% & 99.7 \% \\\hline
            \end{tabular}
        \end{table}
        \item La media de la ventas está dada por $\overline{x} = \sum_{i = 1}^{n} \frac{x_i \cdot f_i}{n}$, es decir
        \[
            \overline{x} = \frac{11\cdot 4 +  13 \cdot 6 + 15 \cdot 8 + 17 \cdot 3}{21} = \frac{293}{21} \approx 13.95.
        \]
        \item Esta información la encontramos con la mediana, sabemos que la mediana está dada por $M_e = L_i + \frac{\frac{n}{2} - F_{i - 1}}{f_i}\cdot (L_s - L_i)$.
        Esto es, la posición $\frac{21 + 1}{2} = 11$ (cómo es un número impar de datos), se encuentra en la intervalo 14 - 16, por lo tanto,
        \[
            M_e = L_i + \frac{\frac{n}{2} - F_{i - 1}}{f_i}\cdot (L_s - L_i) = 14 + \frac{11 - 10}{18}\cdot (16 - 14) \approx 14.11. \qedhere
        \]
        Es decir que el 50\% de las ventas fueron menores que 14.11.
    \end{enumerate}
\end{solution}