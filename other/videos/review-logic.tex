\documentclass[10pt]{beamer}
\usetheme{Boadilla}
\usepackage{ragged2e}
\justifying

\usepackage[utf8]{inputenc}
\usepackage{mlmodern}
\usepackage[T1]{fontenc}
\usepackage[spanish]{babel}

\newtheorem{definicion}{Definición}
\newtheorem{ejemplo}{Ejemplo}
\newtheorem{teorema}{Teorema}

\usepackage{amsthm}
\usepackage{amsmath}
\usepackage{amssymb}

\allowdisplaybreaks
\spanishdecimal{.}
\renewcommand{\qedsymbol}{$\blacksquare$}
\renewcommand{\emptyset}{\varnothing}
\renewcommand{\max}[1]{\ensuremath{máx #1}}
\newcommand{\modulo}[2]{\equiv #1\ (\text{mód}\ #2)}
\renewcommand{\proofname}{\textnormal{\textbf{Demostración}}}

\newcommand{\ie}{\ensuremath{\text{i.e.}}}
\newcommand{\eg}{\ensuremath{\text{e.g.}}}

%Number sets
\newcommand{\N}{\ensuremath{\mathbb{N}}}
\newcommand{\Z}{\ensuremath{\mathbb{Z}}}
\newcommand{\Q}{\ensuremath{\mathbb{Q}}}
\newcommand{\R}{\ensuremath{\mathbb{R}}}
\newcommand{\C}{\ensuremath{\mathbb{C}}}

\newcommand{\positiveSet}[1]{\ensuremath{#1^+}}
\newcommand{\myhrule}[1]{\vspace{#1mm}\hrule}
\newcommand{\negativeSet}[1]{\ensuremath{#1^-}}
\newcommand{\nonnegativeSet}[1]{\ensuremath{#1^{\geq 0}}}

%Useful math commands
\newcommand{\ds}{\displaystyle}
\newcommand{\invers}[1]{\frac{1}{#1}}
\newcommand{\inversD}[1]{\ensuremath{\dfrac{1}{#1}}}
\newcommand{\mcd}[2]{\ensuremath{mcd(#1, #2)}}
\newcommand{\refTheo}[1]{\textbf{Teorema #1}}
\newcommand{\refDef}[1]{\textbf{Definición #1}}
\newcommand{\ddotsr}{\ds\cdot^{\ds\cdot^{\ds\cdot}}}

\title{Repaso sobre proposiciones}
\author{J\_tinoco}
\date{Junio 2025}

\begin{document}
    \frame{\titlepage}
    \begin{frame}
        \frametitle{Un poco de lógica}
        La \textbf{lógica} es una manera sistemática de pensar que nos permite deducir información nueva a partir de
        información ya conocida, enriqueciendo así el significado de los enunciados.

        \vspace{2mm}
        La lógica juega un papel importante en matemáticas.

        \vspace{2mm}
        Con un proceso de deducción podemos obtener información tanto correcta como incorrecta.

        \begin{ejemplo}
            \begin{enumerate}
                \item Un círculo $\omega$ tiene radio 10.
                \item Si un círculo cualquiera tiene radio $r$, entonces su area es $\pi r^2$ unidades cuadradas.
                \item El área de $\omega$ es $100 \pi$ unidades cuadradas.
            \end{enumerate}
        \end{ejemplo}
    \end{frame}

    \begin{frame}
        \frametitle{Proposiciones}
        Una \textbf{proposición} es un enunciado que puede ser falso o verdadero.
        Podríamos pensar a las proposiciones como piezas de información que pueden ser correctas o incorrectas.
        Aplicamos la lógica sobre estas piezas con el fín de producir otras piezas de información (que también son proposiciones).

        \begin{ejemplo}
            \begin{itemize}
                \item Todos los triángulos acutángulos son isósceles.
                \item $\sqrt {2} \notin \Z$
                \item El conjunto $\{1, 3, 5, 45, 2025\}$ tiene cuatro elementos.
                \item Algunos triángulos rectángulos son isósceles.
                \item $\{1, 0, 2\} \cap \N = \emptyset$
                \item 5 = 2
                \item $\sqrt {2025} \in \irr$
                \item $\Z \subseteq \N$
                \item Cualquier número par es divisible por 2.
                \item $2025 \in Z$
            \end{itemize}
        \end{ejemplo}
    \end{frame}

    \begin{frame}
        \frametitle{Proposiciones}
        Ejemplos de oraciones que no son proposiciones.
        \begin{ejemplo}
            \begin{tabular}{l | p{6cm}}
                \hline
                No proposiciones & Proposiciones \\\hline\hline
                Sumando 5 a ambos lados. & Sumando 5 a ambos lados de $x - 5 = 2020$ obtenemos $x = 2025$. \\\hline
                $\Z$ & $1000 \in \Z$\\\hline
                2025 & 2025 no es un número.\\\hline
                ¿Cuál es la solución de $2x = 84$? & La solución de $2x = 84$ es 42.
            \end{tabular}
        \end{ejemplo}

        Es común denotar las proposiciones por letras, por ejemplo
        \begin{center}
            Q: La solución de $2x = 84$ es 42.
        \end{center}
    \end{frame}


    \begin{frame}
        \frametitle{Tipos de proposiciones}
        \begin{itemize}
            \item Hasta el momento hemos visto proposiciones que describen aspectos unitarios, a esto se conoce como \textbf{proposición simple}.
            \item Por otra parte, cuando hacemos combinaciones de proposiciones simples obtenemos nuevas proposiciones llamadas \textbf{proposiciones compuestas}.
        \end{itemize}

        \vspace{2mm}
        Conjunción.

        La letra \textbf{``y''} puede usarse para combinar proposiciones en una nueva proposición llamada \textbf{conjunción}.
        Ejemplo,
        \begin{center}
            P: El número 2 es par y el número 3 es impar.
        \end{center}
        Tabla de verdad
        \begin{center}
            \begin{tabular}{| c | c || c |}
                \hline
                A & B & P: A $\land$ B \\\hline\hline
                V & V & V\\\hline
                V & F & F\\\hline
                F & V & F\\\hline
                F & F & F\\\hline
            \end{tabular}
        \end{center}
    \end{frame}

    \begin{frame}
        \frametitle{Tipos de proposiciones}
        Disyunción.

        Así mismo, la letra \textbf{``o''} nos permite obtener proposición llamadas \textbf{disyunciones}.
        Ejemplo,
        \begin{center}
            P: El número 100 es impar o el número 2025 es par.
        \end{center}
        Tabla de verdad
        \begin{center}
            \begin{tabular}{| c | c || c |}
                \hline
                A & B & P: A $\lor$ B \\\hline\hline
                V & V & V\\\hline
                V & F & V\\\hline
                F & V & V\\\hline
                F & F & F\\\hline
            \end{tabular}
        \end{center}

        \vspace{1mm}
        Negación.

        Consiste en cambiar el valor de verdad de una proposición por su opuesto.
        \begin{center}
            \begin{tabular}{| c || c |}
                \hline
                P & $\neg$ P \\\hline\hline
                V & F\\\hline
                F & V\\\hline
            \end{tabular}
        \end{center}

    \end{frame}

    \begin{frame}
        \frametitle{Tipos de proposiciones}
        Implicación (Si, entonces).

        Una proposición muy importante es la implicación, está nos permite deducir el valor de una proposición con base
        al valor de otra.
        También conocida como proposición condicional. Ejemplo,
        \begin{center}
            R: Si el entero $a$ es múltiplo de seis, entonces $a$ es divisible por dos.
        \end{center}
        Tabla de verdad
        \begin{center}
            \begin{tabular}{| c | c || c |}
                \hline
                A & B & P: A $\implies$ B \\\hline\hline
                V & V & V\\\hline
                V & F & F\\\hline
                F & V & V\\\hline
                F & F & V\\\hline
            \end{tabular}
        \end{center}
    \end{frame}

    \begin{frame}
        \frametitle{Más sobre las implicación}
    \end{frame}


    \begin{frame}
        \frametitle{Ejercicios}
    \end{frame}

    \begin{frame}
        \frametitle{Bibliografía}
    \end{frame}
\end{document}