\section{Polynomial questions}

Polynomial Exam Questions por T. Madas.
Documento encontrado en internet.

\begin{exercise}[*]
    Multiplicar y simplificar
    \[
        (2x^2 - x - 3)(1 + 2x - x^2),
    \]
    escribir la respuesta en potencias ascendentes de $x$.

    Respuesta, $\boxed{-3 - 7x + 3x^2 + 5x^3 - 2x^4}$.
\end{exercise}

\begin{exercise}[*]
    Sea $f(x) = x^3 - 3x^2 + 6x - 40$.
    \begin{tasks}[label=\alph*.](1)
        \task Probar que $(x - 5)$ no es factor de $f(x)$.
        \task Encontrar un factor lineal de $f(x)$.
    \end{tasks}
    Respuesa, $\boxed{(x - 4)}$.
\end{exercise}

\begin{exercise}
    Sea $f(x) = 3x^3 - 2x^2 - 12x + 8$.
    \begin{tasks}[label=\alph*.](1)
        \task Usar el teorema del factor para probar que $(x + 2)$ es factor de $f(x)$.
        \task Factorizar $f(x)$ completamente.
    \end{tasks}
    Respuesta, $\boxed{(3x - 2)(x - 2)(x + 2)}$.
\end{exercise}

\begin{exercise}
    El polinomio $x^3 + 4x^2 + 7x + k$, donde $k$ es una constante, es denotado por $f(x)$.
    \begin{tasks}[label=\alph*.](1)
        \task Dado que $(x + 2)$ es un factor de $f(x)$, probar que $k = 6$.
        \task Expresar $f(x)$ como el producto de un factor lineal y uno cuadrático.
    \end{tasks}
    Respuesta, $\boxed{(x + 2)(x^2 + 2x + 3)}$.
\end{exercise}

\begin{exercise}
    Usar el teorema del factor para demostrar que $(x + 3)$ es un factor de $x^3 + 5x^2 - 2x - 24$.
    Luego, factorize completamente el polinomio.

    Respuesta, $\boxed{(x + 3)(x - 2)(x + 4)}$.
\end{exercise}

\begin{exercise}[*]
    Encontrar el coeficiente de $x^3$ en la expansión de
    \[
        (2x^3 - 5x^2 + 2x - 1)(3x^3 + 2x^2 - 9x + 7).
    \]

    Respuesta, $\boxed{60x^3}$.
\end{exercise}

\begin{exercise}[*]
    Multiplicar y simplificar
    \[
        (1 + x)(1 + x^2)(1 - x + x^2)
    \]
    escribir la respuesta en potencias ascendentes de $x$.

    Respuesta, $\boxed{1 + x^2 + x^3 + x^5}$.
\end{exercise}

\begin{exercise}
    Usar el teorema del factor para probar que $(x - 5)$ es un factor de $x^3 - 19x - 30$.
    Luego, factorize el polinomio en tres factores lineales.

    Respuesta, $\boxed{(x + 3)(x + 2)(x - 5)}$.
\end{exercise}

\begin{exercise}
    Sea $f(x) = ax^3 - x^2 - 5x + b$, donde $a$ y $b$ son constantes.
    Tal que, cuando $f(x)$ se divide por $(x - 2)$ y $(x + 2)$ los restos son 36 y 40 respectivamente.
    Hallar el valor de $a^{b + 2024}$.

    Respuesta, 1.
\end{exercise}

\begin{exercise}
    Un polinomio cúbico es definido en términos de la constante $k$ como
    \[
        P(x) = x^3 + x^2 - x + k,\quad x \in \R.
    \]
    Dado que $(x - k)$ es un factor de $P(x)$ determinar los posibles valores de $k$.

    Respuesta, $\boxed{k = -1, 0}$.
\end{exercise}

\begin{exercise}
    Sea $G(x) = x^3 - 2x^2 + kx + 6$, donde $k$ es una constante.
    \begin{tasks}[label=\alph*.](1)
        \task Dado que $(x - 3)$ es un factor de $G(x)$, probar que $k = -5$.
        \task Factorizar $G(x)$ en tres factores lineales.
        \task Hallar el resto cuando $G(x)$ es dividido por $(x + 3)$.
    \end{tasks}
    Respuesta, $\boxed{(x - 1)(x + 2)(x - 3)}$ y $\boxed{R = -24}$.
\end{exercise}

\begin{exercise}
    Use el teorema del factor para demostrar $(x + 2)$ es factor de $2x^3 + 3x^2 - 5x - 6$.
    Luego, factorize el polinomio en tres factores lineales.

    Respuesta, $\boxed{(x + 1)(x + 2)(2x - 3)}$.
\end{exercise}

\begin{exercise}
    Sea $H(x) = 2x^3 - 7x^2 - 5x + 4$.
    \begin{tasks}[label=\alph*.](1)
        \task Encontrar el resto cuando $H(x)$ es dividido por $(x + 2)$.
        \task Usar el teorema del factor para probar que $(x - 4)$ es un factor de $H(x)$.
        \task Factorizar completamente $H(x)$.
    \end{tasks}

    Respuesta, $\boxed{R = -30}$ y $\boxed{(2x - 1)(x + 1)(x - 4)}$.
\end{exercise}

\begin{exercise}
    Sea $g(x) = x^3 + x^2 + ax + b$, donde $a$ y $b$ son constantes.
    Tal que, cuando $g(x)$ se divide por $(x - 2)$ y $(x + 1)$ los restos son $-7$ y 32 respectivamente.
    Hallar los valores de $a$, $b$ y demostrar que $(x - 3)$ es factor de $g(x)$.

    Respuesta, $\boxed{a=-17}$, $\boxed{b = 15}$.
\end{exercise}

\begin{exercise}
    Sea $R(x) = px^3 - 32x^2 - 10x + q$, donde $p$ y $q$ son constantes.
    Cuando $R(x)$ es dividido por $(x - 2)$ el resto es exactemente igual cuando $R(x)$ es dividido por $(2x + 3)$.
    Probar que $p = 8$.
\end{exercise}

\begin{exercise}
    Resolver la ecuación
    \[
        x^3 + x^2 - (x - 1)(x - 2)(x - 2) = 12.
    \]

    Respuesta, $\boxed{x = -\frac{3}{7}, 2}$.
\end{exercise}

\begin{exercise}
    Sea $P(x) = 3x^3 - 2x^2 - 12x + 8$.
    \begin{tasks}[label=\alph*.](1)
        \task Encontrar el resto cuando $P(x)$ es divido por $(x - 4)$.
        \task Dado que $(x - 2)$ es un factor de $P(x)$ resolver la ecuación $P(x) = 0$.
    \end{tasks}
    Respuesta, $\boxed{R = 120}$ y $\boxed{x = -2, \frac{2}{2}, 2}$.
\end{exercise}