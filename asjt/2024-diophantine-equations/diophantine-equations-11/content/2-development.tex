\section{Desarrollo}

\subsection{Definiciones}

\begin{definition.box}{Fracción continua generalizada}{}
    Definiremos a una fracción continua generalizada como una expresión de la forma:
    \[
        a_0 + \dfrac{b_0}{a_1 + \dfrac{b_1}{a_2 + \dfrac{b_2}{a_{n - 2} + \dfrac{\ddots}{a_{n - 1} + \dfrac{b_{n -1}}{a_n}}}}}
    \]
    donde los $a_i$ y $b_i$, con $i = 0,1, \ldots, n$, son números reales.
\end{definition.box}
Los $a_i$ y $b_i$ se llamarán términos de la fracción continua, es fácil concluir que, es posible encontrar fracciones continuas con una cantidad finita e infinita de términos.
Cuando todos los $b_i$ son iguales a 1 se forman un tipo de fracciones continuas importantes.

\begin{definition.box}{Fracción continua simple}{}
    Si todo $b_i = 1$ y para $i \geq 1$ se tiene que $a_i$ es positivo, entonces la expresión
    \[
        a_0 + \dfrac{1}{a_1 + \dfrac{1}{a_{n - 2} + \dfrac{\ddots}{a_{n - 1} + \dfrac{1}{a_n}}}}
    \]
    se llamará fracción continua simple, y la denotaremos por $[a_0; a_1, a_2, \ldots, a_n]$
\end{definition.box}

\begin{example}
    Hallar la fracción continua simple de $\frac{59}{11}$.
\end{example}
\begin{solution}
    Primero tratamos de llevar la fracción a una fracción propia, es decir
    \[
        \frac{59}{11} = 5 + \frac{4}{11}.
    \]
    Tomamos la fracción $\frac{4}{11}$, la invertimos y obtenemos
    \[
        \frac{59}{11} = 5 + \frac{1}{\frac{11}{4}}.
    \]
    Y ahora realizamos el mismo procedimiento con $\frac{11}{4}$,
    \[
        \frac{59}{11} = 5 + \frac{1}{2 + \frac{3}{4}} = 5 + \frac{1}{2 + \frac{1}{\frac{4}{3}}}.
    \]
    Realizamos el mismo procedimiento con $\frac{4}{3}$,
    \[
        \frac{59}{11} = 5 + \frac{1}{2 + \frac{1}{1 + \frac{1}{3}}}.
    \]
    Se termina el proceso cuando se llega a una fracción con númerador igual a 1, en este caso es $\frac{1}{3}$.
    Luego, podemos concluir que $\frac{59}{11}$ está representada por la fracción continua simple finita $[5; 2, 1, 3]$ y esta representación es única.
\end{solution}
Cuando la fracción ya es propia, el término $a_0$ es igual a cero, por ejemplo:
\[
    \frac{6}{11} = 0 + \frac{1}{1 + \frac{1}{1 + \frac{1}{5}}} \implies \frac{6}{11} = [0; 1,1,5].
\]
Es claro que para un número $[a_0] = a_0$ y para una fracción continua simple infinita se tiene $[a_0; a_1, a_2, \ldots]$.
Esto toma relevancia a partir del siguiente teorema.

\begin{theorem.box}{}{}
    Si $x$ es número racional, entonces $x$ se puede expresar como una fracción continua simple con una cantidad finita de términos.
\end{theorem.box}

Del teorema anterior podemos deducir el siguiente resultado.

\begin{corollary}
    Toda fracción continua simple con cantidad infinita de términos representa un número irracional.
\end{corollary}

Cabe mencionar que el corolario anterior implica que no todo real puede expresarse como una fracción continua.
Además, los números irracionales que sí pueden ser expresados presentan una característica, esto es que sus términos
se repiten de manera cíclica, por ejemplo, $[1; 1, 2,1,2,1,2, \ldots]$.

\begin{definition.box}{Fracción continua periódica}{}
    Definiremos como fracción continua periódica a una fracción continua simple infinita de la forma
    \[
        [a_0; a_1, a_2, \ldots, a_{n -1}, \overline{a_n, a_{n + 1},\ldots, a_{n + k - 1}}],
    \]
    donde $k \geq 1$, el periodo es la sucesión de términos $a_{n}, a_{n + 1},\ldots, a_{n + k  - 1}$ y la longitud del periodo es $k$.
\end{definition.box}

Así, el ejemplo anterior sería $[1; 1, 2,1,2,1,2, \ldots] = [1; \overline{1,2}]$ con periódo $k = 2$, que además representa el número $\sqrt {3}$.
Cuando se tiene que $n = 0$ diremos que la fracción continua $[\overline{a_0;a_1, \ldots, a_{k - 1}}]$ es una fracción continua periódica pura.

Con la definición anterior, podemos entender el siguiente resultado.

\begin{theorem.box}{}{}
    Todo irracional cuadrático puede ser expresado como una fracción continua periódica.
\end{theorem.box}

Donde un irracional cuadrático es un número que es una raíz no racional de una ecuación cuadrática, es decir, un número de la forma $\frac{a + b\sqrt {d}}{c}$ donde $a,b \in \Z$, con $d$ un entero positivo no cuadrado perfecto y $c$ un entero distinto de cero.

\begin{definition.box}{Convergentes $c_k$}{}
    Los convergentes de la fracción continua simple $[a_0; a_1, a_2, \ldots]$ son las fracciones continuas simples finitas
    \[
        c_k = [a_0; a_1, a_2, \ldots, a_k],\ \forall k \in \positiveSet{\Z}.
    \]
\end{definition.box}

La definición de $c_k$ dice que son fracciones continuas finitas, por lo cual estos convergentes representan números racionales,
entonces, podemos decir que $c_k = \frac{p}{q}$, este hecho nos permite encontrar resultados importantes, como por ejemplo los siguientes teoremas.

\begin{theorem.box}{}{}
    Si $c_n = \frac{p_n}{q_n}$ es el $n-$ésimo convergente de la fracción continua simple $[a_0; a_1, a_2, \ldots]$, entonces
    \begin{align*}
        &p_0 = a_0, && q_0 = 1,\\
        &p_1 = a_1 p_0 + 1, && q_1 = a_1,\\
        &p_k = a_k p_{k - 1} + p_{k - 2},\ \forall k \geq 2 && q_k = a_k q_{k - 1} + q_{k - 2},\ \forall k \geq 2
    \end{align*}
\end{theorem.box}

\begin{theorem.box}{}{}
    Si $c_n = \frac{p_n}{q_n}$ es el $n-$ésimo convergente de una fracción continua simple infinita, entonces
    \[
        p_n q_{n - 1} - p_{n - 1}q_n = (- 1)^{n-1}, \ \text{con}\ k \in \Z^{\geq 1}.
    \]
\end{theorem.box}
Del teorema anterior es posible determinar que todo convergente de una fracción continua representa una fracción irreducible, este echo será de útilidad al momento de resolver ecuaciones diofánticas.
\begin{corollary}
    Si $c_k = \frac{p_k}{q_k}$ es un convergente de una fracción continua simple infinita, entonces $\mcd{p_k}{q_k} = 1$.
\end{corollary}



\subsection{Resolución de ecuación diofánticas}

\subsubsection{Ecuaciones lineales}

Dada la ecuación
\[
    ax + by = c,\ \text{con}\ a,b,c \in \Z
\]
y $\mcd a b = 1$.
Consideramos la fracción $\frac{a}{b}$, dicha fracción puede ser expresada como una fracción continua finita de $n$ términos.
Sabiendo esto, es claro que $c_n = \frac{a}{b}$ y usando el teorema 1.4 obtenemos que
\[
    a\cdot q_{n - 1} - b\cdot p_{n - 1} = (-1)^{n- 1}
\]
multiplicando esta expresión por $\left[c \cdot (-1)^{n - 1}\right]$ y ordenado los signos, obtenemos
\[
    a \left[c\cdot q_{n - 1} (-1)^{n - 1}\right] + b \left[c \cdot p_{n - 1} (-1)^n\right] = c
\]
con lo cual las soluciones de la ecuación estaría dada por
\[
    x = c \cdot q_{n - 1} (-1)^{n - 1} \quad \text{y} \quad y = c \cdot p_{n - 1} (-1)^{n}.
\]
Es decir, que al calcular $c_{n - 1}$ podemos resolver la ecuación diofántica lineal.

\begin{example}
    Resolver la ecuación $7x + 11y = 25$.
\end{example}
\begin{solution}
    Consideramos la fracción $\frac{7}{11}$, esta fracción está representada por la fracción continua $[0;1,1,1,3]$, donde $n = 4$.
    Para obtener los convergentes de la fracción podemos utilizar la siguiente tabla, esta tabla se formo con la ayuda del teorema 1.3.
    \begin{table}[H]
        \centering
        \begin{tabular}{c||c|c|c|c|c}
            $n$ & 0 & 1 & 2 & 3 & 4 \\\hline\hline
            $a_n$ & 0 & 1 & 1 & 1 & 3\\
            $p_n$ & 0 & 1 & 1 & 2 & 7\\
            $q_n$ & 1 & 1 & 2 & 3 & 11\\
        \end{tabular}
    \end{table}
    Por lo cual, una solución para la ecuación está determinada por $c_3 = \frac{p_3}{q_3} = \frac{2}{3}$, es decir
    \begin{align*}
        x & = 25 \cdot (3) (-1)^{4 - 1} = -75 \\
        y & = 25 \cdot (2) (-1)^{4} = 50
    \end{align*}
    Así, la solución general es $(x,y) = (-75 +11t, 50 - 7t)$ para todo entero $t$.
\end{solution}

De esta manera podemos resolver ecuaciones lineales por medio de pasos estructurados, estos sería los pasos para
calcular la fracción continua que produce la ecuación.
Cabe resaltar que con este tema hemos visto tres maneras de resolver ecuaciones diofánticas lineales; por congruencias, aplicando el algoritmo de Euclides y fracciones continuas.

\subsubsection{Ecuaciones de Pell}

Sabemos que la ecuación de Pell
\[
    x^2 - d y^2 = 1
\]
siempre tiene una solución mínima $(x_0, y_0)$.
Resulta ser que dicha solución mínima está dada por el convergente
\[
    c_{k - 1} = \frac{p_{k - 1}}{q_{k - 1}} = \frac{x_0}{y_0}
\]
de la fracción continua del número $\sqrt {d}$,
donde $k$ es el periodo de la fracción continua.
Teniendo presente que $\sqrt {d}$ es un número irracional cuadrático se cumple que es equivalente a una fracción continua periódica.

\begin{example}
    Resolver la ecuación $x^2 - 7y^2 = 1$.
\end{example}
\begin{solution}
    Se tiene que $\sqrt {7}$ está representada por la fracción continua $[2;\overline{1,1,1,4}]$, entonces el periodo es $k = 4$.
    Considerando la siguiente tabla, podemos obtener los primeros 5 convergentes de la fracción.
    \begin{table}[H]
        \centering
        \begin{tabular}{c||c|c|c|c|c}
            $k$ & 0 & 1 & 2 & 3 & 4 \\\hline\hline
            $a_k$ & 2 & 1 & 1 & 1 & 4\\
            $p_k$ & 2 & 3 & 5 & 8 & 37\\
            $q_k$ & 1 & 1 & 2 & 3 & 14\\
        \end{tabular}
    \end{table}
    Por lo cual, una solución para esta ecuación estaría dada por
    \begin{align*}
        c_{4 - 1} = c_3 = \frac{p_3}{q_3} = \frac{8}{3} = \frac{x_0}{y_0}.
    \end{align*}
    Así, la solución mínima es $(x,y) = (8,3)$.
\end{solution}


\subsection{Ejercicios y problemas}

Ejercicios y problemas para el autoestudio.

\begin{exercise}
    Encontrar las fracciones continuas de las siguientes fracciones.
    \begin{multicols}{3}
        \begin{enumerate}
            \item $\dfrac{45}{37}$
            \item $\dfrac{51}{25}$
            \item $\dfrac{43}{38}$
            \item $\dfrac{120}{84}$
            \item $\dfrac{35}{14}$
        \end{enumerate}
    \end{multicols}
\end{exercise}

\begin{problem}
    Calcular el valor de
    \[
        \sqrt[8]{2207 - \frac{1}{2207 - \frac{1}{2207 - \ddots}}}.
    \]
    Expresar la respuesta con forma $\frac{a + b\sqrt{c}}{d}$ con $a,b,c,d \in \Z$.
\end{problem}