\begin{problem-wos}
    Sea $P(x)$ un polinomio de grado impar.
    Probar que la ecuación
    \[
        P(P(x)) = 0,
    \]
    tiene al menos tantas raíces reales como la ecuación $P(x) = 0$.
\end{problem-wos}

\begin{problem-wos}
    El polinomio $P(x)$ de grado $n$ satisface $P(i) = \frac{1}{i + 1}$ para $i = 0,1, \ldots, n$.
    Hallar $P(n + 1)$.
\end{problem-wos}

\begin{problem-wos}
    Dado el polinomio $P(x)$ de grado tres.
    Llamaremos a una tripleta de reales distintos $(a, b, c)$ como \textit{cíclica} si $P(a) = b$, $P(b) = c$ y $P(c) = a$.
    Dadas cuatro tripletas cíclicas distintas, probar que no pueden tener la misma suma en sus coordenadas.
\end{problem-wos}

\begin{problem-wos}
    Hans y David juegan un juego.
    Hans piensa en un polinomio $P(x)$ de grado diez.
    En cada turno David dice 10 números distintos, Hans toma uno de esos números, calcula el valor en el polinomio y le dice a David el valor.
    David no conoce qué número tomo Hans.
    ¿Cuántos turnos son necesarios para que David deduzca el polinomio de Hans?
\end{problem-wos}

\begin{exercise}
    Resolver el siguiente sistema en números reales
    \[
        \begin{cases}
            & a^2 + b^2 = 2c\\
            & 1 + a^2 = 2ac\\
            & c^2 = ab
        \end{cases}
    \]
\end{exercise}

\begin{exercise}
    Sean $a_1, a_2, \ldots, a_{2025}$ números reales y para cada entero $1  \leq i \leq 2025$ sea
    \[
        S_n = a_1 + a_2 + \ldots + a_{n}.
    \]
    Sí $a_1 = 2025$ y se cumple $S_n = n^2 a_n$ para todo $n$, determinar el valor de $S_{2025}$.
\end{exercise}