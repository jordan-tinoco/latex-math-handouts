\section{Fundamentos}

En esta segunda sesión repasaremos aspectos fundamentales sobre los polinomios que debemos conocer, veremos conceptos de
polinomios, divisiones, raíces y una serie de teoremas importantes.

\subsection{Conceptos}

\begin{definition}
    Un \textit{monomio} en la variable $x$ es una expresión $c x^k$ donde $c$ es una constante y $k$ un entero no negativo.
\end{definition}

Un polinomio es la suma de finitos monomios.
En otras palabras un polinomio es una expresión de la forma
\[
    a_n x^n + a_{n - 1} x^{n - 1} + \ldots + a_1 x + a_0
\]
Asumamos que $a_n \neq 0$.
En este caso, los números $a_n, a_{n - 1}, \ldots, a_1, a_0$ se llaman los coeficientes del polinomio, y $n$ es llamado el grado del polinomio.

Los polinomios pueden ser sumados y multiplicados.
Para $A(x) = a_0 + a_1 x + a_2 x^2 + \ldots +a_n x^n$ y $B(x) = b_0 + b_1 x + b_2 x^2 + \ldots + b_n x^n$ definimos
\begin{align*}
    A(x) + B(x) &= (a_0 + b_0) + (a_1 + b_1)x + (a_2 + b_2)x^2 + \ldots\\[2mm]
    A(x)B(x) &= a_0 b_0 + (a_0 b_1 + a_1 b_0)x + (a_0 b_2 + a_1 b_1 + a_2 b_0)x^2 + \ldots
\end{align*}
Para los motivos de este documento consideraremos a los polinomios con coeficientes reales, racionales, enteros, complejos o incluso
valores que son residuos en módulo de algún primo $p$.

\subsection{División de polinomios}

\begin{definition}
    Para los polinomios $F(x)$ y $G(x)$, llamamos \textit{cociente} y $\textit{resto}$ a los polinomios $Q(x)$ y $R(x)$,
    respectivamente, si
    \[
        F(x) = Q(x) G(x) + R(x)
    \]
    y $\deg{R} < \deg{G}$.
\end{definition}

\begin{theorem}[División polinómica]
    El cociente y resto siempre existen y son únicos.
\end{theorem}

\begin{theorem}[Bezout version 1]
    El resto de $P(x)$ dividido por $(x - a)$ es igual a $P(a)$.
\end{theorem}

\begin{theorem}[Bezout version 2]
    Un número $a$ es raíz de $P(x)$ si y solo si $(x - a) | P(x)$.
\end{theorem}

\begin{corollary}
    Si $a_1, a_2, \ldots, a_n$ son raíces distintas de $P(x)$, entonces
    \[
        (x - a_1)(x - a_2)\ldots(x - a_n) | P(x).
    \]
\end{corollary}

\begin{theorem}
    El polinomio $P(x)$ con grado $n$ tiene a lo más $n$ raíces.
\end{theorem}

\begin{corollary}
    Si $A(x)$ y $B(x)$ no son iguales, y su grado es a lo máximo $n$, entonces la ecuación $A(x) = B(x)$ tiene a lo sumo $n$ raíces.
\end{corollary}

\begin{example}
    Probar que
    \[
        a \frac{(x - b)(x - c)}{(a - b)(a - c)} + b \frac{(x - c)(x - a)}{(b - c)(b - a)} + c \frac{(x - a)(x - b)}{(c - a)(c - b )} = x.
    \]
\end{example}

\begin{solution}
    Denotemos por $P(x)$ al lado izquierdo de la ecuación.
    Sabemos que $P(x)$ es un polinomio con grado a lo sumo 2 y $P(a) = a$, $P(b) = b$ y $P(c) = c$.
    Por tanto, por el corolario previo $P(x) = x$.
\end{solution}

\begin{example}
    Dado el entero positivo $n$.
    El polinomio $P(x)$ satisface $P(i) = 2^i$ para todo $i = 0, 1, \ldots, n$.
    Probar que $\deg{P} \geq n$.
\end{example}

\begin{solution}
    Considere el polinomio $Q(x) = 2P(x) - P(x + 1)$.
    Es obvio que $\deg{Q} = \deg{P}$.
    Y los números $0, 1, \ldots, n-1$ son raíces de $Q$, por lo cual $\deg{Q} \geq n$.
\end{solution}

\begin{example}
    Dado el polinomio $P(x)$ con grado tres.
    Llamaremos a una tripleta de números reales $(a, b, c)$ \textit{cíclica} si $P(a) = b$, $P(b) = c$ y $P(c) = a$.
    Probar que existen a lo más nueve tripletas cíclicas.
\end{example}

\begin{solution}
    Dividamos la solución
    \begin{enumerate}
        \item Tripletas cíclicas diferentes no tienen elementos compartidos.
        Supongamos lo contrario y que hay dos tripletas cíclicas con números iguales $(a, b, c)$ y $(a, d, e)$.
        Con base en la definición de tripleta cíclica, $b = P(a)$ y $d = P(a)$, por tanto $d = b$.
        Y $c = P(b)$ y $e = P(d) = P(b)$, por tanto $c = e$.
        Esto implica que las tripletas son iguales.
        \item Todos los números en cualquier tripleta cíclica son raíces del polinomio $Q(x) = P(P(P(x))) - x$.
        Consideremos cualquier tripleta cíclica $(a, b, c)$.
        \[
            P(P(P(a))) = P(P(b)) = P(c) = a.
        \]
        \item El grado del polinomio $P(P(P(x))) - x$ es 27, ya que si existen 10 tripletas cíclicas distintas, entonces existen 30 raíces distintas para $Q(x)$.
        Lo cual es absurdo. \qedhere
    \end{enumerate}
\end{solution}

\subsection{Raíces}

\begin{theorem}[Teorema Fundamental del álgebra]
    Cualquier polinomio no constante con coeficientes complejos tiene al menos una raíz compleja.
    Equivalentemente, todo polinomio de grado $n \geq 1$ con coeficientes complejos pude factorizarse como
    \[
        P(z) = a_n (z - z_1)(z - z_2) \ldots (z - z_n),
    \]
    donde $a_n \neq 0$, y $z_1, z_2, \ldots, z_n \in \C$ (considerando la multiplicidad).
\end{theorem}

\begin{example}
    Dado el polinomio $P(x) \in \R [x]$ tal que $P(x) \geq 0$ para todo real $a$.
    Probar que existen polinomios $Q(x), R(x) \in \R [x]$ para el cual $P(x) = Q(x)^2 + R(x)^2$.
\end{example}

\begin{solution}
    Sea $P(x) \in \R[x]$ sea un polinomio con coeficientes reales.
    Por el teorema Fundamental del álgebra, $P(x)$ puede factorizarse como
    \[
        P(x) =  c \prod{j = 1}{c} (x - \alpha_j),
    \]
    donde $c \in \R$ y $\alpha_j \in \C$.
    Ya que $P(x)$ tiene coeficientes reales, cualquier raíz no real $\alpha$ debe tener su conjugado complejo $\overline{\alpha}$ en la factorización.
    De esta manera, los factores reales de $P(x)$ son productos de
    \begin{itemize}
        \item Factores lineales: $(x - r)^{2k}$, correspondientes a raíces reales.
        \item Factores cuadráticos irreducibles: $a x^2 + bx + c$ con $b^2 - 4ac < 0$.
    \end{itemize}
    Si consideramos un polinomio cuadrático $x^2 + bx + c$ sin raíces reales (es decir $b^2 - 4c < 0$) y completando cuadrados, obtenemos que
    \begin{align*}
        x^2 + bx + c = \left(x + \frac{b}{2}\right)^2 + \left(c -\frac{b^2}{4}\right) = \left(x + \frac{b}{2}\right)^2 + \frac{4c - b^2}{4}.
    \end{align*}
    Esto puede escribirse como
    \[
        x^2 + bx + c = \left(x + \frac{b}{2}\right)^2 + \left(\frac{\sqrt{4c - b^2}}{2}\right)^2
    \]
    Por la identidad de Brahmagupta sabemos que el producto de dos sumas de dos cuadrados es igual a la suma de dos cuadrados, esto es
    \[
        (a^2 + b^2)(c^2 + d^2) = (ac - bd)^2 + (ad + bc)^2
    \]
    Así que cada factor de $P(x)$ puede ser representado como la suma de dos cuadrados, por lo cual, $P(x)$ puede ser escrito de la forma
    \[
        P(x) = Q(x)^2 + R(x)^2
    \]
    con $Q(x), R(x) \in \R[x]$.
\end{solution}

\subsubsection{Raíces en intervalos}
Aquí haremos uso de algunos teoremas de cálculo y sus corolarios.

\begin{theorem}[Teorema del valor intermedio]
    Sea $f(x)$ una función continua en un intervalo cerrado $[a, b]$, asumamos que se toman los valores extremos con signos opuestos, es decir
    \[
        f(a) \cdot \f(b) < 0,
    \]
    entonces existe un punto $c \in (a, b)$ tal que $f(c) = 0$.
\end{theorem}



\subsection{Interpolación de Lagrange}

\begin{theorem}
    Para reales distintos $x_0, x_1, \ldots, x_n$ y cualesquiera $y_0, y_1, \ldots, y_n$ definimos
    \[
        P_i(x) = \frac{(x - x_0)(x - x_1)\ldots(x - x_{i - 1})(x - x_{i + 1})\ldots(x - x_n)}{(x_i - x_0)(x_i - x_1)\ldots(x_i - x_{i - 1})(x_i - x_{i + 1})\ldots(x_i - x_n)},
    \]
    así, el único polinomio $P(x)$ con grado como máximo $n$ tal que $P(x_i) = y_i$ para todo $i = 0, 1, \ldots, n$ es igual a
    \[
        y_0 P_0(x) + y_1 P_1(x) + \ldots + y_n P_n(x)
    \]
\end{theorem}

\begin{example}
    Dado el polinomio $P(x)$ con coeficientes reales y grado 10.
    ¿Cuál es el mayor número de intersecciones que $y = P(x)$ y el círculo $x^2 + y^2 = 1$ puede tener?
\end{example}

\begin{solution}
    La respuesta es 20.
    Supongamos que hay más de 20 intersecciones, considerando el polinomio $Q(x) =  x^2 + P(x)^2 - 1$.
    Si $(x_0, y_0)$ es un punto de intersección, entonces $x_0$ es raíz de $Q(x)$.
    Pero $\deg{(Q)} = 20$, así que solo pueden existir a lo máximo 20 valores $x_0$.
\end{solution}