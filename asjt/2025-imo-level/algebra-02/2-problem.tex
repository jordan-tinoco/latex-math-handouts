\section{Problemas}

\begin{problem}
    Dado el polinomio $P(x) \in \R [x]$ tal que $P(x) \geq 0$ para todo real $a$.
    Probar que existen polinomios $Q(x), R(x) \in \R [x]$ para el cual $P(x) = Q(x)^2 + R(x)^2$.
\end{problem}

\begin{problem}
    Encontrar todos los polinomios $P(x) \in \R[x]$ tales que
    \[
        (x - 3)P(x + 1) = (x + 1)P(x).
    \]
\end{problem}

\begin{problem}
    Probar que un polinomio de grado impar tiene al menos una raíz real.
\end{problem}

\begin{problem}
    ¿Existe un polinomio cuadrático $P(x)$ tal que dos de sus coeficientes son enteros y
    \[
        P\left(\frac{1}{2024}\right) = \frac{1}{2025},  \quad P\left(\frac{1}{2025}\right) = \frac{1}{2024} ?
    \]
\end{problem}

\begin{problem}
    Sean $a,b, c$ números reales distintos tales que para el polinomio cuadrático $f(x)$ se tiene
    \[
        f(a) = ab, \quad f(b) = ac, \quad f(x) = ab.
    \]
    Probar que $f(a + b + c) = ab + bc + ca$.
\end{problem}

\begin{problem}
    Sea $k$ un entero positivo tal que
    \[
        1 + x^k + x^{2k} = (1 + a_1 x + x^2)(1 + a_2 x + x^2) \ldots (1 + a_k x + x^2),
    \]
    hallar el valor de $a_1^2 + a_2^2 + \ldots + a_k^2$
\end{problem}

\begin{problem}
    Dado los polinomios $P(x)$ y $Q(x)$, se sabe que estos tienen tres términos cada uno
    ¿cuántos monomios distintos de cero tiene como mínimo el producto $P(x) Q(x)$?
\end{problem}

\begin{problem}
    Sea $P(x) = 1 - \dfrac{x}{2} + \dfrac{x^2}{6}$, definimos
    \[
        Q(x) = P(x)P(x^3)P(x^5)P(x^7)P(x^9) = \sum_{i = 0}^{50} a_i x^i,
    \]
    encontrar $\ds \sum_{i = 0}^{50} |a_i|$.
\end{problem}

\begin{problem}
    Sean $P(x)$ y $Q(x)$ polinomios de segundo grado con coeficientes enteros.
    Probar que existe un polinomio $R(x)$ con coeficientes enteros y de grado a lo sumo dos tal que
    \[
        R(8)R(12)R(2017) = P(8)P(12)P(2017)Q(8)Q(12)Q(2017).
    \]
\end{problem}

\begin{problem}
    Suponga que $f(x)$ es un polinomio de grado 3 con coeficiente principal igual a 2 y
    \[
        f(2024) = 2025, \quad f(2025) = 2026,
    \]
    hallar el valor de $f(2026) - f(2023)$.
\end{problem}

\begin{problem}
    Sea $P(x)$ un polinomio mónico de grado cuatro tal que $P(1 + 2^n) = 1 + 8^n$ para todo $n = 1, 2, 3, 4$.
    Hallar el valor de $P(1)$.
\end{problem}

\begin{problem}
    Sea $f(x) = a_0 + a_1 x + \ldots + a_4 x^4$ con $a_4 \neq 0$.
    El resto del polinomio $f$ cuando es dividido por $(x - 2023)$, $(x - 2024)$, $(x - 2025)$, $(x - 2026)$ y $(x - 2027)$ son 24, -6, 4, -6 y 24, respectivamente.
    Hallar el valor de $f(2028)$.
\end{problem}

\begin{problem}
    Probar que para todo número real $a$ el polinomio
    \[
        x^4 + a^2 x^3 + 2ax^2 + 3a^2 x + a - 1
    \]
    tiene al menos una raíz real.
\end{problem}

\begin{problem}
    Probar que el polinomio
    \[
        P(x) = x^4 + ax^3 + bx^2 + c x - \frac{b}{2} - \frac{1}{4}
    \]
    tiene una raíz real para cualesquiera números reales $a,b,c$.
\end{problem}

\begin{problem}
    Sea $P(x)$ un polinomio arbitrario tal que
    \[
        P(2008) + P(17) < 2025 < P(18) + P(2007).
    \]
    Probar que existen números reales $x,y$ tales que
    \[
        x + y = P(x) + P(y) = 2025.
    \]
\end{problem}

\begin{problem}
    Sean $P(x)$ y $Q(x)$ polinomios mónicos con coeficientes reales y
    \[
        \deg(P) = \deg(Q) = 10.
    \]
    Probar que si la ecuación $P(x) = Q(x)$ no tiene soluciones reales, entonces la ecuación
    \[
        P(x + 1) = Q(x - 1)
    \]
    tiene una solución real.
\end{problem}