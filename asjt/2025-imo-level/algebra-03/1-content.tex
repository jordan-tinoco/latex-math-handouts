\section{Fundamentos}

\begin{definition}[Función]
    Sean $X$ y $Y$ dos conjuntos.
    Una función $f : X \to Y$ es una relación con un valor de $Y$ para cada $x \in X$, y lo denotamos por $f(x) = y$.
\end{definition}

\begin{definition}[Inyectiva]
    Una función $f : X \to Y$ es inyectiva si para todo $y \in Y$, existe a lo más un elemento $x \in X$ tal que $f(x) = y$.
\end{definition}

\begin{definition}[Sobreyectiva]
    Una función $f : X \to Y$ es sobreyectiva si para todo $y \in Y$, existe al menos un elemento $x \in X$ tal que $f(x) = y$.
\end{definition}

\begin{example}
    Encontrar todas las funciones $f: \R \to \R$ tal que
    \[
        f(xf(y) - f(x)) = 2f(x) + xy.
    \]
\end{example}
\begin{solution*}
    La única función que cumple es $f(x) = 1 - x$ para todo $x \in \R$.
    Sea $P(x,y)$ la evaluación en la función.
    Usando $P(1,y)$, obtenemos que
    \[
        f(f(y) - f(1)) = y + 2f(1),\quad \forall y \in \R,
    \]
    por la cual $f$ es biyectiva.
    Como $f$ es biyectiva, existe un $a \in \R$ tal que $f(a) = 0$.
    Usando $P(a,y)$, obtenemos
    \[
        f(af(y)) = ay, \quad \forall y \in \R.
    \]
    Con $y = 0$ en esta ecuación encontramos que $f(af(0)) = 0 = f(a)$ lo que implica que $a (f(0) - 1) = 0$.
    Si $a = 0$, usando $P(x,0)$ tenemos que $f(-f(x)) = 2f(x)$ donde la única solución es $f(x) = -2x$, pero esta función no cumple.
    Por tanto $f(0) = 1$.
    Usando $P(1,1)$, obtenemos que
    \begin{align*}
        f(f(1) - f(1)) &= 2f(1) + 1 \\
        f(0) &= 2f(1) + 1 \\
        0 &= f(1)\\
        f(a) &= f(1)
    \end{align*}
    por lo cual $a = 1$.
    Así, usando $P(1,y)$ obtenemos
    \[
        f(f(y)) = y, \quad \forall y \in \R.
    \]
    Usando $P(f(x),1)$, encontramos
    \[
        f(-x) = f(x) + 2x, \quad \forall x \in \R
    \]
    Usando este último resultado y $P(x, f(y))$, tenemos que
    \begin{align*}
        f(xy - f(x)) &= 2f(x) + xf(y)\\
        \implies f(f(x) - xy) + 2\left[f(x) - xy\right] &= 2f(x) + xf(y)\\
        \iff f(f(x) - xy) &= xf(y) + 2xy\\
        \iff f(f(x) - xy) &= x\left[f(y) + 2y\right]\\
        \implies f(f(x) - xy) &= xf(-y)
    \end{align*}
    Con $y = -1$ en este último resultado, $f(f(x) + x) = 0 = f(1)$ de donde obtenemos que $f(x) = 1 - x$ para toda $x \in \R$.
\end{solution*}