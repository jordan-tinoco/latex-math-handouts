\section{Problemas propuestos}

\begin{section-problem}
    Sea $ABCD$ un cuadrado y sea $X$ un punto en lado $BC$.
    Sea $Y$ un punto en la recta $CD$ tal que $BX = YD$ y $D$ se encuentra entre $C$ y $Y$.
    Demuestra que el punto medio de $XY$ se encuetra sobre la diagonal $BD$.
\end{section-problem}

\begin{section-problem}
    Los puntos $P$, $Q$ y $R$ están sobre los lados $AB$, $BC$ y $CA$ del triángulo acutángulo \theTriangle{ABC}, respectivamente.
    Si $\angle BAQ = \angle CAQ$, $QP \perp AB$, $QR \perp AC$ y $CP$ y $BR$ se intersecan en $S$ probar que $AS \perp BC$.
\end{section-problem}

\begin{section-problem}
    Sea el triángulos \theTriangle{ABC} tal que una circunferencia que pasa por $A$ y $B$ interseca los segmentos $AC$ y $BC$ en $D$ y $E$, respectivamente.
    Las rectas $AB$ y $DE$ se intersecan en $F$, mientras que las rectas $BD$ y $CF$ se intersecan en $M$.
    Probar que $MF = MC$ si y solo si $MB \cdot MD = MC^2$.
\end{section-problem}

\begin{section-problem}
    Los lados opuestos de un hexágono son paralelos.
    Probar que las rectas que pasan por los puntos medio de los lados concurren.
\end{section-problem}

\begin{section-problem}
    Sea $D$ el pie de altura desde $A$ en el triángulo \theTriangle{ABC} y $M$, $N$ puntos en los lados $CA$ y $AB$ talque las rectas $BM$ y $CN$ se intersecan en $AD$.
    Probar que $AD$ es bisectriz del ángulo $\angle MDN$.
\end{section-problem}