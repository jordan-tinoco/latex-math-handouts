\section{Problemas propuestos}

Recordar que los problemas de esta sección son los asignados como \textbf{tarea}.
Es el deber del estudiante resolverlos y entregarlos de manera clara y ordenada el próximo encuentro
(de ser necesario, también se pueden entregar borradores).

\begin{section-exercise}
    Realizar la demostración del teorema de \textbf{Menealao} tanto en su forma normal como trigonométrica.
\end{section-exercise}

\begin{section-problem}
    Sea $ABC$ un triángulo, y sean $A_1$, $B_1$ y $C_1$ los puntos de tangencia del incírculo con $BC$, $CA$ y $AB$, respectivamente.
    Sea $A_2$ el simétrico de $A_1$ con respecto a $B_1 C_1$, y se definen $B_2$ y $C_2$ de manera análoga.
    Sea $A_3$ la intersección de $A A_2$ con $BC$, $B_3$ la intersección de $B B_2$ con $AC$ y $C_3$ la intersección de $C C_2$ con $AB$.
    Demuestre que $A_3$, $B_3$ y $C_3$ con colineales.
\end{section-problem}