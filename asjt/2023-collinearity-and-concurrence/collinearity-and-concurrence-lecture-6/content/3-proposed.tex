\section{Problemas propuestos}

Recordar que los problemas de esta sección son los asignados como \textbf{tarea}.
Es el deber del estudiante resolverlos y entregarlos de manera clara y ordenada el próximo encuentro
(de ser necesario, también se pueden entregar borradores).

\begin{section-exercise}
    Probar que la circunferencia de los nueve puntos biseca cualquier segmento trazado del ortocentro al circuncírculo.
\end{section-exercise}

\begin{section-problem}
    Sea $O$ el centro de la circunferencia $\Omega_1$.
    Sea $l$ cualquier recta que no pasa por $O$.
    Una segunda circunferencia $\Omega_2$ es tangente a $\Omega_1$ en $P$ y también es tangente a $l$ en $Q$.
    La perpendicular por $O$ a $l$ corta a $\Omega_1$ en dos puntos, sea $R$ el que está más alejado de $l$.
    Demostrar que $P$, $Q$ y $R$ están alineados.
\end{section-problem}