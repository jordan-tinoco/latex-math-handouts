\section{Desarrollo}

\begin{figure}[htb]
    \centering
    \includegraphics[width=5cm]{images/alf}
    \caption{Una imagen de Alf}
    \label{fig:figure}
\end{figure}

\begin{section-theorem}[\textbf{Ejemplo de teorema}]
    Hola, este un ejemplo de como escribir un teorema matemático en la plantilla.
\end{section-theorem}

\begin{proof}
    Este es el ejemplo de como escribir la demostración de un teorema matemático en la plantilla.
    \[
        \sen^2(x) + \cos^2(x) = 1. \qedhere
    \]
\end{proof}

\begin{section-exercise}
    Es es un ejemplo de ejercicio.
\end{section-exercise}

\begin{solution}
    Este es un ejemplo de solución.
    \[
        x = x + 1.
    \]
    Solución del ejercicio o problema.
\end{solution}


\begin{problem}
    Nueve celdas de un tablero $10\times10$ están infectadas. Dos celdas son vecinas si tienen un lado en común. En cada minuto, las celdas que tengan al menos dos vecinos infectados se vuelven infectadas. ¿Puede llegar a suceder que todas las celdas del tablero estén infectadas?

    \begin{source-problem}
        Soluciones a los problemas de entrenamiento, problema 4.\\ Tzaloa, 2022
    \end{source-problem}
\end{problem}

\begin{solution}[1]
    Ejemplo de solución enumerada.
\end{solution}

\begin{problem}
    Determina el menor entero positivo $n$ con las siguientes propiedades:
    \begin{enumerate}
        \item Su dígito de las unidades es 6.
        \item Si el último dígito 6 se borra y se coloca al principio del número, el resultado es 4 veces $n$.
    \end{enumerate}
    \begin{source-problem}
        IMO 2040
    \end{source-problem}
\end{problem}

\begin{section-problem.tcb}
        Un agricultor cosechó en el primer día $(x - 2)^{2023}$ granos de maíz y el segundo día $(x - 1)^{2024} + 7$ granos de maíz.
        Si el agricultor almacena los granos de los dos días en sacos, los cuales tiene una capacidad de $x^2 - 3x + 2$ granos cada uno.
        ¿cuál es el polinomio que representa los granos sobrantes?
\end{section-problem.tcb}

Ejemplo de cómo escribir los conjuntos de números
\begin{align*}
    \N & \quad \text{signo que representa los números naturales}\ \left\{ 1, 2, \cdots \right\}\\
    \Q & \quad \text{signo que representa los números racionales}\\
    \Z & \quad \text{signo que representa los números enteros}\\
    \ZP &\quad \text{signo que representa los números enteros positivos}\\
    \ZN &\quad \text{signo que representa los números enteros negativos}\\
    \ZNN &\quad \text{signo que representa los números enteros no negativos}\\
    \R &\quad \text{signo que representa los números reales}\\
    \RP &\quad \text{signo que representa los números reales positivos}\\
    \RN &\quad \text{signo que representa los números reales negativos}\\
    \C &\quad \text{signo que representa los números complejos}
\end{align*}

Ejemplos de cómo utilizar módulos
\begin{gather*}
    x \equiv 1 \pmod{n}\\
    x \fullMod{1}{n}
\end{gather*}

Diferencia entre \emph{enfasís} y \textit{texto modo italico.}

\begin{center}
    \emph{hola} $\overset{\text{¿?}}{=}$ \textit{hola}
\end{center}

\begin{section-theorem.tcb}[Teorema de Ceva sobre la circunferencia]
    Sean $ABC$ y $DEF$ dos triángulos sobre la misma circunferencia.
    Entonces las rectas $AD$, $BE$ y $CF$ son concurrentes si y sólo si
    \[\frac{BD}{DC} \cdot \frac{CE}{EA} \cdot \frac{AF}{FB} = 1.\]
\end{section-theorem.tcb}

\begin{section-theorem.tcb}[Teorema del resto]
    Dado un polinomio $P$, de grado $n$ y $a \in \R$, diremos que el resto de $P$ cuando es dividido por $x - a$ es $P(a)$.
    Es decir
    \[P(a)   = r \Leftrightarrow P(x) = (x-a)Q(x) + r\]
    para algún polinomio $Q(x).$
\end{section-theorem.tcb}

Prueba de algunos comandos matemáticos.

\begin{gather*}
    \underbrace{\overbrace{a + \cdots + a}^{n\ \text{veces}} + 1 + \cdots + 1 + \overbrace{b + \cdots + b}^{m\ \text{veces}}}_{k + n + m\ \text{sumandos}}
\end{gather*}

\begin{gather*}
    f_n(x) =
    \begin{cases}
        -x^2 + n,   & \text{si $x < 0$ y $n$ par},\\
        \alpha + x, & \text{si}\ x > 0, \\
        x^2,        & \text{en otros casos.}
    \end{cases}
\end{gather*}

$p$\nobreakdash-ádico

\begin{gather*}
    \upla{x}{n} & \quad \text{ejemplo de una tupla}\\
    \kupla{x}{k - 10} &\quad \text{ejemplo de una $k$\nobreakdash-upla}\\
    \kupla[m]{r}{i} &\quad \text{ejemplo de una $m$\nobreakdash-upla}\\
    P(x) = \polinom {n}{a} &\quad \text{ejemplo de polinomio}\\
    Q(x) = \polinom[y]{m}{c} &  \quad \text{otro ejemplo de polinomio}\\
    \asum {}{k} &  \quad \text{otro ejemplo de suma}\\
    \gprod {}{k} &  \quad \text{otro ejemplo de producto}\\
    \frac{\asum {}{n}}{n} \ge \sqrt[n]{\gprod {}{n}} & \quad \text{Ejemplo AM-MG}
\end{gather*}



