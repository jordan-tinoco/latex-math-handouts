\section*{\large Problemas}

Estimado estudiante, resolver los siguientes problemas de manera clara y ordenada.
Recordar justificar la respuesta.


\begin{problem}
    Sea $Q(x) = 2x - 4096$ y $P(x) = Q^{12}(x)$, hallar la raíz de $P$.
\end{problem}

\begin{problem}
    Hallar $Q(x)$, si $P\left(Q(x) - 3\right) = 6x + 2$ y $P(x + 3) = 2x + 10$.
\end{problem}

\begin{problem}
    Sea el polinomio $P_0(x) = x^3 + 313x^2 - 77x - 8$.
    Para enteros $n \geq 0$, definimos $P_n(x) = P_{n - 1}(x - n)$.
    ¿Cuál es el coeficiente del término cuadrático en $P_{23}(x)$?
\end{problem}

\begin{problem}
    Demostrar por inducción matemática, que $\forall n \in \Z^{\geq 0}$, se cumple
    \[17 \mid 2^{5n + 3} + 5^n \cdot 3^{n + 2}.\]
\end{problem}

\begin{problem}
    Sean $a$, $b$ y $c$ números reales distintos de cero, con $a + b + c \neq 0$.
    Probar que si
    \[\frac{1}{a} + \frac{1}{b} + \frac{1}{c} = \frac{1}{a + b + c},\]
    entonces para $n$ impar se cumple
    \[\frac{1}{a^n} + \frac{1}{b^n} + \frac{1}{c^n} = \frac{1}{a^n + b^n + c^n}.\]
\end{problem}


