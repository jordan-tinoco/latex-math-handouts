\newpage
\section*{\large Soluciones}

    \textbf{Problema 1.}

Nos piden hallar la raíz de $P(x) = Q^{12}(x)$, lo cual es igual a $P(x) = \underbrace{Q(Q(...Q(}_{12\ veces}x)\dots))$.
Teniendo esto en cuenta, al encontrar $Q^{2}(x)$, $Q^{3}(x)$, $Q^{4}(x)$, $\cdots$ veremos un patrón el cual nos ayuda a deducir la forma de $P(x)$
\begin{gather*}
    Q^{2}(x) = Q(Q(x)) = 2 Q(x) - 4096 = 2(2x - 2^{12}) - 2^{12} = \boxed{2^2 x - 2^{12} (1 + 2)} \\
    Q^{3}(x) = Q(Q^2(x)) = 2 Q^2(x) - 4096 = 2(2^2 x - 2^{12} (1 + 2)) - 2^{12} = \boxed{2^3 x - 2^{12} (1 + 2 + 2^2)}\\
    Q^{4}(x) = Q(Q^3(x)) = 2 Q^3(x) - 4096 = 2\left(2^3 x - 2^{12} (1 + 2 + 3)\right) - 2^{12} = \boxed{2^4 x - 2^{12} (1 + 2 + 2^2 + 2^3)}\\
    \vdots
\end{gather*}
Es decir
\[\boxed{Q^{k}(x) = 2^k x - 2^{12} (1 + 2 + 2^2 + \cdots + 2^{k-1})}\]

Lo cual, por propiedades que ya hemos visto en clase, podemos reducir el lado derecho la expresión como se sigue
\[Q^{k}(x) = 2^k x - 2^{12} \left(\frac{2^k - 1}{2 - 1}\right) = \boxed{2^k x - 2^{12}(2^k - 1)}\]

Por lo tanto, podemos decir que la forma de $P(x)$ es la siguiente
\begin{gather*}
    P(x) = Q^{12}(x) = 2^{12} x - 2^{12}\left(2^{12} - 1\right)\\
    \longrightarrow P(x) = 2^{12} \left(x - (2^{12} - 1)\right)
\end{gather*}

De donde es fácil ver que $\boxed{x = 2^{12} - 1}$ es raíz de $P$.



\textbf{Problema 2.}

Primero encontremos $P(x)$, lo cual lo podemos lograr sustituyendo $x - 3$ por $x$ en la segunda condición del problema, es decir
\begin{gather*}
    P\left((x - 3) + 3\right) = 2(x - 3) + 10\\
    P(x) = 2x - 6 + 10\\
    P(x) = 2x + 4
\end{gather*}

Con este resultado, podemos ver que $P(Q(x) - 3) = 2(Q(x) - 3) + 4$, lo cual nos permite decir que $2(Q(x) - 3) + 4 =  6x + 2$.
De donde rápidamente llegamos a que $\boxed{Q(x) = 3x + 2}$.

\textbf{Problema 3.}

Siguiendo la idea del problema 1, vamos a encontrar los primeros $P_i(x)$ para lograr indentificar un patrón que nos ayude a simplificar los cálculos:
\begin{gather*}
    P_1(x) = P_0(x - 1) = (x - 1)^3 + 313 (x - 1)^2 - 77(x - 1) - 8\\
    \rightarrow \boxed{P_1(x) = (x - 1)^3 + 313 (x - 1)^2 - 77(x - 1) - 8}\\
    P_2(x) = P_1(x - 2) = [(x - 2) - 1]^3 + 313 [(x - 2) - 1]^2 - 77[(x - 2) - 1] - 8\\
    \rightarrow \boxed{P_2(x) = [x - (1 + 2)]^3 + 313 [x - (1 + 2)]^2 - 77[x - (1 + 2)] - 8 }\\
    P_3(x) = P_2(x - 3) = [(x - 3) - (1 + 2)]^3 + 313 [(x - 3) - (1 + 2)]^2 - 77[(x - 3) - (1 + 2)] - 8 \\
    \rightarrow \boxed{P_3(x) = [x - (1 + 2 + 3)]^3 + 313 [x - (1 + 2 + 3)]^2 - 77[x - (1 + 2 + 3)] - 8 }\\
    \vdots
\end{gather*}

Es decir
\[P_k(x) = [x - (1 + 2 + \cdots + k)]^3 + 313 [x - (1 + 2 + \cdots + k)]^2 - 77[x - (1 + 2 + \cdots + k)] - 8 \]
Lo cual por la sumas de Gauss podemos simplificar y obtener
\[P_k(x) = \left[x - \frac{k(k + 1)}{2}\right]^3 + 313 \left[x - \frac{k(k + 1)}{2}\right]^2 - 77\left[x - \frac{k(k + 1)}{2}\right] - 8 \]

Por lo tanto $P_{20}(x)$ es igual a
\begin{gather*}
    P_{20}(x) = \left[x - \frac{20\cdot21}{2}\right]^3 + 313 \left[x - \frac{20\cdot21}{2}\right]^2 - 77\left[x - \frac{20\cdot21}{2}\right] - 8\\
    \rightarrow \boxed{P_{20}(x) = (x - 210)^3 + 313 (x - 210)^2 - 77(x - 210) - 8}
\end{gather*}

En la expansión de $(x - 210)^3$ el único término cuadrático es $3\cdot(-210)\cdot x^2$ y en la expansión de $313 (x - 210)^2$  es $313x^2$.
Por lo tanto el valor que buscamos es $3\cdot (-210) + 313 = -630 + 313 = \boxed{-317}.$


\textbf{Problema 4.}

\textbf{Caso base.} Veamos que pasa para $n = 0$
\begin{gather*}
    17 \mid 2^{5(0) + 3} + 5^{0} \cdot 3^{0 + 2}\\
    17 \mid 2^{3} + 1\cdot 3^{2} \\
    17 \mid 8 + 9\\
    17 \mid 17
\end{gather*}
claramente 17 divide a 17.

\textbf{Hipótesis de inducción.} Supongamos, entonces, que para un entero fijo $k \geq 0$, se cumple que
\[17 \mid 2^{5k + 3} + 5^{k} \cdot 3^{k + 2}\]

\textbf{Paso inductivo.} Vamos a demostrar que como para $k$ se cumple la divisibilidad entonces para $k + 1$ también se va a cumplir.
Es decir
\begin{gather*}
    17 \mid 2^{5(k + 1) + 3} + 5^{k + 1} \cdot 3^{(k + 1) + 2}\\
    17 \mid 2^{5k + 8} + 5^{k + 1} \cdot 3^{k + 3}
\end{gather*}

Por propiedades de potencia podemos hacer lo siguiente
\begin{gather*}
    17 \mid 2^{5k + 8} + 5^{k + 1} \cdot 3^{k + 3}\\
    17 \mid 2^{5k + 3} \cdot 2^5 + (5^{k} \cdot 5^1) \cdot (3^{k + 2} \cdot 3^1)\\
    17 \mid 32\cdot 2^{5k + 3} + 15\cdot 5^{k} \cdot 3^{k + 2}
\end{gather*}

Sumando y restando $17\cdot 5^{k} \cdot 3^{k + 2}$ en el lado derecho, vemos que
\begin{gather*}
    17 \mid 32\cdot 2^{5k + 3} + 15\cdot 5^{k} \cdot 3^{k + 2} \\
    17 \mid 32\cdot 2^{5k + 3} + 15\cdot 5^{k} \cdot 3^{k + 2} + 17\cdot 5^{k} \cdot 3^{k + 2} - 17\cdot 5^{k} \cdot 3^{k + 2}\\
    17 \mid 32\cdot 2^{5k + 3} + 32\cdot 5^{k} \cdot 3^{k + 2} - 17\cdot 5^{k} \cdot 3^{k + 2}\\
    \boxed{17 \mid 32\cdot (2^{5k + 3} + 5^{k} \cdot 3^{k + 2}) - 17\cdot (5^{k} \cdot 3^{k + 2})}
\end{gather*}

Lo cual por la hipótesis de inducción es cierto.

\textbf{Problema 5.}

Al trabajar la condición del problema vemos que
\begin{gather*}
    \frac{1}{a} + \frac{1}{b} + \frac{1}{c} = \frac{1}{a + b + c}\rightarrow
    \frac{ab + bc + ca}{abc} = \frac{1}{a + b + c}\\
    (a + b + c)(ab + bc + ca) = abc\\
    (a + b + c)(ab + bc + ca) - abc = 0
\end{gather*}
Lo cual por propiedad es
\[(a + b)(b + c)(c + a) = 0\]
Como es una ecuación que tiene el producto de 3 números igual a cero, entonces no queda más opción que al menos uno de los números sea cero.
Si $a + b = 0$, entonces $a = -b$.
Para un $n$ impar vemos que $a^n = (-b)^n = - b^n$, por lo tanto la expresión
\begin{gather*}
    \frac{1}{a^n} + \frac{1}{b^n} + \frac{1}{c^n} = \frac{1}{a^n + b^n + c^n}\rightarrow
    \frac{1}{-b^n} + \frac{1}{b^n} + \frac{1}{c^n} = \frac{1}{-b^n + b^n + c^n}\\
    \boxed{\frac{1}{c^n} = \frac{1}{c^n}}
\end{gather*}
Hecho que no pasa con $n$ par.
Si hacemos este análisis para $b + c = 0$ y $c + a = 0$ el resultado es el mismo, luego el problema está hecho.