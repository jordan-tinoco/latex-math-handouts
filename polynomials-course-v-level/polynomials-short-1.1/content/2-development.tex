\section*{\large Problemas}

Estimado estudiante, resolver los siguientes ejercicios de manera clara y ordenada.

\begin{exercise}
    Dado el polinomio $S(x) = (11 - 15x^3)(17x^6 - 43) + 2^8 x^6 (1 - x + x^2)(1 + x)$, responda lo siguiente:
    \begin{multicols}{2}
        \begin{enumerate}
            \item ¿$S(x)$ es mónico? \\R: \rule{1cm}{0.1mm}
            \item ¿$S(x)$ es completo? \\R: \rule{1cm}{0.1mm}
            \item ¿$S(x)$ es simétrico? \\R: \rule{1cm}{0.1mm}
            \item Escriba el coeficiente de $x^6$. \\R: \rule{1cm}{0.1mm}
            \item Escriba el término independiente.\\ R: \rule{1cm}{0.1mm}
            \item ¿Es $S(\sqrt[3]{x})$ un polimonio?\footnote{Justificar la respuesta.} \\ R: \rule{2cm}{0.1mm}
        \end{enumerate}
    \end{multicols}
\end{exercise}

\begin{exercise}
    Si tenemos que
    \begin{align*}
        P(x) = \frac{x - 1}{3} \\
        Q(x) = 3x^2 - 2x \\
        R(x) = (Q \circ P)(x) - (P \circ Q)(x)
    \end{align*}
    ¿Cuál es el valor de $R(1)$?\footnote{Escribir el procedimiento.}
    \begin{multicols}{5}
        \begin{enumerate}
            \item -1
            \item 1
            \item -36
            \item 0
            \item 10
        \end{enumerate}
    \end{multicols}
\end{exercise}

\newpage

\section*{\large Soluciones}
{
    \textbf{Ejercicio 1.}
    \begin{gather*}
        S(x) = (11 - 15x^3)(17x^6 - 43) + 2^8 x^6 (1 - x + x^2)(1 + x)\\
        S(x) = (187 x^6 - 473 - 255x^9 + 645x^3) + 256 x^6 (1 + x^3)\\
        S(x) = (187 x^6 - 473 - 255x^9 + 645x^3) + (256 x^6 + 256 x^9)\\
        S(x) = (256 x^9 - 255x^9) + (256 x^6 + 187 x^6) + 645x^3 - 473\\
        S(x) = x^9 + 543 x^6 + 645x^3 - 473
    \end{gather*}

        \begin{enumerate}
            \item Sí, ya que el su coeficiente principal es $1$.
            \item No, ya que faltan los términos de $x^8$, $x^7$, $x^5$, $x^4$, $x^2$ y $x$.
            \item No, ya que con sólo ver que el coeficiente principal y el término independiente no son iguales el polinomio no es simétrico.
            \item El coeficiente es 645.
            \item El término independiente es $-473$.
            \item Sí, ya que al evaluar el polinomio obtenemos $S(\sqrt[3]{x}) = x^3 + 543 x^2 + 645x - 473$.
        \end{enumerate}

    \textbf{Ejercicio 2.} Primero encontremos $(Q \circ P)(x)$
    \[ (Q \circ P)(x) = Q(P(x)) = 3\left( \frac{x - 1}{3} \right)^2 - 2\left( \frac{x - 1}{3} \right)\]
    \[ (Q \circ P)(x) = 3\left( \frac{x^2 - 2x + 1}{9} \right) - \frac{2x - 2}{3}\]
    \[ (Q \circ P)(x) = \frac{x^2 - 2x + 1}{3} - \frac{2x - 2}{3} = \frac{x^2 - 2x + 1 - (2x - 2)}{3}\]
    \[ (Q \circ P)(x) = \frac{x^2 - 4x + 3}{3}\]
    Seguidamente, $(P \circ Q)(x)$
    \begin{gather*}
        (P \circ Q)(x) = P(Q(x)) = \frac{3x^2 - 2x - 1}{3}
    \end{gather*}
    Sustituimos $(Q \circ P)(x)$ y $(P \circ Q)(x)$ en $R(x)$ y simplificamos
    \begin{gather*}
        R(x) = (Q \circ P)(x) - (P \circ Q)(x) = \frac{x^2 - 4x + 3}{3} -  \frac{3x^2 - 2x - 1}{3}\\
        R(x) = \frac{x^2 - 4x + 3 - (3x^2 - 2x - 1)}{3} = \frac{-2 x^2 - 2x + 4}{3}
    \end{gather*}
    Finalmente, evaluamos $R(1)$
    \[R(1) = \frac{-2 (1)^2 - 2(1) + 4}{3} = \frac{-2 - 2 + 4}{3} = 0\]
    Respuesta correcta es la opción 'd'.
}\label{sec:soluciones}
