{\Large\textbf{Contenido:} Consejos para estudiar matemáticas}

En este pequeño documento presentamos consejos (o heurísticas) básicas para estudiar matemáticas.
Principalmente está orientado a estudiantes que están empezando con temas un tanto más complejos a los vistos en sus centros de estudios.

En general, estudiar matemáticas es un proceso lento y requiere \textbf{mucho tiempo}, lo cual hace necesario ser rigurosos y metódicos en el estudio para lograr fluidez en esta ciencia.
Particularmente, cuando estudiamos temas de álgebra \textbf{practicar mucho} es extremadamente fundamental, y el presente curso hará énfasis en esta cuestión.

A continuación, presentamos algunos puntos que puede mejorar el proceso de estudiar con documentos de matemáticas.

\begin{enumerate}
    \item \textbf{Explorar el documento:}
    Tratá de hacerte una idea general de lo que vas a estudiar y trabajar; títulos, subtítulos, objetivos que se pretenden.
    Es hacer una aproximación de la idea general y composición de lo que se presenta.
    \item \textbf{Preguntar:}
    Es el momento de anotar preguntas que se van apareciendo a través de un trabajo dinámico, reflexivo y crítico.
    \item \textbf{Leer:}
    Realizar una lectura compresiva y no memorística.
    Consejos:
    \begin{enumerate}
        \item Hay que leer activamente.
        Hay que \textbf{subrayar} (usando rayas, puntos, asteriscos o con lo que se te ocurra), resaltar los conceptos, términos, símbolos y procedimientos más importantes o los que no entendás.
        Es decir, personalizar el texto, para hacerlo tuyo y que no sea solo un montón de letras y símbolos.
        \item Hay que saber resumir lo leído, captar los fundamental, expresarlo con tus propias palabras.
    \end{enumerate}
    \item \textbf{Intentar todo:} Esto puede sonar contradictorio con lo anterior, pero nunca leás una solución (en la medida de lo posible) sin antes haber intentado \textbf{resolverlo por tu cuenta}.
    Siempre intentá todo lo que podás intentar; ejemplos, demostraciones, ejercicios y problemas.
    \begin{center}
        "\textit{Toda situación novedosa, como lo es un problema, es una oportunidad única para desafiar nuestro conocimientos y aprender algo nuevo, luego de saber como resolverlo este pierde todo su valor y pasa a ser un mero ejercicio repetitivo.}"
    \end{center}
    \item \textbf{Trabajar en equipo:} Siempre que podás, trata de trabajar en equipo, busca compañeros y dialoga los términos, símbolos, ejercicios, ejemplos y problemas que te inquietan.
    En la mayoría de lo caso, te dará un visión distinta a la que pensabás y te ayudará a comparar tus propias conclusiones.
    \item \textbf{Búscar recursos:} Cuando te topés con algo que te cueste comprender, usa lo medios que tengás disponibles para encontrar información sobre esa cuestión.
    Búsca libros, artículos, foros o vídeos que expliquen los que querás saber, también podés \textbf{pedir información adicional} a tus profesores, en la mayoría de los casos estos te facilitaran materiales.
\end{enumerate}

También, podemos mencionar algunos consejos para la resolución de problemas de matemáticas.
Estos puntos son casi comunes a todo tipo de problemas, sin embargo para que den resultados estos deben ser acompañados por una gran cantidad de practica.

\begin{enumerate}
    \item \textbf{Hacer muchos ejercicios y problemas por tu cuenta:} La mejor manera de aprender a resolver problemas es, precisamente, \textbf{resolver problemas}.
    Cada que tengás oportunidad, buscá un problema y ponte a resolverlo.

    Cada folleto del curso, tendrá una buena cantidad de ejercicios y problemas con los cuales podrás practicar.
    Pero si carecés de problemas desafiantes, buscá con tus profesores, ellos seguro te darán problemas interesantes.
    ¡Lo importante es que hagás \textbf{mucha práctica!}.
    \item \textbf{Resolver el problema, luego redactarlo:} La actividad de resolver un problema está compuesto de dos partes fundamentales; la \textbf{resolución} y la \textbf{redacción}.
    La resolución es el punto donde se ponen en juego todos los conocimientos, habilidades, ingenio, cultura, paciencia e incluso suerte de un individuo al momento de enfrentar un problema.

    La redacción, por otra parte, es resultado final de todo este largo proceso y es claro que este pierde sentido cuando en el proceso previo no existe un buen desarrollo.
    \item \textbf{Escribir todo lo que hacés:} Durante el proceso desafiante de resolver un problema no es fácil llegar a escribir una solución definitiva, clara y concisa (y este no debería de ser el objetivo).
    Por lo que es necesario \textbf{hacer borradores}, los cuales ayudaran a documentar todo lo que realizamos al durante el proceso, además que servirán de insumos para sintetizar la redacción final de nuestra solución, la que posiblemente entreguemos.
    \item \textbf{Practicar:} Nuevamente, hacemos hincapié en este punto, puesto que sin práctica difícilmente se logre adquirir la habilidad para resolver problemas \textbf{!practicá mucho!}
\end{enumerate}


Finalmente, presentamos una lista de recursos disponibles en la web.
Estos son de libre acceso y podés consultarlos en cualquier momento.
Cómo bibliografía general, tenemos los primeros tres libros, los cuales en conjunto exponen la teoría que abordarémos durante el curso.\\
\vspace{2mm}\\
\textbf{\large Recursos}

\begin{itemize}
    \item Libro. Álgebra para Olimpiadas Matemáticas. José Herber Nieto. AVCM, 2015. \href{https://acmfiles.s3.amazonaws.com/Libros/AlgebraParaOlimpiadas.pdf}{Enlace al pdf.}

    Este libro contiene la teoría básica de álgebra, también exponen más a detalle técnicas para la resolución de problemas de matemáticas.
    \item Libro. Problemas de Álgebra y cómo resolverlos. Armando Tori y Juan Ramos. RASCO Editores, 1998.

    Este libro está más orientado a la práctica, contiene una gran cantidad ejercicios, ejemplos y problemas propuestos muy interesantes.
    \item Libro. Algebra. Radmila Bulajich, José Gómez and Rogelio Valdez. Springer, 2015 (en inglés).

    Este libro podemos verlo como un punto intermedio entre dos anteriores, además de que expone una gran variedad de problemas de olimpiadas matemáticas.
    \item Enlace oficial de los documentos de José H. Nieto: \href{https://www.jhnieto.org/libros.htm}{jhnieto.org/libros}.

    En este sitio web, encontrás un catalogo de los documentos libres de este autor.
    \item Canal. Olimpiada Costarricense de Matemáticas: \href{https://www.youtube.com/@Olcomacr}{youtube.com/@Olcomacr}
    \item Canal. Math Gold Medallist (en inglés): \href{https://www.youtube.com/@mathgoldmedalist}{youtube.com/@mathgoldmedalist}
    \item Canal. Polos Olimpicos (en portugués): \href{https://www.youtube.com/@PolosOlimpicos}{youtube.com/@PolosOlimpicos}
    \item Canal. J\_tinoco\footnote{Durante el desarrollo del curso se subirán nuevo vídeos sobre polinomios.}: \href{https://www.youtube.com/channel/UCXFWu2nZqMJ5hEJS37Uf3cA}{youtube.com/@j\_tinoco}
\end{itemize}
