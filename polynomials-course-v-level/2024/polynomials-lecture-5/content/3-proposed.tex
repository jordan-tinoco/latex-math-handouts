\section{Problemas propuestos}

Los problemas de esta sección es la \textbf{tarea}.
El estudiante tiene el deber de entregar sus soluciones en la siguiente sesión de clase (también se pueden entregar borradores).
Recordar realizar un trabajo claro, ordenado y limpio.

\showLine
\begin{multicols}{2}
    \begin{problem}
        Dado los polinomios $P(x) = 2x^5 + x^4 + ax^2 + bx + c$ y $Q(x) = x^4 - 1$, se sabe que $Q \mid P$.
        Hallar $\dfrac{a + b}{a - b}$.
    \end{problem}

    \begin{problem}
        Dado los polinomios $P(x) = 16x^5 + ax^2 + bx + c$ y $Q(x) = 2 x^3 - x^2 + 1$, se sabe que $Q \mid P$.
        Hallar $a + b + c$.
    \end{problem}

    \begin{problem}
        Dado los polinomios $P(x) = 6x^4 + 4x^3 - 5x^2 - 10x + a$ y $Q(x) = 3 x^2 + 2x + b$, se sabe que $Q \mid P$.
        Hallar $a^2 + b^2$.
    \end{problem}

    \begin{problem}
        El polinomio $P(x)$ deja residuo $-2$ en la división entre $x - 1$ y residuo $-4$ en la división entre $x + 2$.
        Encontrar el residuo cuando el polinomio es dividido por $x^2 + x - 2$.
    \end{problem}

    \begin{problem}
        Encontrar el resto cuando el polinomio $x^{81} + x^{49} + x^{25} + x^9 + x$ es dividido entre $x^3 - x$.
    \end{problem}

    \begin{problem}
        Probar que si un polinomio $F(x)$ deja un residuo de la forma $px + q$ cuando es dividido entre $(x - a)(x - b)(x - c)$
        donde $a$, $b$ y $c$ son todos distintos, entonces
        \[
            (b - c)F(a) + (c - a)F(b) + (a - b)F(c) = 0.
        \]
    \end{problem}
\end{multicols}