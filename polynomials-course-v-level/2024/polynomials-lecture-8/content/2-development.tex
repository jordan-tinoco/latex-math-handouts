\section{Desarrollo}

\subsection{Polinomios simétricos elementales}

Diremos que un polinomio de dos variables $P(x,y)$ es simétrico si cumple $P(x, y) = P(y, x)$, \eg\, $P(x,y) = 7x^2 - 5xy + 7y^2$ es simétrico, ya que este es igual a $P(y, x) = 7y^2 - 5yx + 7x^2$.
Análogamente, un polinomio de tres variables $P(x, y, z)$ es simétrico si cumple $P(x, y, z) = P(x, z, y) = P(y, x, z) = P(y, z, x) = P(z, x, y) = P(z, y, x)$.

Por ejemplo, el polinomio $P(x, y, z) = 13x^5 + 13y^5 + 13z^5 - 8xyz + 1$ es simétrico, ya que
\begin{align*}
    P(x, y, z) = 13x^5 + 13y^5 + 13z^5 - 8xyz + 1 &&
    P(x, z, y) = 13x^5 + 13z^5 + 13y^5 - 8xzy + 1\\
    P(y, x, z) = 13y^5 + 13x^5 + 13z^5 - 8yxz + 1 &&
    P(y, z, x) = 13y^5 + 13z^5 + 13x^5 - 8yzx + 1\\
    P(z, x, y) = 13z^5 + 13x^5 + 13y^5 - 8zxy + 1 &&
    P(z, y, x) = 13z^5 + 13y^5 + 13x^5 - 8zyx + 1
\end{align*}
son todos el mismo polinomio.
De manera general, un polinomio $P(x_1, x_2, \dots, x_n)$ de $n$ variables es simétrico si para cualquier permutación de las $n$ variables se obtiene el mismo resultado, \ie
\[
    P(x_1, x_2, \cdots, x_n) = P(x_1, x_3, \cdots, x_n) = \cdots = P(x'_1, x'_2, \cdots, x'_n),
\]
donde $x'_1, x'_2, \cdots, x'_n$ representa una permutación de las variables $x_1, x_2, \cdots, x_n$.
Los polinomios simétricos más simples son aquellos donde todas sus variables tienen grado 1, como por ejemplo $x + y$, $xy$, $x + y + z$, etc.
Estos polinomio son llamados \emphtext{polinomios simétricos elementales} y podemos formalizarlo como se sigue.

\begin{definition}[Polinomio simétrico elemental]
    Sea $n > 0$ un entero y un entero $k$ con $1 \leq k \leq n$.
    Diremos que el $k$-ésimo polinomio simétrico elemental en las variables $x_1, \cdots, x_n$ es el polinomio
    \[
        \sigma_k = \sigma_k(x_1, x_2, \cdots, x_n) = \sum x_{i_1} x_{i_2} \cdots x_{i_k},
    \]
    cuya la suma se realiza sobre los subconjuntos $\left\{ i_1, i_2, \cdots, i_k \right\}$ de $k$ elementos del conjunto $\left\{ 1, 2, \cdots, n\right\}$.
\end{definition}

Donde el polinomio $\sigma_k$ lo leeremos como \emphtext{sigma k} o \emphtext{sigma sub k}.
Ahora, para aclarar esta definición, veamos algunos ejemplos.
\begin{table}[H]
    \centering
    \begin{tabular}{|c|l|}
        \hline
        Variables & Polinomios simétricos elementales \\
        \hline \hline
        $a$, $b$
        &
        $
        \begin{array}{lcl}
            \sigma_1 &=& a + b\\
            \sigma_2 &=& ab
        \end{array}
        $
        \\\hline
        $a$, $b$, $c$
        &
        $
        \begin{array}{lcl}
            \sigma_1 &=& a + b + c\\
            \sigma_2 &=& ab + bc + ca\\
            \sigma_3 &=& abc
        \end{array}
        $
        \\\hline
        $a$, $b$, $c$, $d$
        &
        $
        \begin{array}{lcl}
            \sigma_1 &=& a + b + c + d\\
            \sigma_2 &=& ab + ac + ad + bc + bd + da\\
            \sigma_3 &=& abc + bcd + cda\\
            \sigma_4 &=& abcd
        \end{array}
        $
        \\\hline
        $a$, $b$, $c$, $d$, $e$
        &
        $
        \begin{array}{lcl}
            \sigma_1 &=& a + b + c + d + e\\
            \sigma_2 &=& ab + ac + ad + ae + bc + bd + be + cb + ce + de\\
            \sigma_3 &=& abc + abd + abe + acd + ace + ade + bcd + bce + bde + ced\\
            \sigma_4 &=& abcd + abce + abde + acde + bced\\
            \sigma_5 &=& abcde
        \end{array}
        $
        \\\hline
    \end{tabular}
    \caption{Ejemplos de polinomios simétricos elementales.}
\end{table}

Otra notación útil para polinomios simétricos, es la siguiente.

\begin{definition}[Suma simétrica de potencias]
    La suma simétrica de la $k$-ésimas potencias en las variables $x_1, \cdots, x_n$ es el polinomio $s_k$ definido por
    \[s_k = s_k(x_1, x_2, \cdots, x_n) = x_1^k + x_2^k + \cdots + x_n^k, \quad \forall k \in \Z^{\geq 0}.\]
\end{definition}

Existen relaciones entre las sumas simétricas de potencias y los polinomios simétricos elementales.
Por ejemplo $s_2 = \sigma_1^2 - 2\sigma_2$ para cualquier cantidad de variables, $s_k = \sigma_1 s_{k - 1} - \sigma_2 s_{k - 2}$ para dos variables y $s_k = \sigma_1 s_{k - 1} - \sigma_2 s_{k - 2} + \sigma_3 s_{k - 3}$ para tres variables.
Estas relaciones son casos particulares de la llamadas \textbf{Fórmulas de Newton}.
Se invita al lector investigar más sobre estas fórmulas.

\begin{example}
    Hallar el valor de $(x^2 + y^2)$ sabiendo que $x + y = 1\ \land\ x^4 + y^4 = 7$.
\end{example}
\begin{solution}[1]
    Elevando al cuadrado la primera ecuación $x^2 + 2xy + y^2 = 1 \implies x^2 + y^2 = 1 - 2xy$.
    Por tanto, el ejercicio se reduce a encontrar el valor de $xy$.
    Veamos que
    \begin{align*}
    (x^2 + y^2)^2 &= (1 - 2xy)^2 && 2x^2 y^2 - 4xy - 6 = 0\\
    x^4 + 2x^2 y^2 + y^4 &= 1 - 4xy + 4x^2 y^2 && 2(x^2 y^2 - 2xy - 3) = 0\\
    x^4 + y^4 &= 2x^2 y^2 - 4xy + 1 && 2(xy - 3)(xy + 1) = 0\\
    7 &= 2x^2 y^2 - 4xy + 1 && \implies xy = 3\ \lor \ xy = -1
    \end{align*}
    Sabiendo los posibles valores de $xy$, vemos que $(x^2 + y^2)$ puede tomar los valores de $-5$ y 3.
\end{solution}

\begin{solution}[2]
    Claramente $\sigma_1 = s_1 = 1$ y $s_4 = 7$.
    De la relación $s_2 = \sigma_1^2 - 2\sigma_2 \implies s_2 = 1 - 2\sigma_2 \implies \sigma_2 = \frac{1 - s_2}{2}$.
    De la relación $s_k = \sigma_1 s_{k - 1} - \sigma_2 s_{k - 2}$ obtenemos
    \begin{align*}
        s_3 &= \sigma_1 s_2 - \sigma_2 s_1 && s_4 = \sigma_1 s_3 - \sigma_2 s_2\\
        s_3 &= s_2 - \sigma_2 && s_4 = s_3 - \sigma_2 s_2
    \end{align*}
    Operando $s_4 = s_2 - \sigma_2 - \sigma_2 s_2 = s_2 (1 - \sigma_2) - \sigma_2 = s_2\left(\dfrac{s_2 + 1}{2}\right) + \left(\dfrac{s_2 - 1}{2}\right) \implies 2s_4 = s_2^2 + 2s_2 - 1 \implies s_2^2 + 2s_2 - 15 = 0$, lo cual puede factorizarse como $(s_2 + 5)(s_2 - 3)$.
    Luego, los valores son $-5$ y 3.
\end{solution}



\subsection{Fórmulas de Vieta}

Ya conociendo los polinomios simétricos elementales, podemos volver a ver la definición de las fórmulas de Vieta que ya conocemos.
Sea $P(x) = a_n x^n + a_{n - 1} x^{n - 1} + \cdots  + a_1 x + a_0$ un polinomio con raíces $r_1, r_2, \cdots, r_n$,
entonces
\begin{gather*}
    \sigma_k = (-1)^k\times \frac{a_{n - k}}{a_n}.
\end{gather*}

\begin{example}
    Determine el producto de las raíces de $50x^{50} + 49x^{49} + \cdots + x + 1.$
\end{example}
\begin{solution}
    Por las fórmulas de Vieta, tenemos que $\sigma_{50} = (-1)^{50} \cdot \dfrac{a_0}{a_{50}} = \dfrac{1}{50}.$
\end{solution}



\subsection{Ejercicios y problemas}

Ejercicios y problemas para el autoestudio.

\showLine
\begin{multicols}{2}
    \begin{problem}
        Consideremos el polinomio $P(x) = x^n - (x - 1)^n$, donde $n$ es un entero positivo impar.
        Encontrar el valor de la suma y el valor del producto de sus raíces.
    \end{problem}

    \begin{problem}
        Encontrar los números reales $x$ y $y$, que satisfacen
        \[
            \left\{
            \begin{array}{rcl}
                x^3 + y^3 & =& 7\\
                x^2 + y^2 + x + y + xy & =& 4
            \end{array}
            \right.
        \]
    \end{problem}

    \begin{problem}
        Encontrar todas las soluciones reales del sistema
        \[
            \left\{
            \begin{array}{rcl}
                x + y + z & =& 1\\
                x^3 + y^3 + z^3 + xyz & =& x^4 + y^4 + z^4 + 1
            \end{array}
            \right.
        \]
    \end{problem}

    \begin{problem}
        Sean $x$, $y$ y $z$ números reales, encontrar todas las soluciones del siguiente sistema de ecuaciones
        \[
            \left\{
            \begin{array}{rcl}
                x + y + z & =& 6\\
                x^2 + y^2 + z^2 & =& 30\\
                x^3 + y^3 + z^3 & =& 132
            \end{array}
            \right.
        \]
    \end{problem}

    \begin{problem}
        Si $\alpha + \beta + \gamma = 0$, demostrar que
        \[3 (\alpha^2 + \beta^2 + \gamma^2) (\alpha^5 + \beta^5 + \gamma^5) = 5 (\alpha^3 + \beta^3 + \gamma^3) (\alpha^4 + \beta^4 + \gamma^4).\]
    \end{problem}

    \begin{problem}
        Sean $a$, $b$ y $c$ números reales distintos de cero, con $a + b + c \neq 0$.
        Probar que si
        \[\frac{1}{a} + \frac{1}{b} + \frac{1}{c} = \frac{1}{a + b + c},\]
        entonces para $n$ impar se cumple
        \[\frac{1}{a^n} + \frac{1}{b^n} + \frac{1}{c^n} = \frac{1}{a^n + b^n + c^n}.\]
    \end{problem}
\end{multicols}