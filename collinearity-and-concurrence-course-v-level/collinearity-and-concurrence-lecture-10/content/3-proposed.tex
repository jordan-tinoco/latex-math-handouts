\section{Problemas propuestos}

Los diez problemas restantes para el de trabajo de concurrencia y colinealidad.

\begin{section-problem}
    Sea $P$ un punto en el plano del triángulo \theTriangle{ABC} y sea $Q$ su conjugado isogonal respecto a \theTriangle{ABC}.
    Probar que
    \[
        \frac{AP \cdot AQ}{AB \cdot AC} + \frac{BP \cdot BQ}{BA \cdot BC} + \frac{CP \cdot CQ}{CA \cdot CB} = 1.
    \]
\end{section-problem}

\begin{section-problem}
    Sea el triángulo \theTriangle{ABC} y $P$ un punto en su interior.
    Sea $A_1$, $B_1$ y $C_1$ las intersecciones de $AP$, $BP$ y $CP$ con los lados $BC$, $CA$ y $AB$, respectivamente.
    Considerando a $X$, $Y$ y $Z$ como la intersecciones de $BC$ con $B_1 C_1$, $CA$ con $C_1 A_1$ y $AB$ con $A_1 B_1$, respectivamente.
    Probar que $X$, $Y$ y $Z$ son colineales.
\end{section-problem}

\begin{section-problem}
    Sea el triángulo \theTriangle{ABC} con incentro $I$.
    Sean $D, E, F$ puntos de tangecias de su circuncírculo con los lados $BC$, $CA$ y $AB$, respectivamente.
    Probar que los circuncírculos de los triángulos \theTriangle{AID}, \theTriangle{BIE} y \theTriangle{CIF} tiene dos puntos en común.
\end{section-problem}

\begin{section-problem}
    Sea $BCXY$ un rectángulo construido fuera del triángulo \theTriangle{ABC}.
    Sea $D$ pie de altura desde $A$ hacía $BC$ y sean $U$ y $V$ los puntos de intersección de $DY$ con $AB$ y $DX$ con $AC$, respectivamente.
    Probar que $UV || BC$.
\end{section-problem}

\begin{section-problem}
    Sea el triángulo \theTriangle{ABC} con $AB < AC$, el punto $H$ denota el ortocentro.
    Los puntos $A_1$ y $B_1$ son pies de alturas desde $A$ y $B$, respectivamente.
    El punto $D$ es la reflexión de $C$ respecto al punto $A_1$.
    Si $E = AC \cap DH$, $F = DH \cap A_1 B_1$ y $G = AF \cap BH$, probar que las rectas $CH$, $EG$ y $AD$ concurren.
\end{section-problem}

\begin{section-problem}
    Las circunferencias $C_1$ y $C_2$ son tangentes externamente.
    Las rectas tangentes desde $O_1$ hacia $C_2$ la tocan en $A$ y $B$; mientras que las rectas tangentes desde $O_2$ hacia $C_1$ la tocan en $C$ y $D$, respectivamente.
    Sean $E = O_1 A \cap O_2 C$ y $F = O_1 B \cap O_2 D$.
    Demostrar que $EF$, $O_1 O_2$, $AD$ y $BC$ concurren.
\end{section-problem}

\begin{section-problem}
    Sea el triángulo \theTriangle{ABC}, y sean los puntos $B_1$, $C_1$ sobre los lados $CA$ y $AB$ respectivamente.
    Sea $\Gamma$ el incírculo del \theTriangle{ABC} y sean $E$ y $F$ los puntos de tangencias de $\Gamma$ con los mismos lados $CA$ y $AB$, respectivamente.
    Además, se dibujan las tangentes desde $B_1$ y $C_1$ a \theTriangle{ABC} y se toma los puntos de tangecias $Z$ y $Y$, respectivamente.
    Probar que las rectas $B_1 C_1$, $EF$ y $YZ$ son concurrentes.
\end{section-problem}

\begin{section-problem}
    Sea el triángulo \theTriangle{ABC} y sea $P$ un punto en el interior del triángulo pedal \theTriangle{DEF}.
    Suponga que las rectas $DE$ y $DF$ son perpendiculares.
    Probar que si $Q$ es el conjugado isogonal de $P$ con respecto al triángulo \theTriangle{ABC}, entonces $Q$ es el ortocentro del triángulo \theTriangle{AEF}.
\end{section-problem}

\begin{section-problem}
    Sea \theTriangle{ABC} un triángulo cualquiera y $D$, $E$ y $F$ puntos cualesquiera sobre las rectas $BC$, $CA$ y $AB$ tal que las rectas $AD$, $BE$ y $CF$ concurren.
    La paralela a $AB$ por $E$ interseca a la recta $DF$ en el punto $Q$, la paralela a $AB$ por $D$ interseca a $EF$ en $T$.
    Probar que la rectas $CF$, $DE$ y $QT$ son concurrentes.
\end{section-problem}

\begin{section-problem}
    El punto $D$ está sobre el lado $AB$ del triángulo \theTriangle{ABC}.
    Sea $\omega_1$ y $\Omega_1$, $\omega_2$ y $\Omega_2$ los incírculos y los excírculos (tangentes al segmento $AB$) de los triángulos \theTriangle{ACD} y \theTriangle{BCD}, respectivamente.
    Probar que las tangentes externas comunes a $\omega_1$ y $\omega_2$, $\Omega_1$ y $\Omega_2$ se intersecan en $AB$.
\end{section-problem}

Nota: los problemas no están ordenados por orden de dificultad.