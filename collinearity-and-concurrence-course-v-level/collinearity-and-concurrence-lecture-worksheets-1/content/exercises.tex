\section{Ejercicios}

\subsection{Ejercicios de construcción - Pappus}

\begin{section-exercise}
    Aplicar el teorema de Pascal al hexágono $ABCDEF$. \hspace{1cm}
    \begin{tabular}{|c|c|c|}
        \hline
        \ \ \ \ && \\\hline
        &\ \ \ \ & \\\hline\hline
        &&\ \ \ \ \\\hline
    \end{tabular}
    \vspace*{\fill}
    \begin{figure}[H]
        \centering
        
%dash pattern=on 5pt off 2pt
%[fill = white, rounded corners = 5pt, inner sep=0.8pt]
\begin{tikzpicture}[scale = 1.2]
    \clip(-7.18,-0.99) rectangle (4.94,8.68);
    \draw [line width=1.2pt] (-3.5,4.81) circle (2.98cm);
    \begin{scriptsize}
        \normalsize
        \fill [color=black] (-4.5,7.62) circle (2.5pt);
        \draw[color=black] (-4.75,7.97) node {$A$};
        \fill [color=black] (-2.24,2.11) circle (2.5pt);
        \draw[color=black] (-1.96,1.72) node {$B$};
        \fill [color=black] (-5.84,2.97) circle (2.5pt);
        \draw[color=black] (-6.26,2.86) node {$C$};
        \fill [color=black] (-0.91,3.34) circle (2.5pt);
        \draw[color=black] (-0.54,3.22) node {$D$};
        \fill [color=black] (-0.52,4.85) circle (2.5pt);
        \draw[color=black] (-0.13,5.03) node {$E$};
        \fill [color=black] (-0.72,5.86) circle (2.5pt);
        \draw[color=black] (-0.4,6.13) node {$F$};
    \end{scriptsize}
\end{tikzpicture}
    \end{figure}
    \vspace*{\fill}
\end{section-exercise}

\newpage
\begin{section-exercise}
    Aplicar el teorema de Pappus, sabiendo que
    \begin{tabular}{|c|c|c|}
        \hline
        B & C & A\\\hline
        D & E & F\\
        \hline \hline
        &&\\
        \hline
    \end{tabular}
    \vspace*{\fill}
    \begin{figure}[H]
        \centering
        
%dash pattern=on 5pt off 2pt
%[fill = white, rounded corners = 5pt, inner sep=0.8pt]
\begin{tikzpicture}[scale = 1.2]
    \clip(-7.18,-0.99) rectangle (4.94,8.68);
    \draw [line width=1.2pt] (-3.5,4.81) circle (2.98cm);
    \begin{scriptsize}
        \normalsize
        \fill [color=black] (-4.5,7.62) circle (2.5pt);
        \draw[color=black] (-4.75,7.97) node {$A$};
        \fill [color=black] (-2.24,2.11) circle (2.5pt);
        \draw[color=black] (-1.96,1.72) node {$B$};
        \fill [color=black] (-5.84,2.97) circle (2.5pt);
        \draw[color=black] (-6.26,2.86) node {$C$};
        \fill [color=black] (-0.91,3.34) circle (2.5pt);
        \draw[color=black] (-0.54,3.22) node {$D$};
        \fill [color=black] (-0.52,4.85) circle (2.5pt);
        \draw[color=black] (-0.13,5.03) node {$E$};
        \fill [color=black] (-0.72,5.86) circle (2.5pt);
        \draw[color=black] (-0.4,6.13) node {$F$};
    \end{scriptsize}
\end{tikzpicture}
    \end{figure}
    \vspace*{\fill}
\end{section-exercise}




\newpage
\subsection{Ejercicios de perspectiva - Desargues}

\begin{section-exercise}
    Encuentre, con la ayuda de una regla, el punto y la recta para los cuales los triángulos \theTriangle{ABC} y \theTriangle{XYZ} están en perspectiva.

    \vspace*{\fill}
    \begin{figure}[H]
        \centering
        
%dash pattern=on 5pt off 2pt
%[fill = white, rounded corners = 5pt, inner sep=0.8pt]
\begin{tikzpicture}[scale = 0.95]
    \clip(-12.74,-2.62) rectangle (4.52,7.28);
    \draw [line width=1.2pt] (0.35,1.15) circle (2.64cm);
    \begin{scriptsize}
        \normalsize
        \fill [color=black] (-1.56,3.64) circle (3pt);
        \fill[color=black]  (3.62,4.09) circle (3pt);
        \fill[color=black] (2.68,-0.65) circle (3pt);
        \fill [color=black] (-0.26,-2.07) circle (3pt);
        \fill [color=black] (1.42,3.9) circle (3pt);
        \fill [color=black] (-2.67,0.54) circle (3pt);

        \draw[color=black] (-1.79,4.09) node {$A$};
        \draw[color=black] (3.84,4.47) node {$D$};
        \draw[color=black] (3.1,-0.63) node {$F$};
        \draw[color=black] (0.1,-2.23) node {$B$};
        \draw[color=black] (1.34,4.33) node {$E$};
        \draw[color=black] (-3.03,0.23) node {$C$};
    \end{scriptsize}
\end{tikzpicture}
    \end{figure}
    \vspace*{\fill}
\end{section-exercise}

\newpage
\begin{section-exercise}
    Encuentre, con la ayuda de una regla, el punto y la recta para los cuales los triángulos \theTriangle{ABC} y \theTriangle{XYZ} están en perspectiva.

    \vspace*{\fill}
    \begin{figure}[H]
        \centering
        
%dash pattern=on 5pt off 2pt
%[fill = white, rounded corners = 5pt, inner sep=0.8pt]
\begin{tikzpicture}[scale = 0.95]
    \clip(-12.74,-2.62) rectangle (4.52,7.28);
    \draw [line width=1.2pt] (0.35,1.15) circle (2.64cm);
    \begin{scriptsize}
        \normalsize
        \fill [color=black] (-1.56,3.64) circle (3pt);
        \fill[color=black]  (3.62,4.09) circle (3pt);
        \fill[color=black] (2.68,-0.65) circle (3pt);
        \fill [color=black] (-0.26,-2.07) circle (3pt);
        \fill [color=black] (1.42,3.9) circle (3pt);
        \fill [color=black] (-2.67,0.54) circle (3pt);

        \draw[color=black] (-1.79,4.09) node {$A$};
        \draw[color=black] (3.84,4.47) node {$D$};
        \draw[color=black] (3.1,-0.63) node {$F$};
        \draw[color=black] (0.1,-2.23) node {$B$};
        \draw[color=black] (1.34,4.33) node {$E$};
        \draw[color=black] (-3.03,0.23) node {$C$};
    \end{scriptsize}
\end{tikzpicture}
    \end{figure}
    \vspace*{\fill}
\end{section-exercise}