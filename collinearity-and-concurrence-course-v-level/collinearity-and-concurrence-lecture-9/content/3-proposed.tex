\section{Problemas propuestos}

Primeros diez problemas del trabajo de concurrencia y colinealidad.

\begin{section-problem}
    Sea $D$ el pie de altura desde $A$ en el triángulo \theTriangle{ABC} y $M$, $N$ puntos en los lados $CA$ y $AB$ talque las rectas $BM$ y $CN$ se intersecan en $AD$.
    Probar que $AD$ es bisectriz del ángulo $\angle MDN$.
\end{section-problem}

\begin{section-problem}
Sea \theTriangle{ABC} un triángulo con incentro $I$.
Sea $\Gamma$ un círculo centrado en $I$ con radio mayor al inradio.
Sean $X_1$ la intersección de $\Gamma$ con $AB$ más cercana a $B$; $X_2$ y $X_3$ las intersecciones de $\Gamma$ con $BC$ donde $X_2$ es más cercana a $B$; y $X_4$ la intersección de $\Gamma$ con $CA$ más cercana a $C$.
Sea $K$ la intersección de $X_1 X_2$ con $X_3 X_4$.
Probar que $AK$ biseca $X_2 X_3$.
\end{section-problem}

\begin{section-problem}
    Sea $ABCD$ un cuadrado y sea $X$ un punto en lado $BC$.
    Sea $Y$ un punto en la recta $CD$ tal que $BX = YD$ y $D$ se encuentra entre $C$ y $Y$.
    Demuestra que el punto medio de $XY$ se encuetra sobre la diagonal $BD$.
\end{section-problem}

\begin{section-problem}
Sean $\Gamma_1$ una circunferencia y $P$ un punto fuera de $\Gamma_1$.
Las rectas tangentes desde $P$ a $\Gamma_1$ tocan a la circunferencia en los puntos $A$ y $B$.
Considera $M$ el punto medio del segmento $PA$ y $\Gamma_2$ la circunferencia que pasa por los puntos $P, A$ y $B$.
La recta $BM$ interseca de nuevo a $\Gamma_2$ en el punto $C$, la recta $CA$ interseca de nuevo a $\Gamma_1$ en el punto $D$, el segmento $DB$ interseca de nuevo a $\Gamma_2$ en el punto $E$ y la recta $PE$ interseca a $\Gamma_1$ en el punto $F$ (con $E$ entre $P$ y $F$).
Muetra que las rectas $AF, BP$ y $CE$ concurren.
\end{section-problem}

\begin{section-problem}
Sea $\Omega$ el circuncírculo del triángulo \theTriangle{ABC} y sea $\omega_a$ la circunferencia tangente al segmento $CA$, segmento $AB$ y $\Omega$.
Se definen $\omega_b$ y $\omega_c$ de manera análoga.
Sea $A', B', C'$ los puntos de toque de $\omega_a, \omega_b, \omega_c$ con $\Omega$, respectivamente.
Probar que $AA', BB', CC'$ concurren en la recta $OI$ donde $O$ e $I$ son el cincuncentro y el incentro de \theTriangle{ABC}, respectivamente.
\end{section-problem}

\begin{section-problem}
    Sea $ABCD$ un trapezoide con $AB > CD$ y $AB || CD$.
    Sean los puntos $K, L$ sobre los segmentos $AB, CD$, respectivamente, tal que $\frac{AK}{KB} = \frac{DL}{LC}$.
    Suponga que existen los puntos $P, Q$ en la recta $KL$ que satisfacen $\angle APB = \angle BCD$ y $\angle CQD = \angle ABC$.
    Probar que los puntos $P, Q, B, C$ con concíclicos.
\end{section-problem}


\begin{section-problem}
    Los puntos $P$, $Q$ y $R$ están sobre los lados $AB$, $BC$ y $CA$ del triángulo acutángulo \theTriangle{ABC}, respectivamente.
    Si $\angle BAQ = \angle CAQ$, $QP \perp AB$, $QR \perp AC$ y $CP$ y $BR$ se intersecan en $S$ probar que $AS \perp BC$.
\end{section-problem}

\begin{section-problem}
    Sea \theTriangle{ABC} un triángulo con circuncentro $O$ y baricentro $G$.
    Sean $A', B', C'$ las reflexiones de los puntos medios de $BC, CA, AB$ con respecto a $O$, respectivamente.
    Probar que $AA', BB', CC'$ y $GO$ con concurrente.
\end{section-problem}

\begin{section-problem}
    Un triángulo isósceles \theTriangle{ABC} tiene base $AB$ y altura $CD$ con $BC = CA$.
    Sean $P$ un punto sobre $CD$, $E$ la intersección de la recta $AP$ con $BC$ y $F$ la intersección de la recta $BP$ con $CA$.
    Suponga que los incírculos del triángulo \theTriangle{ABP} y del cuadrilátero $PECF$ son congruentes.
    Demuestre que los incírculos de \theTriangle{ADP} y \theTriangle{BCP} son también congruentes.
\end{section-problem}

\begin{section-problem}
    Sea el triángulo \theTriangle{ABC} con $AC = BC$, sea $P$ un punto dentro del triángulo tal que $\angle PAB = \angle PBC$.
    Si $M$ es el punto medio de $AB$, entonce probar que $\angle APM + \angle BPC = 180^{\circ}$.
\end{section-problem}

Nota: los problemas no está ordenados por orden de dificultadad.