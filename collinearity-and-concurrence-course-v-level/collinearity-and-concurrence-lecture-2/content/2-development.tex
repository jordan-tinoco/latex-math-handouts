\section{Desarrollo}



\begin{section-theorem.tcb}[\textbf{Teorema de Ceva sobre la circunferencia}]
    Sean $ABC$ y $DEF$ dos triángulos sobre la misma circunferencia.
    Entonces las rectas $AD$, $BE$ y $CF$ son concurrentes si y sólo si
    \[\frac{BD}{DC} \cdot \frac{CE}{EA} \cdot \frac{AF}{FB} = 1.\]
\end{section-theorem.tcb}

\begin{section-definition.tcb}[\textbf{Triángulo circunceviano}]
    A todo punto que no esté sobre alguno de los lados de un triángulo dado es posible asignarle un nuevo triángulo, que surge a partir de la intersección de las cevianas con el circuncírculo del triángulo.
\end{section-definition.tcb}

\begin{section-theorem.tcb}[\textbf{Teorema de Steinbart}]
    Sea $ABC$ un triángulo, $D$, $E$ y $F$ los puntos de tanagencia del incírculo con los lados $BC$, $CA$ y $AB$, respectivamente.
    Sean $P$, $Q$ y $R$ puntos sobre el incírculo de $ABC$.
    Llamemos $A'$, $B'$ y $C'$ las intersecciones de $EF$ con $PD$, $DF$ con $QE$ y $DE$ con $FR$.
    Entonces $AP$, $BQ$ y $CR$ son concurrentes si y sólo si $DP$, $EQ$ y $FR$ son concurrentes.
\end{section-theorem.tcb}

\begin{section-theorem.tcb}[\textbf{Teorema de Jacobi}]
    Sea $ABC$ un triángulo, y sean $X$, $Y$, $Z$ tres puntos en el plano tales que $\angle YAC = \angle BAZ$, $\angle ZBA = \angle CBX$, $\angle XCB = \angle ACY$.
    Entonces las rectas $AX$, $BY$ y $CZ$ son concurrentes.
\end{section-theorem.tcb}

\begin{section-definition.tcb}[\textbf{Puntos isotómicos}]
    Dos puntos son isotómicos si estos coinciden al ser reflejados por el punto medio del segmento al que pertenecen.
\end{section-definition.tcb}

\begin{section-definition.tcb}[\textbf{Conjugados isotómicos}]
    Dado un triángulo $ABC$ se tienen tres cevianas $AD$, $BE$ y $CF$ las cuales son concurrentes en un punto $P$.
    Sean $D'$, $E'$ y $F'$ las reflexiones de $D$, $E$ y $F$ sobre los puntos medios de $BC$, $CA$ y $AB$ respectivamente.
    Entonces las rectas $AD'$, $BE'$ y $CF'$ son concurrentes.
\end{section-definition.tcb}

\begin{section-definition.tcb}[\textbf{Cevianas isogonales}]
    Dos cevianas son isogonales del $\triangle ABC$ si ambas parte del mismo vértice del triángulo y una es la reflexión de la otra con respecto a la bisectriz interna de $\triangle ABC$.
\end{section-definition.tcb}

\begin{section-definition.tcb}[\textbf{Conjugados isogonales}]
    Dado un triángulo $ABC$ se tienen tres cevianas $AD$, $BE$ y $CF$ las cuales son concurrentes en un punto $P$.
    Sean $AD'$, $BE'$ y $CF'$ las reflexiones de $AD$, $BE$ y $CF$ sobre las bisectrices de $\angle A$, $\angle B$ y $\angle C$ respectivamente.
    Entonces las rectas $AD'$, $BE'$, $CF'$ son concurrentes.
\end{section-definition.tcb}



\section{Ejercicios y Problemas}
Sección de ejercicios y problemas para el autoestudio.

\begin{section-exercise}
    En un triángulo $ABC$ en el cual se traza la altura $BH$, la mediana $AM$ y la ceviana $CN$ las cuales concurren en el punto $P$.
    Si $BP = 3PH$ y $NB = 16$.
    Hallar $AN$.
\end{section-exercise}

\begin{section-exercise}
    Si $P$ y $Q$ son puntos en $AB$ y $AC$ del triángulo $ABC$ de tal forma que $PQ$ es paralelo a $BC$, y si $BQ$ y $CP$ se cortan en $O$, demuestra que $AO$ es una mediana.
\end{section-exercise}

\begin{section-exercise}
    Sean $L$, $M$ y $N$ puntos en los lados $BC$, $CA$ y $AB$ de un triángulo, respectivamente.
    Si $AL$, $BM$ y $CN$ concurren en $O$, demostrar que
    \[\frac{OL}{AL} + \frac{OM}{BM} + \frac{ON}{CN} = 1.\]
\end{section-exercise}

\begin{section-exercise}
    Sean $L$, $M$ y $N$ puntos en los lados $BC$, $CA$ y $AB$ de un triángulo, respectivamente.
    Si $AL$, $BM$ y $CN$ concurren en $O$, demostrar que
    \[\frac{AO}{OL} = \frac{AN}{NB} + \frac{AM}{MC}.\]
\end{section-exercise}

\begin{section-problem}
    Sea $ABC$ un triángulo.
    Se toman los puntos $D$, $E$ y $F$ en las mediatrices de $BC$, $CA$ y $AB$ respectivamente.
    Probar que las rectas que pasan por $A$, $B$ y $C$ que son perpendiculares a $EF$, $FD$ y $DE$, respectivamente, son concurrentes.
\end{section-problem}
