\documentclass[12pt]{article}
\usepackage{mlmodern}
\usepackage[T1]{fontenc}
\usepackage[spanish]{babel}
\usepackage{fancyhdr}
\usepackage[left=2.5cm, right=2.5cm, top=2cm, bottom=2cm]{geometry}
\usepackage{multicol}
\usepackage{tasks}

\settasks{label=\Alph*)}

\renewcommand{\normalsize}{\fontsize{13}{15}\selectfont} %propuesta 1 (13.5, 16)

\setlength{\parskip}{1.7mm}
\setlength{\parindent}{0pt}
\setlength{\headsep}{3.5mm}

\title{Manual introductorio a Python}
\author{Kenny J. Tinoco}
\date{Febrero de 2025}

\begin{document}
    \maketitle

    \section{Variables}
    Una variable en Python es una dirección un nombre que el programador define la cual funciona como un puntero a cierta zona de memoria la cual tiene un valor.
    
    \section{Operadores}
    
    \section{Números}
    
    \section{Sentencias de control}

    \section{Cíclos}
    
    \section{Funciones}

\end{document}