\documentclass[12pt]{article}
\usepackage{mlmodern}
\usepackage[T1]{fontenc}
\usepackage[spanish]{babel}
\usepackage{fancyhdr}
\usepackage[left=2.5cm, right=2.5cm, top=2cm, bottom=2cm]{geometry}
\usepackage{multicol}
\usepackage{tasks}
\usepackage{listings}
\usepackage{xcolor}

\settasks{label=\Alph*)}

\lstset{
    language=Python,
    basicstyle=\ttfamily\small,
    numbers=left,            % Números de línea (elimínalos si no los quieres)
    numberstyle=\color{blue}, % Estilo de los números de línea
    numbersep=-6pt, % Ajusta este valor (puedes probar con 3pt o 2pt)
    showspaces=false,
    showstringspaces=false,
    tabsize=4,
    frame=single, % Agrega un marco alrededor del código
    rulecolor=\color{blue!30}, % Color del borde del recuadro
    backgroundcolor=\color{cyan!10} % Fondo celeste muy suave
}
\renewcommand{\normalsize}{\fontsize{13}{15}\selectfont} %propuesta 1 (13.5, 16)

\setlength{\parskip}{1.7mm}
\setlength{\parindent}{0pt}
\setlength{\headsep}{3.5mm}

\title{Manual introductorio a Python}
\author{Kenny J. Tinoco}
\date{Febrero de 2025}

\begin{document}
    \maketitle

    \tableofcontents

    \section{Variables}
    Una \textbf{variable} en Python es el nombre que se asigna un dato almacenado con los cuales se harán operaciones,
    estos datos pueden ser de distintas naturalezas que denotaremos como \textbf{tipo}, es decir, pueden ser de tipos
    enteros (\textbf{int}), decimales (\textbf{float} o \textbf{double}), caracteres (\textbf{char}),
    textos (\textbf{string}), valores booleanos (\textbf{bool}), listas (\textbf{list}), etc.

    Para crear una variable escribimos su nombre y luego asignamos su valor por medio del operador de asignación ``$=$''.

    Una cosa importante a tener en cuenta es que el nombre de una variable tiene ciertas restricciones, la primera es
    que el nombre no puede iniciar por un número, la segunda es que no puede ser una palabra reservada del lenguaje y
    la tercera es que no puede tener caracteres especiales excepto el ``\_'', vamos algunos ejemplos.
    \begin{lstlisting}
        aString = "cuatro"
        aNumber = 4
        aChar = 'c'
        x = -1
        isTrue = true
        isFalse = false
        xIsPositive = x > 0  #(es falso)
    \end{lstlisting}

    Variables validas:
    \begin{lstlisting}
        a1239892874 = 1
        AGE = 67
        my_lastname = "lopez"
        _name = "luis"
    \end{lstlisting}

    Variables invalidas
    \begin{lstlisting}
        1x = 1
        name-value = "luis"
        first% = 3
    \end{lstlisting}

    Otro aspecto muy importante es escribir nombre de variables que tenga un sentido dentro del código, evitar escribir
    variables de un solo caracter y en su lugar escribir un sustantivo.
    
    \section{Operadores}
    -- Pendiente --
    
    \section{Números}
    -- Pendiente --
    
    \section{Sentencias de control}
    -- Pendiente --

    \section{Cíclos}
    -- Pendiente --
    
    \section{Funciones}

    -- Pendiente --

\end{document}