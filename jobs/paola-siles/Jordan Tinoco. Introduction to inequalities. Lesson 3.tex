\documentclass[12pt]{article}

\usepackage{mlmodern}
\usepackage[T1]{fontenc}
\usepackage[spanish]{babel}

\usepackage{cite}
\usepackage{hyperref}
\usepackage{fancyhdr}
\usepackage[Lenny]{fncychap}
\usepackage{microtype}
\usepackage{makeidx}
\usepackage{bookmark}
\usepackage
[
    a5paper,
    left = 1cm,
    right = 1cm,
    top = 2cm,
    bottom = 1cm
]
{geometry}
\hypersetup
{
    colorlinks = true,
    linkcolor = black
}

\setlength{\parskip}{2mm}
\setlength{\parindent}{0pt}

\makeindex

\renewcommand{\qedsymbol}{$\blacksquare$}
\addto\captionsspanish{\renewcommand{\proofname}{\textnormal{\textbf{Demostración}}}}
\DeclareSymbolFont{yhlargesymbols}{OMX}{yhex}{m}{n}
\DeclareMathAccent{\wideparen}{\mathord}{yhlargesymbols}{"F3}
\DeclarePairedDelimiter{\abs}{\lvert}{\rvert}

%\renewcommand{\theenumi}{\alph{enumi}}

%Number sets
\newcommand{\N}{\ensuremath{\mathbb{N}}}
\newcommand{\Z}{\ensuremath{\mathbb{Z}}}
\newcommand{\Q}{\ensuremath{\mathbb{Q}}}
\newcommand{\R}{\ensuremath{\mathbb{R}}}
\newcommand{\C}{\ensuremath{\mathbb{C}}}


%Useful commands
\newcommand{\ds}{\displaystyle}
\newcommand{\ie}{\ensuremath{\text{i.e.}}}
\newcommand{\eg}{\ensuremath{\text{e.g.}}}
\renewcommand{\emptyset}{\varnothing}
\newcommand{\fullMod}[2]{\equiv #1 \pmod{#2}}
\newcommand{\mcd}[2]{\ensuremath{mcd(#1, #2)}}
\newcommand{\mcm}[2]{\ensuremath{mcm(#1, #2)}}

%Enviroments
\newenvironment{solution}[1][]
{
    \ifstrempty{#1}
    {
        \begin{proof}[\textnormal{\textbf{Solución}}]
    }
    {
        \begin{proof}[\textnormal{\textbf{Solución #1}}]
    }
    }{
    \end{proof}
}
\usepackage{ifthen}
\usepackage{tcolorbox}
\usepackage{varwidth}

\tcbuselibrary{theorems}
\tcbuselibrary{breakable}
\tcbuselibrary{skins}

\theoremstyle{definition}

\newtheorem{counter}{Contador}[section]
\newtheorem{problem}{Problema}[section]
\newtheorem{corollary}{Corolario}[chapter]
\newtheorem{example}{Ejemplo}[chapter]
\newtheorem{definition}{Definición}[chapter]
\newtheorem{remark}{Observación}
\newtheorem*{note}{Nota}

\newenvironment{solution}[1][]
{
    \begin{proof}[\textnormal{\textbf{Solución\ifthenelse{\equal{#1}{}}{}{ #1}}}]
    }{
    \end{proof}
}

\newtcbtheorem[number within=chapter]{theorem}{Teorema}
{
    enhanced
    ,colback = gray!10!white
    ,frame hidden
    ,boxrule = 0sp
    ,borderline west = {2.5pt}{0pt}{black}
    ,sharp corners
    ,attach title to upper
    ,coltitle = black
    ,fonttitle = \bfseries
    ,description font = \mdseries
    ,separator sign none,
    ,terminator sign={.\hspace{2mm}}
    ,description delimiters parenthesis,
    right=1mm,
    top=0mm,
    left=1.5mm,
    bottom=0mm,
    breakable = true
}{t}




\title{Introducción a desigualdades\\Medias}
\author{Kenny J. Tinoco}
\date{Marzo de 2025}

\begin{document}
    \maketitle

    \section{Definiciones}

    La desigualdad de medias es una parte importante en el estudio de las desigualdades, esta nos permite
    simplificar el tratamiento que realizamos sobre las expresiones, reduciendo el tiempo y esfuerzo de resolución.

    \begin{theorem}[Teorema de las medias]
        Dado los números reales positivos $a_1, a_2, \ldots , a_k$ se cumple que
        \[
            \sqrt {\frac{a_1^2 + a_2^2 + \cdots + a_k^2}{k}} \geq \frac{a_1 + a_2 + \cdots + a_k}{k} \geq \sqrt[k]{a_1 a_2 \ldots a_k} \geq \frac{k}{\frac{1}{a_1} + \frac{1}{a_2} + \cdots + \frac{1}{a_k}},
        \]
        donde el caso de igualdad se da si y solo si $a_1 = a_2 = \cdots = a_k$.
    \end{theorem}

    De izquierda a derecha llamaremos a estas cantidades como media cuadrática (\textbf{MC}), media aritmética (\textbf{MA}), media geométrica (\textbf{MG}) y media armónica (\textbf{MH}).

    Muchas desigualdades pueden ser resueltas con una aplicación concreta de las medias de la mano de un buen manipuleo.

    \section{Problemas}

    \begin{prob-without-section}
        Sean $a,b,c > 0$ probar que
        \[
            \frac{a + b}{c} + \frac{b + c}{a} + \frac{c + a}{b} \geq 6.
        \]
    \end{prob-without-section}

    \begin{prob-without-section}
        Probar para todo $x$ real que se cumple
        \[
            \frac{x^2 + 2}{\sqrt {x^2 + 1}} \geq 2.
        \]
    \end{prob-without-section}

    \begin{prob-without-section}
        Para los números no negativos $a,b,c,d$ demostrar que se cumple
        \[
            \sqrt {(a + c)(b + d)} \geq \sqrt {ab} + \sqrt {cd}.
        \]
    \end{prob-without-section}

    \begin{prob-without-section}
        Sean $a,b,c > 0$ probar que
        \[
            \frac{(a + b + c)^3}{27} \geq \frac{(a + b)(b + c)(c + a)}{8}.
        \]
    \end{prob-without-section}

    \begin{prob-without-section}
        Sean los reales positivos $a,b,c$, demostrar que
        \[
            (a + b + c)^3 \geq a^3 + b^3 + c^3 + 24abc.
        \]
    \end{prob-without-section}

    \begin{prob-without-section}
        Dados los reales positivos $a,b$ y $c$, probar que
        \[
            \frac{a^3 + b^3 + c^3}{3} \geq \frac{a^2 b + b^2 c + c^2 a + a b^2 + b c^2 + c a^2}{6} \geq abc.
        \]
    \end{prob-without-section}

    \begin{prob-without-section}
        Dado los números positivos $w,x,y,z$ probar que
        \[
            (w + x + y + z)\left(\frac{1}{w} + \frac{1}{x} + \frac{1}{y} + \frac{1}{z}\right) \geq 16.
        \]
        De manera general, demostrar que para $n$ números reales positivos $x_i$ se cumple
        \[
            (x_1 + x_2 + \ldots + x_n)\left(\frac{1}{x_1} + \frac{1}{x_2} + \ldots + \frac{1}{x_n}\right) \geq n^2.
        \]
    \end{prob-without-section}

    \begin{prob-without-section}
        Dado los reales postivos $a,b,c$, probar que se cumple
        \[
            a^3 + b^3 + c^3 + ab^2  + bc^2  + ca^2  \geq 2(a^2 b + b^2 c + c^2 a).
        \]
    \end{prob-without-section}

    \begin{prob-without-section}
        Demostrar que para todo entero $n$ y los reales positivos $a,b$ se cumple que
        \[
            a^{n + 1} + b^{n + 1} \geq a^n b + a b^n.
        \]
    \end{prob-without-section}

    \begin{prob-without-section}
        Dado los reales positivos $x,y$ demostrar que
        \[
            \frac{1}{xy} \geq \frac{x}{x^4 + y^2} + \frac{y}{y^4 + x^2}.
        \]
    \end{prob-without-section}

    \begin{prob-without-section}
        Sí $a,b,c > 0$. Demostrar que
        \[
            \frac{1}{a^2} + \frac{1}{b^2} + \frac{1}{c^2} \geq \frac{a + b + c}{abc}
        \]
    \end{prob-without-section}

    \begin{prob-without-section}
        Para cualquiera números reales $x, y > 1$, demostrar que
        \[
            \frac{x^2}{y - 1} + \frac{y^2}{x - 1} \geq 8.
        \]
    \end{prob-without-section}

    \begin{prob-without-section}
        Para todos los reales no negativos $x,y,z$ demostrar que
        \[
            \frac{(x + y + z)^2}{3} \geq x \sqrt {yz} + y \sqrt {zx} + z \sqrt {xy}.
        \]
    \end{prob-without-section}

    \begin{prob-without-section}
        Para números reales positivos $a,b,c$ tales que $a + b + c = 1$, demostrar que
        \[
            \frac{1}{3} \geq ab + bc + ca.
        \]
    \end{prob-without-section}

    \begin{prob-without-section}
        Sean $x,y,z$ reales positivos, probar que
        \[
            \frac{2}{x + y} + \frac{2}{y + z} + \frac{2}{z + x} \geq \frac{9}{x + y + z}.
        \]
    \end{prob-without-section}

    \begin{prob-without-section}
        Sean $a_1, a_2, \ldots, a_k$ números reales positivos tales que $a_1 a_2 \cdot a_k = 1$, demostrar que
        \[
            (1 + a_1)(1 + a_2) \cdots (1 + a_k) \geq 2^n.
        \]
    \end{prob-without-section}

\end{document}