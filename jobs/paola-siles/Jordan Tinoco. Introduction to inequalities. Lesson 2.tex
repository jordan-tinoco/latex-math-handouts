\documentclass[12pt]{article}
\usepackage{amsmath}

\usepackage{mlmodern}
\usepackage[T1]{fontenc}
\usepackage[spanish]{babel}
\usepackage{fancyhdr}
\usepackage[left=2.5cm, right=2.5cm, top=2cm, bottom=2cm]{geometry}
\usepackage{multicol}
\usepackage{tasks}

\settasks{label=\Alph*)}

\renewcommand{\normalsize}{\fontsize{13}{15}\selectfont} %propuesta 1 (13.5, 16)
%\renewcommand{\normalsize}{\fontsize{13}{15.5}\selectfont} %propuesta 2 (13, 15.5)

\setlength{\parskip}{1.7mm}
\setlength{\parindent}{0pt}
\setlength{\headsep}{3.5mm}

\renewcommand{\qedsymbol}{$\blacksquare$}
\renewcommand{\emptyset}{\varnothing}
\renewcommand{\max}[1]{\ensuremath{máx #1}}
\newcommand{\modulo}[2]{\equiv #1\ (\text{mód}\ #2)}
\renewcommand{\proofname}{\textnormal{\textbf{Demostración}}}

\newcommand{\ie}{\ensuremath{\text{i.e.}}}
\newcommand{\eg}{\ensuremath{\text{e.g.}}}

%Number sets
\newcommand{\N}{\ensuremath{\mathbb{N}}}
\newcommand{\Z}{\ensuremath{\mathbb{Z}}}
\newcommand{\Q}{\ensuremath{\mathbb{Q}}}
\newcommand{\irr}{\ensuremath{\mathbb{Q}'}}
\newcommand{\R}{\ensuremath{\mathbb{R}}}
\newcommand{\C}{\ensuremath{\mathbb{C}}}

\newcommand{\positiveSet}[1]{\ensuremath{#1^+}}
\newcommand{\myhrule}[1]{\vspace{#1mm}\hrule}
\newcommand{\negativeSet}[1]{\ensuremath{#1^-}}
\newcommand{\nonnegativeSet}[1]{\ensuremath{#1^{\geq 0}}}

%Useful math commands
\newcommand{\ds}{\displaystyle}
\newcommand{\invers}[1]{\frac{1}{#1}}
\newcommand{\inversD}[1]{\ensuremath{\dfrac{1}{#1}}}
\newcommand{\mcd}[2]{\ensuremath{\text{mcd}(#1, #2)}}
\newcommand{\refTheo}[1]{\textbf{Teorema #1}}
\newcommand{\refDef}[1]{\textbf{Definición #1}}
\newcommand{\ddotsr}{\ds\cdot^{\ds\cdot^{\ds\cdot}}}

\newcommand{\symLegendre}[2]{\ensuremath{\left(\dfrac{#1}{#2}\right)}}
\theoremstyle{definition}

%With section
\newtheorem{lemma}{Lema}[section]
\newtheorem{example}{Ejemplo}[section]
\newtheorem{theorem}{Teorema}[section]
\newtheorem{problem}{Problema}[section]
\newtheorem{problem-wos}{Problema}
\newtheorem{property}{Propiedad}[section]
\newtheorem{exercise}{Ejercicio}
\newtheorem{prob-without-section}{Problema}
\newtheorem{corollary}{Corolario}[section]
\newtheorem{definition}{Definición}[section]
\newtheorem{axiom}{Axioma}[section]
\newtheorem{principle}{principio}[section]


\newtheorem{proposition}{Proposición}


\newcounter{activitycounter}
\NewDocumentEnvironment{activity}{O{} O{} +b}
{
    \stepcounter{activitycounter}
    \textbf{Act. \theactivitycounter}\ \textbf{[#1]}\textbf{[#2]}.\hspace{2mm}#3\par
    \smallskip
}{}

%Without numeration
\newtheorem*{note}{Nota}
\newtheorem*{hint}{Pista}

\newenvironment{solution}[1][]
{
    \begin{proof}[\textnormal{\textbf{Solución\ifthenelse{\equal{#1}{}}{}{ #1}}}]
    }{
    \end{proof}
}


\newtcbtheorem[use counter*= theorem, number within=section]{theorem.box}{Teorema}
{
    enhanced
    ,colback = orange!23!white
    ,boxrule = 0.1mm
    ,colframe = gray
    ,borderline west = {2.6pt}{0pt}{black}
    ,attach title to upper
    ,coltitle = black
    ,fonttitle = \bfseries
    ,description font = \mdseries
    ,separator sign none,
    ,terminator sign={.\hspace{2mm}}
    ,description delimiters parenthesis,
    right=1mm,
    top=0mm,
    left=1.5mm,
    bottom=0mm,
    breakable = true,
    arc = 3pt,
    outer arc = 3pt,
    before skip=0.5\baselineskip,
    after skip=0.4\baselineskip
}
{t}

\newtcbtheorem[number within=section]{definition.box}{Definición}
{
    colback = teal!16!white
    ,coltitle = black
    ,colframe = white
    ,boxrule = 0.3mm
    ,attach title to upper
    ,fonttitle = \bfseries
    ,description font = \mdseries
    ,separator sign none
    ,terminator sign={.\hspace{2mm}}
    ,description delimiters parenthesis,
    right=1mm,
    top=0mm,
    left=1mm,
    bottom=0mm,
    arc = 3pt,
    outer arc = 3pt,
    before skip=0.5\baselineskip,
    after skip=0.4\baselineskip
}
{d}




\newtcolorbox[auto counter]{remark.box}[1][]
{
    breakable,
    title = Observación~\thetcbcounter.,
    colback = white,
    colbacktitle = violet!15!white,
    coltitle = black,
    fonttitle = \bfseries,
    bottomrule = 0pt,
    toprule = 0pt,
    leftrule = 0pt,
    rightrule = 0pt,
    titlerule = 0pt,
    arc = 2pt,
    outer arc = 2pt,
    colframe = black,
    right=1mm,
    top=0mm,
    left=1mm,
    bottom=0mm,
}




\newtcbtheorem[use counter*= principle, number within=section]{principle.box}{Principio}
{
    enhanced
    ,colback = orange!23!white
    ,boxrule = 0.1mm
    ,colframe = gray
    ,borderline west = {2.5pt}{0pt}{black}
    ,attach title to upper
    ,coltitle = black
    ,fonttitle = \bfseries
    ,description font = \mdseries
    ,separator sign none,
    ,terminator sign={.\hspace{2mm}}
    ,description delimiters parenthesis,
    right=1mm,
    top=0mm,
    left=1.5mm,
    bottom=0mm,
    breakable = true,
    arc = 3pt,
    outer arc = 3pt
}
{t}
\renewcommand{\qedsymbol}{$\blacksquare$}
\renewcommand{\emptyset}{\varnothing}
\renewcommand{\max}[1]{\ensuremath{máx #1}}
\newcommand{\modulo}[2]{\equiv #1\ (\text{mód}\ #2)}
\renewcommand{\proofname}{\textnormal{\textbf{Demostración}}}

\newcommand{\ie}{\ensuremath{\text{i.e.}}}
\newcommand{\eg}{\ensuremath{\text{e.g.}}}

%Number sets
\newcommand{\N}{\ensuremath{\mathbb{N}}}
\newcommand{\Z}{\ensuremath{\mathbb{Z}}}
\newcommand{\Q}{\ensuremath{\mathbb{Q}}}
\newcommand{\irr}{\ensuremath{\mathbb{Q}'}}
\newcommand{\R}{\ensuremath{\mathbb{R}}}
\newcommand{\C}{\ensuremath{\mathbb{C}}}

\newcommand{\positiveSet}[1]{\ensuremath{#1^+}}
\newcommand{\myhrule}[1]{\vspace{#1mm}\hrule}
\newcommand{\negativeSet}[1]{\ensuremath{#1^-}}
\newcommand{\nonnegativeSet}[1]{\ensuremath{#1^{\geq 0}}}

%Useful math commands
\newcommand{\ds}{\displaystyle}
\newcommand{\invers}[1]{\frac{1}{#1}}
\newcommand{\inversD}[1]{\ensuremath{\dfrac{1}{#1}}}
\newcommand{\mcd}[2]{\ensuremath{\text{mcd}(#1, #2)}}
\newcommand{\refTheo}[1]{\textbf{Teorema #1}}
\newcommand{\refDef}[1]{\textbf{Definición #1}}
\newcommand{\ddotsr}{\ds\cdot^{\ds\cdot^{\ds\cdot}}}

\newcommand{\symLegendre}[2]{\ensuremath{\left(\dfrac{#1}{#2}\right)}}
\theoremstyle{definition}

%With section
\newtheorem{lemma}{Lema}[section]
\newtheorem{example}{Ejemplo}[section]
\newtheorem{theorem}{Teorema}[section]
\newtheorem{problem}{Problema}[section]
\newtheorem{problem-wos}{Problema}
\newtheorem{property}{Propiedad}[section]
\newtheorem{exercise}{Ejercicio}
\newtheorem{prob-without-section}{Problema}
\newtheorem{corollary}{Corolario}[section]
\newtheorem{definition}{Definición}[section]
\newtheorem{axiom}{Axioma}[section]
\newtheorem{principle}{principio}[section]


\newtheorem{proposition}{Proposición}


\newcounter{activitycounter}
\NewDocumentEnvironment{activity}{O{} O{} +b}
{
    \stepcounter{activitycounter}
    \textbf{Act. \theactivitycounter}\ \textbf{[#1]}\textbf{[#2]}.\hspace{2mm}#3\par
    \smallskip
}{}

%Without numeration
\newtheorem*{note}{Nota}
\newtheorem*{hint}{Pista}

\newenvironment{solution}[1][]
{
    \begin{proof}[\textnormal{\textbf{Solución\ifthenelse{\equal{#1}{}}{}{ #1}}}]
    }{
    \end{proof}
}


\newtcbtheorem[use counter*= theorem, number within=section]{theorem.box}{Teorema}
{
    enhanced
    ,colback = orange!23!white
    ,boxrule = 0.1mm
    ,colframe = gray
    ,borderline west = {2.6pt}{0pt}{black}
    ,attach title to upper
    ,coltitle = black
    ,fonttitle = \bfseries
    ,description font = \mdseries
    ,separator sign none,
    ,terminator sign={.\hspace{2mm}}
    ,description delimiters parenthesis,
    right=1mm,
    top=0mm,
    left=1.5mm,
    bottom=0mm,
    breakable = true,
    arc = 3pt,
    outer arc = 3pt,
    before skip=0.5\baselineskip,
    after skip=0.4\baselineskip
}
{t}

\newtcbtheorem[number within=section]{definition.box}{Definición}
{
    colback = teal!16!white
    ,coltitle = black
    ,colframe = white
    ,boxrule = 0.3mm
    ,attach title to upper
    ,fonttitle = \bfseries
    ,description font = \mdseries
    ,separator sign none
    ,terminator sign={.\hspace{2mm}}
    ,description delimiters parenthesis,
    right=1mm,
    top=0mm,
    left=1mm,
    bottom=0mm,
    arc = 3pt,
    outer arc = 3pt,
    before skip=0.5\baselineskip,
    after skip=0.4\baselineskip
}
{d}




\newtcolorbox[auto counter]{remark.box}[1][]
{
    breakable,
    title = Observación~\thetcbcounter.,
    colback = white,
    colbacktitle = violet!15!white,
    coltitle = black,
    fonttitle = \bfseries,
    bottomrule = 0pt,
    toprule = 0pt,
    leftrule = 0pt,
    rightrule = 0pt,
    titlerule = 0pt,
    arc = 2pt,
    outer arc = 2pt,
    colframe = black,
    right=1mm,
    top=0mm,
    left=1mm,
    bottom=0mm,
}




\newtcbtheorem[use counter*= principle, number within=section]{principle.box}{Principio}
{
    enhanced
    ,colback = orange!23!white
    ,boxrule = 0.1mm
    ,colframe = gray
    ,borderline west = {2.5pt}{0pt}{black}
    ,attach title to upper
    ,coltitle = black
    ,fonttitle = \bfseries
    ,description font = \mdseries
    ,separator sign none,
    ,terminator sign={.\hspace{2mm}}
    ,description delimiters parenthesis,
    right=1mm,
    top=0mm,
    left=1.5mm,
    bottom=0mm,
    breakable = true,
    arc = 3pt,
    outer arc = 3pt
}
{t}

\title{Introducción a desigualdades}
\author{Kenny J. Tinoco}
\date{Diciembre de 2024}

\begin{document}
    \maketitle

    \section{Definiciones}
    Empezaremos recordando unas tautologías sobre las proposiciones.\\

    \textbf{Implicación ($\implies$)}\\
    Dada dos proposiciones lógicas $A$ y $B$ diremos que $A$ implica a $B$ si el valor lógico de $A$ causa el valor lógico de $B$.
    Denotaremos esta relación por el símbolo
    \[
        A \implies B.
    \]
    Existe varias maneras de verbalizar esta relación, por ejemplo: si $A$, entonces $B$, $A$ causa $B$, para que $B$ es necesario $A$, entre otros.\\

    \textbf{Doble implicación ($\iff$)}\\
    Si tenemos que $A \implies B$ y $B \implies A$ a la vez, podemos decir que existe una doble implicación entre $A$ y $B$.
    Lo denotamos por
    \[
        A \iff B.
    \]
    Verbalizando esto es igual a: $A$ si y solo si $B$, $A \equiv B$.

    Ahora recordemos las propiedades importantes de las desigualdades.
    Sean $a, b$ y $c$ dos números reales, se cumple.
    \begin{multicols}{2}
        \begin{enumerate}
            \item \textbf{Tricotomía}:\\ Se cumple que $a < b$, $a = b$ o $a > b$.
            \item \textbf{Suma:}\\ Si $a \geq b$, entonces $a + c \geq b + c$ para cualquier $c$.
            \item \textbf{Multiplicación:}\\
            Si $a \geq b$ y $c \geq 0$, entonces $ac \geq bc$\\
            Si $a \geq b$ y $c \leq 0$, entonces $ac \leq bc$
            \item \textbf{Reflexividad:}\\ $a \geq a$
            \item \textbf{Antisimetría:}\\ Si $a \geq b$ y $b\geq a$, entonces $a = b$.
            \item \textbf{Transitividad:}\\ Si $a \geq b$ y $b \geq c$, entonces $a \geq c$\\.
        \end{enumerate}
    \end{multicols}

    \section{Técnicas}
    Cuando resolvemos desigualdades hay maneras comunes de avanzar con la solución, estas maneras son las técnicas.\\

    \textbf{Comparación directa}
    \begin{enumerate}
        \item La desigualdad $A \geq B$ es verdadera si y solo si $A - B \geq 0$ es también verdadera.
        \item La desigualdad $A \geq B$ con $B \neq 0$ es verdadera si y solo si $\frac{A}{B} \geq 1$ es verdadera.
    \end{enumerate}

    \textbf{Maximizar y reducir}
    \par
    Si tenemos la desigualdad $A \geq B$, entonces podemos partir la demostración encontrando valores $A_1, A_2, \ldots, A_k$ tales que
    \[
        A \geq A_1 \geq A_2 \geq \ldots \geq A_k \geq B.
    \]

    \textbf{Análisis de la desigualdad}
    \par
    Si tenemos la desigualdad $A \geq B$, entonces podemos partir la demostración encontrando desigualdades $(A_1 \geq B_1), (A_2 \geq B_2), \ldots, (A_k \geq B_k)$ tales que
    \[
       A \geq B \iff (A_1 \geq B_1) \iff (A_2 \geq B_2) \iff \ldots \iff (A_k \geq B_k)
    \]
    Muchas veces utilizaremos teoremas sobre desigualdades para simplificar las demostraciones de las desigualdades.

    \section{Problemas}

    \begin{prob-without-section}
        Si $a,b,c$ son números reales arbitrarios, entonces probar que
        \[
            a^2 + b^2 + c^2 \geq ab + bc + ca.
        \]
    \end{prob-without-section}

    \begin{prob-without-section}
        Si $a,b$ son números reales tales que $a + b = 2$, probar que
        \[
            a^4 + b^4 \geq 2.
        \]
    \end{prob-without-section}

    \begin{prob-without-section}
        Si $a,b,c$ son números reales positivos, probar que
        \[
            a^3 + b^3 + c^3 + ab^2 + bc^2 + ca^2 \geq 2(a^2b + b^2 c + c^2 a).
        \]
    \end{prob-without-section}

    \begin{prob-without-section}
        Sea $a,b,x,y$ números reales cualesquiera.
        Demostrar que
        \[
            (a^2 + b^2)(x^2 + y^2) \geq (ax + by)^2.
        \]
    \end{prob-without-section}

    \begin{prob-without-section}
        Probar que $a^2 + b^2 + c^2 + 3 \geq 2(a + b + c)$
    \end{prob-without-section}

    \begin{prob-without-section}
        Si $x,y$ son reales positivos, probar que $x^2 + y^2 \geq \dfrac{(x - y)^2}{2}$.
    \end{prob-without-section}

    \begin{prob-without-section}
        Probar que $x^4 + y^4 + z^2 + 1 \geq 2x(xy^2 - x + z + 1)$.
    \end{prob-without-section}

    \begin{prob-without-section}
        Probar que para cualquier real $x$ se cumple que
        \[
            (x - 1)(x - 3)(x - 4)(x - 6) + 10 > 0.
        \]
    \end{prob-without-section}

\end{document}