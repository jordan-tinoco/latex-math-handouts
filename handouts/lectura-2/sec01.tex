\section{¿Por qué apreder matemáticas a través de la escritura?}

Hay varias razones para contestar al título de esta sección; unas son más nobles que otras.
Entre las menos nobles se cuenta la irritación ante una situación que lleva enquistada años y años y ante la que pocos hacen algo.
Me estoy refiriendo a las respuestas de los alumnos en los exámenes.
La mayor parte de los profesores admiten respuestas que son poco menos que un código privado interno.
Cuando corregimos, nos vemos obligados a adivinar lo que quieren decir los alumnos, a interpretarlo cual exégetas de
una lengua muerta; nos vemos forzados a separar forma y contenido brutalmente en contra de la propia naturaleza de las matemáticas; admitimos casi cualquier garabato como la solución de un problema.
La irónica paradoja es que en clase ven las demostraciones que primorosamente reproducimos para que las aprendan —que
no las aprenden, pues no las viven —.
Sin embargo, aún más paradójico es que los matemáticos profesionales y los profesores de matemáticas están escribiendo
matemáticas todo el tiempo.
¿Por qué los alumnos no escriben matemáticas también?
¿Cómo les podemos enseñar matemáticas sin un énfasis profundo y continuado en la escritura?

Una buena escritura es un reflejo de un pensamiento claro.
Un pensamiento deficiente nunca podrá producir una buena escritura.
Demasiado frecuentemente, cometemos el error de confundir familiaridad con conocimiento.
Lo que nos escriben nuestros alumnos en los exámenes es en la mayor parte de los casos una muestra de su familiaridad
con el tema, probablemente adquirida a toda prisa los días previos al examen.
Conocer o entender algo es muy distinto a reconocerlo.
La escritura, por la carga de reflexión que lleva, permite ese
asentamiento, esa vivencia del conocimiento.
He aquí unas cuantas ventajas de la escritura como método de enseñanza:

\begin{enumerate}
    \item Escribir matemáticas hace las clases más activas.
    El alumno tiene que escribir en las clases y mostrar su escritura al resto de la clase, quien hará los comentarios pertinentes para mejorarla.
    \item Escribir matemáticas enfrenta a los alumnos a su propio conocimiento.
    Escribir una demostración correctamente implica un alto nivel de revisión que fuerza a que se aprenda el material con más profundidad.
    \item Escribir siempre fomenta la creatividad, y ello es cierto también en el caso de la escritura matemática.
    \item Escribir matemáticas hará mejores lectores a los alumnos.
    Tendrán que practicar la lectura comprensiva más a fondo.
    \item La entrega de ejercicios escritos al profesor proporciona a este una valiosísima oportunidad de comprobar la comprensión de la materia y reaccionar en consecuencia (bien repitiendo explicaciones, poniendo ejercicios complementarios, dando material adicional a alumnos concretos, etc.).
    \item La escritura matemática, sobre todo si se combina con métodos colaborativos, da lugar a discusiones muy fructíferas entre los alumnos.
\end{enumerate}

Sin embargo, la principal razón para que los alumnos escriban, y lo hagan con rigor y calidad, reside en los valores de las matemáticas.
Los principales valores asociados a las matemáticas son la capacidad para ensanchar y agudizar los mecanismos de aprendizaje, el sentido del conocimiento y el genio del pensamiento profundo.
Enseñar matemáticas a los alumnos a través de la escritura está en clara consonancia con esos valores.
Estos valores, por supuesto, no son privativos de las matemáticas; están presentes también en otras áreas del saber.
